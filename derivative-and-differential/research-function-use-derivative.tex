
\section{利用导数研究函数的性质}
\label{sec:research-function-use-derivative}

\subsection{函数为常数的条件}
\label{sec:condition-for-constant-function}

关于可导函数为常数函数的条件,有如下定理

\begin{theorem}
  某区间上的可导函数的函数值恒保持为某个常数的充分必要条件是,它的导函数为零函数.
\end{theorem}

\begin{proof}[证明]
  先证充分性,如果函数的导函数为常数函数,那么对于定义域上的任何两个不同的实数$x_1,x_2$,存在介于它俩之间的另一实数$\xi$,使得
  \[ \frac{f(x_1)-f(x_2)}{x_1-x_2} = f'(\xi) = 0 \]
  于是$f(x_1)=f(x_2)$,由$x_1,x_2$的任意性,知函数为常数函数,另外,常数函数的导函数为零函数,定理得证。
\end{proof}

进一步有如下结论
\begin{inference}
  如果某区间上的两个可导函数的导函数恒相等,则这两个函数仅相差一个常数。
\end{inference}
这只要考察两个函数的差函数就清楚了。

\subsection{单调性与极值}
\label{sec:research-monotonicity-and-minmax-value}

导数可以用来研究函数的单调性和极值,我们先建立如下关于单调性的定理
\begin{theorem}
  如果函数$f(x)$的一阶导函数$f'(x)$在某个区间上恒满足$f'(x) \geqslant 0$,那么函数在此区间上单调不减,如果不等式反向,即$f'(x) \leqslant 0$恒成立,那么函数在此区间上单调不增。如果不等式中的等号总不成立或者至多仅在有限个点处成立,那么函数相应的是严格单调的。
\end{theorem}

\begin{proof}[证明]
  如果函数在某区间上恒成立$f'(x) \geqslant 0$,那么对于区间上任意两个不同的数$x_1$与$x_2$,设$x_1<x_2$,按照拉格朗日中值定理,存在$\xi \in (x_1,x_2)$,使得
  \[ f'(\xi) = \frac{f(x_1)-f(x_2)}{x_1-x_2} \]
  由$f'(\xi) \geqslant 0$即得$f(x_1) \leqslant f(x_2)$,于是函数单调不减。函数单调不增的证明也是完全类似的。

  显然,如果上面的证明中不等式$f'(x) \geqslant 0$中的等号永远不成立,即永远只能取大于号,那么相应的得出的结论是$f(x_1)<f(x_2)$,即是严格递增的。

  现在来证明,如果导函数$f'(x) \geqslant 0$恒成立,但等号仅在有限个点处取得,那么函数仍然是严格增加的。

  首先可以肯定函数是单调不减的了,把区间划分成多个小区间,使得每个小区间上至多只有一个导函数零点,如果函数在每个小区间上是严格增加的,而连续性又保证了端点处的接续,那么就能得出函数在整个区间上都是严格增加的,所以只需要证明导函数只在一个点处取零点的情形就可以了。

  假定导函数$f'(x)$仅在区间$[a,b]$内的某一点$c$处取零值,采用反证明法来证明函数是严格增加的,若不然,假如存在$x_1,x_2 \in [a,b]$使得$x_1<x_2$且$f(x_1)=f(x_2)$,那么$f'(x_1)$与$f'(x_2)$中至多只有一个为零,而另一个必为正值,不妨设就是$f'(x_1)>0$,于是存在$x_1$的某个右邻域$(x_1,x_1+\delta)$,在此右邻域上恒有$f(x)>f(x_1)=f(x_2)$,只要把$\delta$限制的充分小,就能保证$x_1+\delta<x_2$,于是函数在$(x_1,x_1+\delta)$上的函数值就都大于$f(x_2)$,这与函数的单调不减是矛盾的,从而得证。
\end{proof}

\begin{example}
  关于定理中导函数恒非负,仅在有限个点处取零值也能保证函数的严格增加这一点,函数$y=x^3$提供了一个例子,它的导函数$y'=3x^2 \geqslant 0$,但仅在$x=0$处取零值,并不影响原来函数是严格增加的这一事实。
\end{example}

\begin{example}
需要说明的是,对于函数单调不减与单调不增来讲,条件$f'(x) \geqslant 0$或者$f'(x) \leqslant 0$是充分必要条件(在函数可导的前提下),但对于严格单调来说,不等式成立且至多仅在有限个点处取等号,则只是充分条件而非必要条件,我们会举一个反例加以说明。

在正方形区域$[-1,1]\times[-1,1]$中,用$x=y=\pm \frac{1}{n}(n=1,2,\ldots)$划分网格,并在每一个小正方形区域$[\frac{1}{n+1},\frac{1}{n}] \times [\frac{1}{n+1},\frac{1}{n}]$内,将左下角的角点$\left(\frac{1}{n+1},\frac{1}{n+1}\right)$与右上角的角点$\left(\frac{1}{n},\frac{1}{n}\right)$用一段正弦曲线连接起来,这段正弦曲线是将$y=\sin{x}$在$\left[ -\frac{\pi}{2}, \frac{\pi}{2} \right]$中的部分进行平移和沿坐标轴方向的拉伸变换得来,使得正好被小正方形区域框住,这样一段一段的正弦曲线相接,构成一个全新的函数,在每一个区间$[\frac{1}{n+1},\frac{1}{n}]$上,函数的表达式是:
\[ f(x) = \frac{1}{2} \left( \frac{1}{n+1} + \frac{1}{n} \right) + \frac{1}{2n(n+1)} \sin{\pi n(n+1) \left[ x - \frac{1}{2} \left( \frac{1}{n+1} + \frac{1}{n} \right) \right]} \]
在$x=0$处令$f(0)=0$,然后将曲线中心对称到第三象限,得到另一半图象,从而得出完整的函数图象,显然这个函数是严格增加的,但导函数在每一个$x = \pm \frac{1}{n}$处都取零值,我们再证明$f'(0)=1$就可以了。

设
\[ x = \frac{1}{2} \left( \frac{1}{n+1} + \frac{1}{n} \right) +t, \  |t| \leqslant \frac{1}{2}\left( \frac{1}{n} - \frac{1}{n+1} = \frac{1}{2n(n+1)} \right) \]
于是
\begin{align*}
  \frac{f(x)-f(0)}{x-0} & = \frac{f(x)}{x} \\
                        & = \frac{\frac{1}{2} \left( \frac{1}{n+1} + \frac{1}{n} \right)+\frac{1}{2n(n+1)}\sin{\pi n(n+1)t}}{\frac{1}{2} \left( \frac{1}{n+1} + \frac{1}{n} \right)+t} \\
  & = \frac{2n+1+\sin{\pi n(n+1)t}}{2n+1+2n(n+1)t}
\end{align*}
注意到$|2n(n+1)t| \leqslant 1$,得出
\[ \lim_{x \to 0} \frac{f(x)}{x} = 1 \]
即$f'(0)=1$ .

这个例子表明,即便导函数有无穷多个零点,函数也是有可能严格单调的,那么这个条件到底放宽到何种程度,方能得出充分必要条件呢,这个问题在实变函数的测度理论中方能给出答案。
\end{example}


关于极值,费马极值定理已经表明,极值点处如果可导,则导数只能是零,但是如何判断导函数的零点是否是原来函数的极值点呢,有如下定理
\begin{theorem}
  设函数$f(x)$在$x_0$的邻域内可导,如果它满足以下两条,那么函数在$x_0$处取极值.
  \begin{enumerate}
  \item $f'(x_0)=0$.
  \item 存在$x_0$的某个足够小的邻域,使得函数在两侧空心邻域内各自保持恒定的符号,且两侧的符号正好相反,具体的说,如果左正右负,则函数在$x_0$处取极大值,反之,若左负右正,则函数在$x_0$处取极小值。
  \end{enumerate}
\end{theorem}

\begin{proof}[证明]
 只证明极大值的情况,如果函数$f(x)$满足$f'(x_0)=0$,且在它的某个邻域的左侧$(x_0-\delta,x_0)$上有$f'(x)>0$,而在右侧$(x_0,x_0+\delta)$上有$f'(x)<0$,我们来证明$f(x_0)$是一个极大值,因为在左侧导函数恒正,函数严格增加,同理函数在右侧严格减少,所以$f(x_0)$是一个极大值。极小值也同理可证。
\end{proof}

\begin{example}
  注意这个定理中的条件是充分条件,但不是必要条件(假定函数总是可导的)。实际上,极值的邻域内都并不一定有确定的单调性,可以仿照前面的例子构造相应的反例,此处从略。
\end{example}

\begin{example}[光的折射定律]
  假定光在甲、乙两种介质中的传播速度分别是$v_1$和$v_2$,如图所示,图中直线是两种介质的分界面,现在光从介质甲中的$A$点发出,经分界面折射后,经过介质乙中的$B$点,光在同一种介质中是必定沿直线传播的,我们知道,光总是按照传播用时最短的路径前进,那么问题就来了,光线应该在分界面上何处折射,才能使得传播用时最短?

  设$A$、$B$两点与分界面的距离分别记为$a$、$b$,并且沿分界面的距离是$l$,假定光线到达分界面上的点$P$处,点$P$沿分界面与$A$、$B$的距离分别是$x$和$l-x$,那么光线传播所用的时间是
  \[ f(x) = \frac{\sqrt{x^2+a^2}}{v_1}+\frac{\sqrt{(l-x)^2+b^2}}{v_2} \]
  为了求得最小值,求导并令其为零,得
  \[ f'(x) = \frac{x}{v_1\sqrt{x^2+a^2}} - \frac{l-x}{v_2 \sqrt{(l-x)^2+b^2}} \]
  根据物理意义,这最小值一定存在,设当$x=x_0$时用时最短,则必有$f'(x_0)=0$,于是有
  \[ \frac{x_0}{v_1\sqrt{x_0^2+a^2}} - \frac{l-x_0}{v_2 \sqrt{(l-x_0)^2+b^2}}\]
  设入射角为$\alpha$,折射角为$\beta$,上式即为
  \[ \frac{\sin{\alpha}}{v_1} = \frac{\sin{\beta}}{v_2} \]
  通常用入射角和折射角来标记点$P$的位置,而不是用$x_0$,上式就是光线传播最短路径所应满足的条件,其被称为光的折射定律。
\end{example}

\subsection{证明不等式}
\label{sec:proof-inequality-use-derivative}

\subsection{函数的凸性与拐点}
\label{sec:convert-of-function}

函数的凸性是关于函数图象的拱形特征的刻画,如果函数的图象在区间上向上拱起,则它图象上任意两点间的部分,都位于这两点连线段的上方,从而引出如下定义
\begin{definition}
  如果定义在区间$(a,b)$上的函数$f(x)$对区间上的任意两个数$x_1,x_2$以及任意满足$\alpha+\beta=1(\alpha \geqslant 0,\beta \geqslant 0)$的一对实数$\alpha$、$\beta$都成立
  \[ f(\alpha x_1+\beta x_2) \geqslant \alpha f(x_1) + \beta f(x_2) \]
  则称函数在这区间上是\emph{上凸函数},如果式中的等号总是取不到,则称为\emph{严格上凸函数},把不等式反向,则得到\emph{下凸函数}和\emph{严格下凸函数}的定义。
\end{definition}

例如,幂函数$y=x^n(n \in \mathbb{N}_+)$是$\mathbb{R}$上的下凸函数。

对于连续函数而言,上面的定义与下面这个定义等价
\begin{definition}
  如果区间上的连续函数$f(x)$对区间上的任意$x_1,x_2$都成立
  \[ f \left( \frac{x_1+x_2}{2} \right) \geqslant \frac{f(x_1)+f(x_2)}{2} \]
  则称它是区间上的\emph{上凸函数},类似的可得到严格上凸、下凸、严格下凸的定义。
\end{definition}

即对于连续函数来说,只需要前面定义中$\alpha=\beta=\frac{1}{2}$就够了,这一点在我关于初等数学的笔记中已经证明过,这里从略。

由定义立即可以得到著名的 \emph{琴生不等式}.
\begin{theorem}
  对于某区间上的上凸函数而言,在区间上任意取定$n$个实数$x_i(i=1,2,\ldots,n)$,有
  \[ f \left( \frac{1}{n} \sum_{i=1}^n x_i \right) \geqslant \frac{1}{n} \sum_{i=1}^n f(x_i) \]
\end{theorem}

这定理的证明同样见于我的初等数学笔记,仍然从略。

\begin{example}
  琴生不等式是一系列重要不等式的来源,例如,在后面我们将证明,对数函数$y=\ln{x}$在$(0,+\infty)$上是上凸的,于是套用琴生不等式,对任意$n$个正实数$x_i(i=1,2,\ldots,n)$,就有
  \[ \ln{\left( \frac{1}{n} \sum_{i=1}^n x_i \right)} \geqslant \frac{1}{n} \sum_{i=1}^n \ln{x_i} \]
  即
  \[ \frac{1}{n} \sum_{i=1}^n x_i \geqslant \sqrt[n]{\prod_{i=1}^n x_i} \]
  这就是著名的\emph{均值不等式}.
\end{example}

利用数学归纳法,可以得到
\begin{theorem}
  如果函数$f(x)$在某区间上上凸,那么对于该区间上任意$n$个实数$x_i(i=1,2,\ldots,n)$以及满足
  \[ \sum_{i=1}^n \alpha_i=1(\alpha_i \geqslant 0,i=1,2,\ldots,n) \]
  的一组权值$\alpha_i(i=1,2,\ldots,n)$,成立不等式
  \[ f \left( \sum_{i=1}^n \alpha_i x_i \right) \geqslant \sum_{i=1}^n \alpha_i f(x_i) \]
\end{theorem}

现在我们利用导数工具来研究函数的凸性,自然的,仅限于可导函数的凸性。

\begin{theorem}
  如果函数的导函数在某区间上是单调不增的,那么函数在该区间上是上凸的。
\end{theorem}

\subsection{方程的近似解}
\label{sec:approx-solve-of-equation}





%%% Local Variables:
%%% mode: latex
%%% TeX-master: "../calculus-note"
%%% End:
