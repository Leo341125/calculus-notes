
\section{导数}
\label{sec:derivative}

\subsection{概念与几何意义}
\label{sec:concept-of-derivative}

\begin{definition}
  设函数$f(x)$在$x_0$的某邻域内有定义,如果极限
  \[ \lim_{x \to x_0} \frac{f(x)-f(x_0)}{x-x_0} \]
  存在,则称函数$f(x)$在$x_0$处\emph{可导},此极限值称为$f(x)$在$x_0$处的\emph{导数},并记为$f'(x_0)$,即
  \[ f'(x_0) = \lim_{x \to x_0} \frac{f(x)-f(x_0)}{x-x_0} \]
  如果前述极限不存在,则称函数在该点处不可导. 类似的,把极限过程分别替换为左极限和右极限,就得出\emph{左导数}$f_-'(x_0)$和\emph{右导数}$f_+'(x_0)$的概念,即
  \[ f_-'(x_0) = \lim_{x \to x_0^-} \frac{f(x)-f(x_0)}{x-x_0}, \ f_+'(x_0) = \lim_{x \to x_0} \frac{f(x)-f(x_0)}{x-x_0} \]
\end{definition}

显然,函数在某点处可导的充分必要条件是,该点既左可导又右可导,且左右导数相等。

容易看出,如果在某点可导,则必在该点处连续。

\begin{example}
  常数函数$f(x)=C$,由于函数的增量始终为零,所以它在任意点处都可导,且导数为零。
\end{example}

\begin{example}
  设$n \in \mathbb{Z}$,我们来求幂函数$f(x)=x^n$在$x_0$处的导数,因为
  \[ \frac{x^n-x_0^n}{x-x_0} = x^{n-1}+x^{n-2}x_0+\cdots+xx_0^{n-2}+x_0^{n-1} \]
  上式右端在$x \to x_0$时显然以$nx_0^{n-1}$为极限.
\end{example}

\begin{example}
 我们在 \autoref{example:limit-of-sinx-sinx0-over-x-x0} 中已经证明了下面两个极限:
  \[ \lim_{x \to x_0} \frac{\sin{x}-\sin{x_0}}{x-x_0} = \cos{x_0}, \  
   \lim_{x \to x_0} \frac{\cos{x}-\cos{x_0}}{x-x_0} = -\sin{x_0} \]
 这表明正弦函数和余弦函数在任意点处都是可导的。
\end{example}

\begin{example}
  我们在\autoref{example:function-with-continuous-at-single-point}中利用狄利克雷函数构造了仅在一个点处连续的函数例子:$f(x)=xD(x)$,我们指出,这个函数虽然在$x=0$处连续,但在此处却不可导,因为当$x$分别取有理数值和无理数值并趋于零时,变化率
  \[ \frac{f(x)-f(0)}{x-0} = \frac{f(x)}{x} = D(x) \]
  分别趋于两个不同的极限值,所以不可导,但是如果我们把函数中的无穷小因子升次,变成
  \[ f(x)=x^2D(x) \]
  此时有
  \[ \lim_{x \to 0} \frac{f(x)-f(0)}{x-0} = \lim_{x \to 0} xD(x) = 0 \]
  这表明这个新构造的函数在$x=0$处可导,且导数为零,然而它在除$x=0$之外的其它点处均不连续,所以也就不可导,于是它成为了仅在一个点处连续且可导的函数例子。
\end{example}

如果函数$f(x)$在$x_0$处可导,则在此处有
\[ \frac{\Delta y}{\Delta x} = f'(x_0)+\alpha \]
其中$\alpha$是一个无穷小,于是有
\[ \Delta y = f'(x_0)\Delta x + o(\Delta x) \]
即由一个$\Delta x$的一个线性式子加上一个$\Delta x$的高阶无穷小.这公式可以用于对函数进行估值,只要给定了$x_0$处的函数值以及该点处的导数,就可以近似的估计该点附近的函数值。

下面给出导函数的定义
\begin{definition}
  如果函数$f(x)$在某个区间上有定义,并且在该区间上处处可导(闭区间上的端点处只要单侧可导),则称函数在该区间上可导,而由自变量映射导数值构成一个新的函数$f'(x)$,称为函数$f(x)$的\emph{导函数}.
\end{definition}

\begin{example}
  我们在前面的例子中实际上已经求得了幂函数$f(x)=x^n(n \in \mathbb{Z})$的导函数是$f'(x)=nx^{n-1}$,以及三角函数的导函数:
  \[ (\sin{x})'=\cos{x}, \  (\cos{x})'=-\sin{x} \]
\end{example}

函数在某点处可导,其函数值增量与自变量增量之比是
\[ \frac{f(x)-f(x_0)}{x-x_0} \]
这是函数图象上的点$(x_0,f(x_0))$与其邻近点$(x,f(x))$连线的斜率,即割线的斜率,在$x \to x_0$时,割线将无限的与函数图象在$x_0$处的切线靠近,其极限位置就是该点处的切线,因此,上式的极限便是函数在$x_0$处的切线的斜率,所以如果函数在$x_0$处有导数为$f'(x_0)$,则函数图象在$(x_0,f(x_0))$处的切线方程是(下式中$y_0=f(x_0)$)
\[ y-y_0 = f'(x_0)(x-x_0) \]
这就是导数的几何意义.

\subsection{求导法则}
\label{sec:rule-of-derivative}

\begin{theorem}
  设函数$f(x)$和$g(x)$在$x_0$的某个邻域内有定义且在$x_0$处可导,则函数$u(x)=f(x)+g(x)$和$v(x)=f(x)-g(x)$在$x_0$处也可导,并且
  \[ u'(x_0)=f'(x_0)+g'(x_0), \  v'(x)=f'(x_0)-g'(x_0) \]
\end{theorem}

只要注意到$\Delta u=\Delta f+\Delta g$以及$\Delta v=\Delta f+ \Delta g$即可得知结论成立。

\begin{theorem}
  设函数$f(x)$和$g(x)$在$x_0$的某个邻域内有定义并且在$x_0$处可导,则函数$u(x)=f(x)g(x)$在$x_0$处也可导,并且
  \[ u'(x_0)=f'(x_0)g(x_0)+f(x_0)g'(x_0) \]
\end{theorem}

\begin{proof}[证明]
  因为
  \begin{eqnarray*}
    u(x)-u(x_0) & = & f(x)g(x)-f(x_0)g(x_0) \\
                & = & [f(x)-f(x_0)]g(x)+f(x_0)[g(x)-g(x_0)]
  \end{eqnarray*}
  于是
  \[ \frac{u(x)-u(x_0)}{x-x_0} = \frac{f(x)-f(x_0)}{x-x_0}g(x)+f(x_0)\frac{g(x)-g(x_0)}{x-x_0} \]
  两边令$x \to x_0$便得结论.
\end{proof}

利用数学归纳法,可以把这结论推广到多个函数相乘的情况:
\begin{theorem}
  设有若干个函数$f_i(x)(i=1,2,\ldots,n)$都在$x_0$的某个邻域内有定义,又若它们都在$x_0$处可导,则函数$P(x)=\prod\limits_{i=1}^nf_i(x)$在$x_0$处也可导,并且
  \[ P'(x_0) = \prod_{i=1}^nf_i(x_0) \cdot \sum_{i=1}^n \frac{f_i'(x_0)}{f_i(x_0)} \]
\end{theorem}

\begin{theorem}
  设函数$f(x)$在$x_0$的某邻域内有定义且$f(x) \neq 0$,并且在$x_0$处可导,则函数$u(x)=\dfrac{1}{f(x)}$在$x_0$处也可导,且有
  \[ u'(x) = -\frac{f'(x_0)}{f^2(x_0)} \]
\end{theorem}

\begin{proof}[证明]
  因为
  \begin{eqnarray*}
    \frac{u(x)-u(x_0)}{x-x_0} & = & \frac{\frac{1}{f(x)}-\frac{1}{f(x_0)}}{x-x_0} \\
    & = & - \frac{1}{f(x)f(x_0)} \cdot \frac{f(x)-f(x_0)}{x-x_0}
  \end{eqnarray*}
  令$x \to x_0$取极限即得结论.
\end{proof}

由以上两个定理,只要把两个函数的商函数$\dfrac{f(x)}{g(x)}$写成$f(x)\cdot \dfrac{1}{g(x)}$,就得到
\begin{theorem}
  设函数$f(x)$和$g(x)$都在$x_0$的某个邻域内有定义且$g(x)\neq 0$,如果两者都在$x_0$处可导,则$u(x)=\dfrac{f(x)}{g(x)}$在$x_0$处可导,并且有
  \[ u'(x_0) = \frac{f'(x_0)g(x_0)-f(x_0)g'(x_0)}{g^2(x_0)} \]
\end{theorem}



\subsection{反函数的导数}
\label{sec:derivative-of-revert-function}

\subsection{复合函数的导数}
\label{sec:derivative-of-embed-function}

\subsection{求导公式表}
\label{sec:table-of-derivative-formule}

\subsection{参变量函数的导数}
\label{sec:derivative-of-parametered-function}

\subsection{高阶导数}
\label{sec:high-level-derivative}

\subsection{微分与高阶微分}
\label{sec:difference-and-high-level-difference}



%%% Local Variables:
%%% mode: latex
%%% TeX-master: "../calculus-note"
%%% End:
