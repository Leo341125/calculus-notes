
\section{微分中值定理}
\label{sec:differential-mean-value-theorems}

\subsection{费马极值定理}
\label{sec:fermat-limit-value-theorem}

\begin{theorem}
  设函数$f(x)$在$x_0$的某邻域内有定义,如果它在$x_0$处取得极值(极大或者极小均可)且在$x_0$处可导,则必有$f'(x_0)=0$.
\end{theorem}

\begin{proof}[证明]
  假如函数在$x_0$处取的是极大值,用反证法证明该点处如果可导,则导数只能是零,这是因为,假设$f'(x_0)>0$,则必定存在$x_0$的某个充分小的邻域,在此邻域上恒有
  \[ \frac{f(x)-f(x_0)}{x-x_0} > 0 \]
  于是在这个充分小的邻域的右侧部分,就有$f(x)>f(x_0)$,这与函数在$x_0$处达到极大值相矛盾. 同理,如果$f'(x_0)<0$,则在$x_0$的某个充分小的左邻域上,亦必有$f(x)>f(x_0)$,同样导致矛盾,因此,如果$f'(x_0)$存在,则只能是零。
\end{proof}


\subsection{罗尔定理}
\label{sec:rolle-theorem}

\begin{theorem}[罗尔(Rolle)定理]
  设函数$f(x)$在闭区间$[a,b]$内连续,在开区间$(a,b)$内可导,且有$f(a)=f(b)$,则开区间$(a,b)$上至少存在一个点$x_0$,使得$f'(x_0)=0$.
\end{theorem}

\begin{proof}[证明]
  因为闭区间上的连续函数必定同时存在着最大值和最小值,如果两个最值相等,则函数值恒为常数,此时结论自然是成立的,如果最大值和最小值不相等,则两个最值必然有一个与两个端点处的函数值不同,从而只能在开区间内取得,于是按费马极值定理,该点处的导数便是零。
\end{proof}

\subsection{拉格朗日中值定理}
\label{sec:lagrange-middle-value-theorem}

\begin{theorem}[拉格朗日(Lagrange)中值定理]
  设函数$f(x)$在闭区间$[a,b]$上连续,在开区间$(a,b)$内可导,则存在$x_0 \in (a,b)$,使得
  \[ f'(x_0) = \frac{f(b)-f(a)}{b-a} \]
\end{theorem}

这定理的几何意义是,闭区间上的连续函数的图象上,存在某个点处的切线平行于两个端点的连线。

\begin{proof}[证明]
  两个端点相连直线对应的线性函数是
  \[ g(x) = f(a) + \frac{f(b)-f(a)}{b-a}(x-a) \]
  构造函数
  \[ L(x) = f(x)-g(x) \]
  容易验证,函数$L(x)$满足罗尔定理的条件,由罗尔定理即得结论。
\end{proof}

在证明了拉格朗日定理之后,我们注意到一个的现象,罗尔定理是拉格朗日定理的特殊情形,但由特殊情形的罗尔定理可以得到一般情形的拉格朗日定理,这与我们通常的认识相违背,但这揭示了一种特殊与一般之间的等价关系,这在数学上很多地方都可以看到这种关系,有时候为了证明一个定理,我们往往先证明定理的特殊情形,再由特殊情形通过某种变换,可以得出一般情形的结论。

\subsection{柯西中值定理}
\label{sec:cauchy-middle-value-theorem}

\begin{theorem}[柯西(Cauchy)中值定理]
  设函数$f(x)$和$g(x)$都在闭区间$[a,b]$上连续且在开区间$(a,b)$内可导,并且两个函数的导数不同时为零以及$g(a) \neq g(b)$,则存在$x_0 \in (a,b)$,使得
  \[ \frac{f'(x_0)}{g'(x_0)} = \frac{f(b)-f(a)}{g(b)-g(a)} \]
\end{theorem}

\begin{proof}[证明]
  同拉格朗日定理相仿,作函数
  \[ h(x) = f(a) + \frac{f(b)-f(a)}{g(b)-g(a)}(g(x)-g(a)) \]
  再作
  \[ C(x) = f(x) - h(x) \]
  则可以验证,$C(x)$符合罗尔定理的条件,由罗尔定理即得结论。
\end{proof}

\subsection{导函数的进一步性质}
\label{sec:some-perproties-of-derivative-function}

\begin{theorem}[导函数介值定理]
  设函数$f(x)$在闭区间$[a,b]$内可导,且$f'_+(a) \neq f'_-(b)$,则对于介于$f'_+(a)$与$f'_-(b)$之间的任意实数$k$,都存在$x_0 \in (a,b)$,使得$ f'(x_0) = k $.
\end{theorem}

\begin{proof}[证明]
  作函数$g(x)=f(x)-kx$,则$g'(x)=f'(x)-k$,显然$g'(a)g'(b)<0$,不妨假设是$g'(a)>0$而$g'(b)<0$,则函数$g(x)$在$a$的某个右邻域上单调增加,同时在$b$的某个左邻域上单调减小,因此,连续函数$g(x)$必然在开区间$(a,b)$内的某点处达到它的最大值,记此点为$x_0$,由费马极值定理可知$g'(x_0)=0$,即$f'(x_0)=k$.
\end{proof}

\begin{theorem}
  设函数$f(x)$在$x_0$的某邻域内连续,且在其空心邻域内可导,如果导数函数$f'(x)$在$x_0$处存在左右极限,则函数在$x_0$处存在左导数和右导数,并且有
  \begin{eqnarray*}
   f_-'(x_0) & = & \lim_{x \to x_0^-} f'(x)  \\
   f_+'(x_0) & = & \lim_{x \to x_0^+} f'(x) 
  \end{eqnarray*}
\end{theorem}

\begin{proof}[证明]
  只证明右侧导数的部分,左导数也是完全类似的。设导函数$f'(x)$在$x_0$处的右极限是$A$,只要证明$f(x)$在$x_0$处右可导且导数值也是$A$即可,设$x>x_0$,则函数在$[x_0,x]$上显然满足拉格朗日中值定理的条件,于是存在$x_1 \in (x_0,x)$,使得
  \[ \frac{f(x)-f(x_0)}{x-x_0} = f'(x_1) \]
  当$x \to x_0^+$时,显然亦必有$x_1 \to x_0^+$,而导函数$f'(x)$在$x_0$处有右极限为$A$,所以
  \[ \lim_{x \to x_0^+} \frac{f(x)-f(x_0)}{x-x_0} = \lim_{x_1 \to x_0^+} f'(x_1) = A \]
  这就表明函数$f(x)$在$x_0$处右可导,且导数值为$A$.
\end{proof}

这两个定理表明,导函数虽然不一定是连续函数,但却在一定程度上具有连续函数的某些性质,也正是因此,不是任何一个函数都可以成为某个函数的导函数的。


%%% Local Variables:
%%% mode: latex
%%% TeX-master: "../calculus-note"
%%% End:
