
\section{数列极限}
\label{sec:limit-of-number-sequence}

\subsection{数列极限的概念}
\label{sec:concept-of-limit-for-number-sequence}

在中学数学中,我们熟知反比例函数$f(x)=1/x$的图象在向无穷远处延伸时,它会无限的向横轴靠近,当$x$取正值并无限的增大时,其第一象限的一支会无限的向$x$轴正半轴靠近,但无论$x$取多大,因为$1/x>0$,所以它始终不会与$x$轴相交,这给了我们一种“无限接近但是不会相等”的直观感受。

同样的情况还有许多,我们就准备来详细的讨论下这种“无限接近又不相等”的现象。

上面反比例函数的例子是针对函数而言的,我们先从较为简单的数列开始,同样可以得到数列$x_n=1/n$这个数列,在$n$取正整数并无限增大时,数列的值无限的接近零,但却总是大于零,我们来从这种现象中提取 \emph{极限} 的概念。

首先要指出的是这里“无限接近但不等于”中的“不等于”其实是无关紧要的,例如在数列$1/n$中把下标为偶数的项全部换成零,那么这个"无限"接近并没有被破坏,而且它仍然给我们以极限的印象,只是它在下标增大的过程中,可以无限次的取极限值,而且从这个例子中还可得知,数列的单调性也不是必要的。

我们先给出一个初步的定义: 如果数列$x_n$在随着$n$的无限增大过程中可以无限的接近一个常数$A$,则称$A$是这数列当$n$趋于无穷大时的极限。

这个定义不会令人满意,因为作为数学上的一个定义,它需要具备精确性,而这个定义中出现了“无限接近”这样含义模糊不清的描述,利用这个定义,我们很难说明一个给定常数是否是一个数列的极限,我们需要将它严格化。

所谓数列$x_n$“无限接近”于常数$A$,自然指的是差值$|x_n-A|$可以任意的小,所以我们进行第一步严格化:把数列$x_n$无限接近常数$A$严格化成差值$|x_n-A|$可以任意小,于是极限的定义可以重新叙述为: 对于数列$x_n$和常数$A$,如果数列$x_n$当$n$无限增大时差值$|x_n-A|$可以任意的小,则称常数$A$是数列$x_n$当$n$趋向于无穷大时的极限。

然后我们考虑如何刻画“可以任意的小”,那就是说,差值$|x_n-A|$可以小于任意的正实数$\varepsilon$,而不管这正实数$\varepsilon$有多小。初看起来,“可以小于任意的正实数”,似乎只要存在正整数$n$,使得$|x_n-A|<\varepsilon$就可以了,也就是如下的极限定义: 对于数列$x_n$和常数$A$,如果对于无论多么小的正实数$\varepsilon$,总存在正整数$N$,使得$|x_N-A|<\varepsilon$成立,则称常数$A$是数列$x_n$在下标趋于无穷大时的极限。

这个定义看上去似乎非常符合$1/n$这个数列的特征,不管多么小的正实数$\varepsilon$,总能找到使$1/n < \varepsilon$成立的$n$,只要$n$取的足够大。然而这个定义却有一个严重的问题,我们把数列$1/n$中下标为偶数的项全部换成1,所得到的新数列显然不应该有极限,因为它的奇数下标项趋于零而偶数下标项恒为1,按我们的直观感受,它的值并不无限靠近零,也不无限靠近1,0和1都不应该是它的极限,但是按照上面的定义,它却是符合条件的!

问题出在哪呢,仍然以这个把偶数下标项都替换为1的新数列为例,显然,存在正整数$N$使得$|x_n-A|<\varepsilon$这个条件,是无法保证数列的全部项都向常数$A$靠近的,它只能保证数列中有一部分项会向常数$A$靠近,刚才这个例子也说明了这一点,所以我们需要一个更强的能保证数列的所有项都要向常数$A$靠近,我们把存在正整数$N$使得$|x_N-A|<\varepsilon$成立,改为存在正整数$N$,使得$n>N$时$|x_n-A|<\varepsilon$恒成立,这样一来这个新数列就不满足这条件了,而原来的数列$1/n$却满足这条件。

这个新的条件,利用$n>N$时$|x_n-A|<\varepsilon$恒成立,来保证了数列向$A$靠近的总体趋势。这就是我们最终的极限定义,这个定义,从模糊到精确,别看在这几段话就给出了,实际上在历史上经过了几代数学家的努力,最后才由德国数学家魏尔斯特拉斯(Weierstrass, 1815.10.31-1897.2.19)在总结前人成果的基础上给出,这个精确定义,是人类智慧的结晶。

\begin{definition}
  对于实数数列${a_n}$和实数$a$,如果对于任意小的正实数$\varepsilon$,都存在某一下标$N$,使得该数列在这之后的所有项(即$n>N$)都满足
  \begin{equation}
    \label{eq:the-definition-of-sequence-limit}
    |a_n-a|<\varepsilon
  \end{equation}
  则称该数列存在极限,实数$a$称为该数列的极限。也称该数列为收敛数列,并且收敛到实数$a$,记为
  \begin{equation}
    \label{eq:limit-definition-for-number-sequence}
    \lim_{n \to \infty}x_n = a
  \end{equation}
\end{definition}

极限为零的数列称为无穷小数列,简称\emph{无穷小}。如果数列不存在有限的极限,称为数列为\emph{发散数列}。

如果数列无论对于多大的实数$M>0$,总能从某项开始,后续的全部项都有$a_n>M$,则称数列为\emph{正无穷大}。类似的也有负无穷大和(绝对值)无穷大的概念。

\subsection{一些极限的例}
\label{sec:some-examples-about-limit-of-number-sequence}

\begin{example}
  \label{example:limit-of-1-devide-by-n-power-p}
  前面在提炼极限定义时用了$\dfrac{1}{n}$在$n$无限增大时的情形,现在利用极限定义来证明这个极限:
  \[ \lim_{n \to \infty} \frac{1}{n} = 0 \]
  对于任意小的正实数$\varepsilon$,要使$\dfrac{1}{n}<\varepsilon$,只要$n>\dfrac{1}{\varepsilon}$就可以了,于是可以取$N=1+\left[ \dfrac{1}{\varepsilon} \right]$就可以了,这就证明了它的极限是零,实际上这里的$N$,只要比$\dfrac{1}{\varepsilon}$大就都可以了,以后我们就不专门作取整处理了。

  仿此还可以得到,当$p$是正有理数时,有
  \[ \lim_{n \to \infty} \frac{1}{n^p} = 0 \]
\end{example}

\begin{example}
  \label{example:limit-of-q-power-n}
  设实数$q$满足$|q|<1$,则
  \[ \lim_{n \to \infty} q^n = 0 \]
  对于任意小的正实数$\varepsilon$,要使不等式$|q^n - 0| < \varepsilon$,只要$n>\dfrac{\ln{\varepsilon}}{\ln{q}}$就行了,这就证明得了结论。
\end{example}

\begin{example}
  \label{example:limit-of-1-devide-by-a-power-n}
  设实数$a>1$,则
  \[ \lim_{n \to \infty} \frac{1}{a^n} = 0 \]
  这实际上是\autoref{example:limit-of-q-power-n}的特例,但这里换一种方法来证明它.
  
  对于无论多么小的正实数$\varepsilon$,为了找到极限定义中所要求的$N$,考虑不等式
  \[ \frac{1}{a^n} < \varepsilon \]
  也就是$a^n>1/\varepsilon$,设$a=1+\lambda$,则$\lambda>0$,按二项式定理有\footnote{我们这里并没有从$a^n>1/\varepsilon$中直接使用对数来得出$n>\log_a{(1/\varepsilon)}$,这是因为尽管中学数学中已经学过对数概念,但那时还没有给出无理指数幂的定义,所以指数的定义是不完整的,因此我们无法确认,对于底数$a$,正实数$1/\varepsilon$的对数是否存在,以后我们将在\autoref{sec:irrational-power}中专门讨论指数的定义和值域问题。}
  \[ a^n = (1+\lambda)^n = 1 + n\lambda + \frac{n(n-1)}{2!}\lambda^2+\cdots+\lambda^n > 1+n \lambda \]
  所以只要$1+n\lambda>1/\varepsilon$,便能保证$a^n>1/\varepsilon$成立,也就是只需要$n > (1/\varepsilon-1) / (a-1)$就行了,所以只要选择$N>(1/\varepsilon-1)/(a-1)$就行了,这就证得了此极限。
\end{example}

\begin{example}
  \label{example:limit-of-n-devide-by-a-power-n}
  我们来建立一个基本的极限,对于大于1的正实数$a$,有
  \[ \lim_{n \to \infty} \frac{n}{a^n} = 0 \]
  事实上,仍同\autoref{example:limit-of-1-devide-by-a-power-n}一样,记$a=1+\lambda(\lambda>0)$,则
  \[ a^n=(1+\lambda)^n=\sum_{i=0}^nC_n^i\lambda^i > C_n^2 \lambda^2 \]
  因此有
  \[ \frac{n}{a^n} < \frac{n}{C_n^2 \lambda^2} = \frac{2}{(n-1)\lambda^2} \]
  在$n>2$时,又有$n-1>\dfrac{n}{2}$,所以此时更有
  \[ \frac{n}{a^n} < \frac{4}{n\lambda^2} \]
  所以对于任意小的正实数$\varepsilon$,只要$n>\max\{2,\dfrac{4}{\lambda^2\varepsilon}\}$,便有$\dfrac{n}{a^n} < \varepsilon$,这就证得所要的极限。
\end{example}

\begin{example}
  \label{example:limit-of-lnn-devide-by-n}
  在\autoref{example:limit-of-n-devide-by-a-power-n}的基础上,还可以建立下面的极限
  \[ \lim_{n \to \infty} \frac{\ln{n}}{n} = 0 \]
  对于任意小的正实数$\varepsilon$,为了使得$n$充分大时$\dfrac{\ln{n}}{n} < \varepsilon$,只要使$n<(e^{\varepsilon})^n$就可以了,而根据\autoref{example:limit-of-n-devide-by-a-power-n}中所建立起的极限,这是可以在$n$充分大时恒成立的,故此就建立了此处的极限。
\end{example}

\begin{example}
  \label{example:limit-of-n-sqrt-a-when-a-greater-than-1}
  设实数$a>1$且$a \neq 1$,则 $\lim_{n \to \infty} \sqrt[n]{a} = 1$

  \begin{proof}[证明一]
    利用乘法公式$x^n-1=(x-1)(x^{n-1}+x^{n-2}+\cdots+1)$可得
    \[ \sqrt[n]{a}-1 = \frac{a-1}{(\sqrt[n]{a})^{n-1}+(\sqrt[n]{a})^{n-2}+\cdots+1} < \frac{1}{n}(a-1) \]
   于是对于任意正实数$\varepsilon$,只要取$N>\frac{a-1}{\varepsilon}$便能保证$n>N$时有$0<\sqrt[n]{a}-1<\varepsilon$,所以这极限得证。
  \end{proof}

  \begin{proof}[证明二]
    设$z_n=\sqrt[n]{a}-1$,则
    \[ a = (1+z_n)^n = 1+ nz_n+\frac{1}{2!}z_n^2+\cdots+z_n^n > 1+ n z_n \]
    所以得到
    \[ 0<z_n<\frac{1}{n}(a-1) \]
    下同证明一.
  \end{proof}
\end{example}

\begin{example}
  运用与\autoref{example:limit-of-n-sqrt-a-when-a-greater-than-1}完全一样的手法,还可以得出下面的极限
  \[ \lim_{n \to \infty} \sqrt[n]{n} = 1 \]
\end{example}

\begin{example}
  \label{example:mean-value-of-converge-number-sequence}
  假定数列$a_n$以$A$为极限,我们来考察一下它的前$n$项的算术平均序列
  \[ M_n=\frac{x_1+x_2+\cdots+x_n}{n} \]
  作为一个数列的收敛情况。

  先可以直观的想象一下,数列以$A$为极限,那么数列的项随着下标的无限增大将与$A$无限接近,除前面的项外,越往后的连续片段的算术平均也应该与$A$无限接近,而是越着$n$的无限增大,数列最前面的与$A$相差的项对算术平均的影响也将越来越小,因此猜想$M_n$也收敛到$A$.

  事实上也的确如此,因为对于任意正实数$\varepsilon$,存在正整数$N$,使得$n>N$时恒有$|a_n-A|<\varepsilon$,即$A-\varepsilon<a_n<A+\varepsilon$,于是有
  \[ (n-N)(A-\varepsilon) < a_{N+1}+a_{N+2}+\cdots+a_n < (n-N)(A+\varepsilon) \]
  所以
  \[ \sum_{i=1}^Na_i + (n-N)(A-\varepsilon) < a_{1}+a_{2}+\cdots+a_n < \sum_{i=1}^Na_i + (n-N)(A+\varepsilon) \]
  从而
  \[ A-\varepsilon + \frac{1}{n} \left( \sum_{i=1}^Na_i - N(A-\varepsilon) \right) < M_n <  A+\varepsilon + \frac{1}{n} \left( \sum_{i=1}^Na_i - N(A+\varepsilon) \right)\]
  因为$\lim\limits_{n \to \infty}\dfrac{1}{n} = 0$,所以对于前述正实数$\varepsilon$,令
  \[ \varepsilon_1= \frac{\varepsilon}{\left| \sum_{i=1}^Na_i - N(A-\varepsilon) \right|} \]
  和正实数
  \[ \varepsilon_2= \frac{\varepsilon}{\left| \sum_{i=1}^Na_i - N(A+\varepsilon) \right|} \]
  再令$\varepsilon_0=\max\{\varepsilon_1,\varepsilon_2\}$,则存在正整数$N_0$,使得当$n>N_0$时,$\dfrac{1}{n}<\varepsilon_0$,于是在$n>N_1=\max\{N,N_0\}$时有
  \[ A-2\varepsilon < M_n < A+2\varepsilon \]
  这便说明$\lim\limits_{n \to \infty}M_n = A$,需要提醒的是这里出现的$2\varepsilon$似乎与极限定义不一致,但实际上,只要在这整个过程将$\varepsilon$替换为$\dfrac{1}{2}\varepsilon$,就可以与极限定义完全一致,而且从另一个角度来说,极限定义中只是说$\varepsilon$是任意正实数,那么如果$\varepsilon$是任意正实数,$2\varepsilon$照样可以取遍任意正实数,所以这并没有本质上的不同,关于这一点以后将不再说明了。
\end{example}

\subsection{极限的性质与运算}
\label{sec:properties-and-operation-of-limit}

\begin{theorem}[极限唯一性]
  如果数列$x_n$收敛,则极限唯一。
\end{theorem}

\begin{proof}[证明]
  反证法,假若有两个数$a1$和$a2$都是数列的极限,假定$a_1<a_2$,则对于任意正数$\varepsilon > 0$,数列都能从某项起同时成立着 $|x_n-a_1| < \varepsilon$ 和 $|x_n-a_2| < \varepsilon$,于是取$\varepsilon < (a_2-a_1)/2$,则前述两个不等式因为无交集而产生矛盾。
\end{proof}

\begin{theorem}
  如果数列$x_n$收敛到实数$a$,则对于任意一个小于$a$的实数$x$,数列都能从某项起恒大于$x$,同样对于任意一个大于$a$的实数$y$,数列也能从某项起恒大于$y$。
\end{theorem}

\begin{proof}[证明]
  只要在极限的定义中取 $\varepsilon < a-x$即可得前半部分结论,同样再取$\varepsilon < y-a$即得后半部分结论。
\end{proof}

\begin{inference}[保号性]
  如果数列收敛到一个正的实数,则数列必从某项起恒保持正号,同样,若收敛到一个负的实数,则必从某项起恒保持负号。
\end{inference}

\begin{theorem}[保不等式性]
  设数列$a_n$与数列$b_n$分别收敛到$A$与$B$,且当$n$充分大时恒有$a_n<b_n$,则必有$A \leqslant B$.
\end{theorem}

\begin{proof}[证明]
  反证法,若$A<B$,则取$\varepsilon=\dfrac{1}{2}(B-A)$,在$n$充分大时必同时有$a_n<A+\varepsilon=\dfrac{1}{2}(A+B)$以及$b_n>B-\varepsilon=\dfrac{1}{2}(A+B)$成立,这时显然有$a_n<b_n$,与定理条件矛盾,所以$A \leqslant B$.
\end{proof}

要注意的是由定理中的条件并不能得出$A<B$的结论来,例如$a_n=\dfrac{1}{n^2}$与$b_n=\dfrac{1}{n}$.

\begin{theorem}[夹逼准则,迫敛性]
 若三个数列$a_n$、$b_n$、$c_n$在$n$充分大时恒有$a_n \leqslant b_n \leqslant c_n$,并且$a_n$与$c_n$都收敛到同一极限$M$,则$b_n$亦必收敛到此极限值. 
\end{theorem}

\begin{proof}[证明]
  对任意小的正实数$\varepsilon$,显然当$n$充分大时有
  \[ M-\varepsilon < a_n < b_n < c_n < M+\varepsilon \]
  由此即得定理.
\end{proof}

\begin{theorem}
  如果数列$x_n$和$y_n$分别收敛到$x$和$y$,则数列$x_n+y_n$、$x_n-y_n$、$x_ny_n$、$x_n/y_n$都收敛,而且极限分别是$x+y$、$x-y$、$xy$、$x/y$,在商的情况要求$y \neq 0$。
\end{theorem}

这定理可以推广到任意有限个数列的情形。

\begin{proof}[证明]
  和差的情况是容易证明的,只证明积和商的情况。

  先证明乘积的情形,由
  \begin{equation*}
    |x_ny_n-xy| = |(x_ny_n-xy_n) + (xy_n-xy)| \leqslant |y_n||x_n-x| + |x| |y_n-y|
  \end{equation*}
  任取$\varepsilon > 0$,则存在$N>0$,使得$n>N$时同时恒有$|x_n-x|<\varepsilon$和$|y_n-y|<\varepsilon$\footnote{本来对同一个$\varepsilon$,两个数列的$N$是不同的,但是可以取比这两个$N$都大的$N$,这时就同时有那两个不等式。},另外再由收敛数列的有界性,存在$M>0$,使得$ |y_n| < M$,于是就有 $|x_ny_n-xy| < (M+|x|)\varepsilon$,所以$x_ny_n$收敛到$xy.$

  再来证明商的情况,先证明一个结论,如果数列$y_n$收敛到一个非零实数$y$,那么数列$1/y_n$必收敛,且收敛到$1/y$,这是因为
  \begin{equation*}
    \left| \frac{1}{y_n} - \frac{1}{y} \right| = \left| \frac{y_n-y}{yy_n} \right|
  \end{equation*}
  对于任意正实数$\varepsilon>0$,上式的分子能从某一个下标$N$开始恒小于$\varepsilon$,同时再取另外一个正实数$|y|/2$,数列能从某项起恒有$|y_n|>|y|/2$,于是从某个下标开始,上式就能恒小于$2\varepsilon / y^2$,所以数列$1/y_n$收敛到$1/y$,再将$x_n/y_n$视为$x_n$乘以$1/y_n$并利用乘积的结果,便得商的情形。
\end{proof}

\begin{inference}
  \label{inference:limit-of-power-of-number-sequence}
  若数列$a_n$收敛到实数$A$,则对于固定的整数$m$,数列$a_n^m$收敛到实数$A^m$.
\end{inference}

\begin{example}
  \label{example:limit-of-n-sqrt-a}
  设实数$a>0$且$a \neq 1$,证明极限$\lim_{n \to \infty} \sqrt[n]{a} = 1$.

  我们在 \autoref{example:limit-of-n-sqrt-a-when-a-greater-than-1}中已经证明了$a>1$时的情形,现在假设$0<a<1$,则有
  \[ \lim_{n \to \infty} \sqrt[n]{a} = \lim_{n \to \infty} \frac{1}{\sqrt[n]{\frac{1}{a}}} = 1 \]
\end{example}

\begin{example}
  \label{example:limit-of-n-power-m-devide-by-a-power-n}
  在\autoref{example:limit-of-n-devide-by-a-power-n}中,我们已经证明了下面的极限
  \[ \lim_{n \to \infty} \frac{n}{a^n} = 0 \]
  其中$a>1$,现在我们将它推广,设$m$是一个正整数,则
  \[ \lim_{n \to \infty} \frac{n^m}{a^n} = 0 \]
  这是因为,由于$\sqrt[m]{a}>1$,根据\autoref{example:limit-of-n-devide-by-a-power-n}中的结论,有
  \[ \lim_{n \to \infty} \frac{n}{(\sqrt[m]{a})^n} = 0 \]
 再由\autoref{inference:limit-of-power-of-number-sequence},便得出
  \[ \lim_{n \to \infty} \frac{n^m}{a^n} = 0 \]
  这表明在$n$趋于无限大时,多项式与指数相比,它将变得微不足道,也就是说,指数是比多项式更高阶的无穷大。
\end{example}


\subsection{无穷小与无穷大}
\label{sec:infinite-small-and-great}

\begin{definition}
  如果数列收敛到零,则称它在$n$无限增大时是一个\emph{无穷小}.
\end{definition}

\begin{definition}
  如果数列$a_n$满足: 对任意大的正实数$M$,总存在正整数$N$,使得当$n>N$时恒有$|a_n|\geqslant M$,则称这数列在$n$无限增大时是一个\emph{无穷大}.
\end{definition}

显然,当$a_n$是非零的无穷小时,$\dfrac{1}{a_n}$是无穷大,反之也对。

例如$\dfrac{1}{n}$便是无穷小的例子,而$n^2$则是一个无穷大的例子。

\begin{definition}
   设数列$a_n$与$b_n$都是无穷小,即在$n$无限增大时都收敛到零,\\
  (1). 若$\lim\limits_{n \to \infty} \dfrac{a_n}{b_n} = 0$,则称$a_n$是$b_n$的\emph{高阶无穷小},记作$a_n=o(b_n)$. \\
  (2). 若$\lim\limits_{n \to \infty} \dfrac{a_n}{b_n} = K(\neq 0)$,则称$a_n$与$b_n$是\emph{同阶无穷小},特别的,如果$K=1$,则称它俩是\emph{等价无穷小},记作$a_n \sim b_n$.
\end{definition}

例如,$\dfrac{1}{n^2}$是$\dfrac{1}{n}$的高阶无穷小,这个概念体现了无穷小之间的比较,需要说明的是,两个无穷小之间并不必然有阶的高低之分,因为它俩之比可以不收敛。


关于无穷小,还有如此结论
\begin{theorem}
  如果数列$a_n$是无穷小,数列$b_n$有界,则数列$a_nb_n$是无穷小。
\end{theorem}

\begin{proof}[证明]
  由条件,存在正实数$M$,使得$b_n$中的所有项都满足$|b_n|\leqslant M$,所以$|a_nb_n| \leqslant M |a_n|$总是成立的,而$a_n$为无穷小,所以对于无论多么小的正实数$\varepsilon$,数列$a_n$总能从某项起恒满足$|a_n|<\varepsilon/M$,从而$|a_nb_n|<\varepsilon$,于是结论成立。
\end{proof}

由此知道,数列$\dfrac{\sin{n}}{n}$是无穷小.

\begin{theorem}
  有限个无穷小之和仍是无穷小.
\end{theorem}

利用极限定义即可简单证明,略去。

关于同阶无穷小,则有如下结论
\begin{theorem}
  若$a_n$与$b_n$是同阶无穷小,$c_n$是任一数列,则数列$a_nc_n$与$b_nc_n$的收敛性情况相同,即要么都收敛要么都发散,并且在收敛的情况下,如果又是等价无穷小,则还具有相同的极限值.
\end{theorem}

\subsection{Stolz 定理}
\label{sec:stolz-theorem}

\begin{theorem}[Stolz 定理]
  对于两个数列$x_n$和$y_n$,其中$y_n$是一个严格增加到正无穷或者严格减小到负无穷的数列,如果$\lim\limits_{n\to\infty}\dfrac{x_{n+1}-x_n}{y_{n+1}-y_n} = M$,那么有$\lim\limits \dfrac{x_n}{y_n} = M$,这里的$M$可以是一个实数,也可以是正无穷或者负无穷。
\end{theorem}

\begin{proof}[证明]
  只证明$b_n$严格增加到正无穷以及$l$是一个有限数的情况,此时由两个数列增量之比收敛到$l$,所以对于任意小的正数$\varepsilon$,当$n$充分大时($n \geqslant N$)恒有
  \[ (l-\varepsilon)(b_{n+1}-b_n) < a_{n+1}-a_n < (l+\varepsilon)(b_{n+1}-b_n) \]
  累加可得
  \[ (l-\varepsilon)(b_{n}-b_N) < a_{n}-a_N < (l+\varepsilon)(b_{n}-b_N) \]
  三边同时加上$a_N$再除以$b_n$可得($b_n$必定从某一项开始恒保持正号)
  \[ l-\varepsilon+\frac{a_N-(l-\varepsilon)b_N}{b_n} < \frac{a_n}{b_n} < l+\varepsilon+\frac{a_N-(l+\varepsilon)b_N}{b_n}\]
  因为$b_n$收敛到正无穷,所以当$n$充分大时,上式左右两边以$b_n$为分母的两项的绝对值可以任意小,从而当$n$充分大时就有
  \[ l-2\varepsilon < \frac{a_n}{b_n} < l + 2\varepsilon \]
  这就表明两个数列之比也收敛到$l$.

  要说明的是,上述证明过程中,$N$的取值并不是恒定不变的,每出现一次“当$n$充分大时”之类的字眼,就意味着$N$可能需要取更大的值,以使得前面的不等式与新引入的不等关系能够同时成立。
\end{proof}

\begin{inference}
  如果数列$x_n$收敛到$A$,则$(x_1+x_2+\cdots+x_n)/n$也收敛到$A$,反之亦然。
\end{inference}

这正是我们在\autoref{example:mean-value-of-converge-number-sequence}中所得到的结论,那里是用极限定义证明了这个事实,现在有了 Stolz 定理,它就是一个显而易见的结论了。

\begin{proof}[证明]
  只要在 Stolz 定理中取$a_n=x_1+x_2+\cdots+x_n$,$b_n=n$便得此结论。
\end{proof}


\subsection{单调有界定理}
\label{sec:theorem-of-monotone-bounded}

\begin{theorem}[收敛数列的有界性]
  收敛数列必有界。
\end{theorem}

\begin{proof}[证明]
  这其实从定义就可以得出了,随便取一个$\varepsilon>0$,即知数列从某项起全部落在区间$(a-\varepsilon, a+\varepsilon)$内,这里$a$是数列极限,再扩大此区间把前面的那些项(有限个)包含进来,于是数列便有界。
\end{proof}

\begin{example}
  我们来研究一个利用确界得出极限的例子:已知非负实数数列$\{a_n\}$对任意两个正整数$n$和$m$都满足$a_{n+m} \leqslant a_n+a_m$,求证:数列$\{\frac{a_n}{n}\}$收敛。

这道题目跟1997年的一道CMO试题的条件一模一样,只是结论不同,题目如下: 已知非负实数数列$\{a_n\}$对任意两个正整数$n$和$m$都满足$a_{n+m} \leqslant a_n+a_m$,求证:对任意$n \geqslant m$都有$a_n \leqslant ma_1+\left( \frac{n}{m}-1\right)a_m$。

由于这个关联性,所以原题目的解答过程参考了后者的思路\footnote{见参考文献\cite{olympic-math}.}.

\begin{proof}[证明]
设$n>m$,有
\begin{align*}
\frac{a_n}{n} - \frac{a_m}{m} < & \frac{a_{n-m}+a_m}{n} - \frac{a_m}{m} \\
 = & \frac{n-m}{n} \left( \frac{a_{n-m}}{n-m} - \frac{a_m}{m} \right)
\end{align*}
这有点类似于辗转相除法,如果还有$n-m>m$,则继续上述步骤,经过有限步之后,必然得出
\[ \frac{a_n}{n} - \frac{a_m}{m} \leqslant \frac{s}{n} \left( \frac{a_s}{s} - \frac{a_m}{m} \right) \]
其中正整数$s$满足$1 \leqslant s \leqslant m$,但是这个$s$,一般的说是依赖于$n$的,但是它的取值集合却是有限个,所以右端除去因子$\frac{1}{n}$以外的部分是有界的,所以我们在上式中令$n \to \infty$,便能得出右端在$n \to \infty$时趋于零,这就说明: 对于每一个固定的$m$和任意小的正实数$\delta$,当$n$充分大时恒有
\[ \frac{a_n}{n} \leqslant \frac{a_m}{m} + \delta \]
这便是我们解答问题的关键所在,因为显然还能得出$a_n \leqslant a_{n-1}+a_1 \leqslant (a_{n-2}+a_1)+a_1 \leqslant \cdots \leqslant na_1$,从而$0 \leqslant \frac{a_n}{n} \leqslant a_1$说明数列$\{\frac{a_n}{n}\}$是有上下界的,因此我们只要证明如下这个引理就可以了:

引理: 如果数列$\{ x_n \}$有界,并且对于数列中的每一项$x_m$和任意小的正实数$\delta$,当$n$充分大时恒有$x_n \leqslant x_m+\delta$,那么数列收敛。

证明是很容易的,既然数列有下界,便有下确界,我们将证明这下确界便是其极限,设其下确界为$M$,那么对于无论多么小的正实数$\varepsilon$,存在数列中的某一项$x_m$满足$M \leqslant x_m  < M+\varepsilon$,然而由条件,当$n$充分大时恒有$x_n \leqslant x_m+\delta$,我们让这个$\delta<M+\varepsilon-x_m$,从而这时就有$M \leqslant x_n  < M+\varepsilon$,按极限定义,这下确界$M$便是其极限。
\end{proof}
\end{example}


\begin{theorem}
  单调递增有上界的数列必定收敛,而且收敛到它的上确界。单调递减有下界的数列也类似。
\end{theorem}

\begin{proof}[证明]
  只证明单调递增有上界的情况,假如数列$x_n$就是这样的数列,它的上确界是$M$,则数列中的全部项都满足$x_n \leqslant M$,另外,对于任意小的正实数$\varepsilon$,由上确界定义,总存在某个$x_N$满足$x_N>M-\varepsilon$,再由单调性即知对于$n>N$恒有$M-\varepsilon < x_n \leqslant M < M+\varepsilon$,所以$M$就是这数列的极限。
\end{proof}


\subsection{数$e$}
\label{sec:a-import-sequence-limit}

这一小节我们来证明下面这个数列有极限:
\[ x_n=\left( 1+\frac{1}{n} \right)^n \]

\begin{proof}[证明一]\footnote{这个证明来自参考文献\cite{olympic-math}.}
  由多元均值不等式,把$x_n$看成$n$个$(1+1/n)$的乘积,再添加上一个因数1构成$n+1$个数的乘积,有
  \[ \left( 1+\frac{1}{n} \right)^n = 1 \cdot \left( 1+\frac{1}{n} \right)^n < \left( \frac{1+n\left( 1+\frac{1}{n} \right)}{n+1} \right)^{n+1} = \left( 1+\frac{1}{n+1} \right)^{n+1} \]
  这便表明它是递增的。

  下证它是有上界的,把$n+1$拆分成$\frac{5}{6}$和$n$个$1+\frac{1}{6n}$,由均值不等式得
  \[ n+1 = \frac{5}{6} + n \left( 1+\frac{1}{6n} \right) > (n+1)\sqrt[n+1]{\frac{5}{6} \cdot \left( 1+\frac{1}{6n} \right)^n} \]
  整理即得
  \[ \left( 1+\frac{1}{6n} \right)^n < \frac{6}{5} \]
  所以
  \[ \left( 1+\frac{1}{6n} \right)^{6n} < \left( \frac{6}{5} \right)^6 < 3 \]
  于是由单调性便知
  \[ \left( 1+\frac{1}{n} \right)^n < 3 \]
  所以数列$x_n$单调增加且有上界,故此存在极限。
\end{proof}

\begin{proof}[证明二]\footnote{这个证明来自于参考文献\cite{math-analysis}.}
  把$x_n$按二项式定理展开得
  \begin{eqnarray*}
    x_n & = & \sum_{i=0}^n C_n^i \frac{1}{n^i} \\
    & = & \sum_{i=0}^n \frac{1}{i!}\left( 1-\frac{1}{n} \right) \left( 1-\frac{2}{n} \right)\cdots \left( 1-\frac{i-1}{n} \right)
  \end{eqnarray*}
  易见对于$x_{n+1}$而言,在上式的基础上会多出$i=n+1$的一个正项,并且其它项是把上式中每一个项中的每一个因子$1-\frac{i}{n}$更换为更大的因子$1-\frac{i}{n+1}$,所以$x_{n+1}>x_n$,这是一个递增的数列.

  将每一个项中的所有$(1-i/n)$因子全部放大为1,则有
  \[  x_n < 1+\frac{1}{1!}+\frac{1}{2!}+\cdots+\frac{1}{n!} \]
  接下来有两种放缩方式都可以证明它有上限:
  \[ \frac{1}{k!} < \frac{1}{k(k-1)} = \frac{1}{k-1} - \frac{1}{k} \]
  和
  \[ \frac{1}{k!} < \frac{1}{2^{k-1}} \]
  于是
  \[ x_n < 2 + \sum_{i=2}^n \left( \frac{1}{i-1}-\frac{1}{i} \right) = 3-\frac{1}{n} < 3 \]
  或者
  \[ x_n < 2 + \sum_{i=2}^n \frac{1}{2^{i-1}} = 3-\frac{1}{2^{n-1}} < 3 \]
  所以数列单调递增有上界,故此有极限.
\end{proof}



\subsection{闭区间套定理}
\label{sec:theorem-of-closed-interval-sequence}

\subsection{柯西收敛准则}
\label{sec:cauchy-convergence-rule}

\begin{theorem}[柯西收敛准则]
  数列$x_n$收敛的充分必要条件是,对于任意正实数$\varepsilon$,总存在正整数$N>0$,使得任意$n_1>N$和任意$n_2>N$及任意恒有$|x_{n_1}-x_{n_2}| < \varepsilon$。
\end{theorem}

\begin{proof}[证明]
  先证必要性.

  如果数列$x_n$收敛到$x$,那么对于任意正实数$\varepsilon$,都有正整数$N$,使得$n>N$时恒有$|x_n-x|<\varepsilon / 2$,于是对于任意$n_1>N$及$n_2>N$,便有$|x_{n_1}-x_{n_2}|=|(x_{n_1}-x)- (x_{n_2}-x)|\leqslant |x_{n_1}-x|+|x_{n_2}-x|<\varepsilon / 2+\varepsilon / 2 = \varepsilon$。必要性得证。
\end{proof}

\begin{example}
  前$n$个正整数的平方倒数和
  \[ S_n = 1 + \frac{1}{2^2} + \cdots + \frac{1}{n^2} \]
  对它的片段有
  \begin{eqnarray*}
    S_{m+p}-S_m  & = & \frac{1}{(m+1)^2} + \frac{1}{(m+2)^2} + \cdots + \frac{1}{(m+p)^2} \\
                 & < & \frac{1}{m(m+1)} + \frac{1}{(m+1)(m+2)} + \cdots + \frac{1}{(m+p-1)(m+p)} \\
                 & = & \left( \frac{1}{m} - \frac{1}{m+1} \right) + \left( \frac{1}{m+1} - \frac{1}{m+2} \right) + \cdots + \left( \frac{1}{m+p-1} - \frac{1}{m+p} \right) \\
    & = & \frac{1}{m} - \frac{1}{m+p} < \frac{1}{m}
  \end{eqnarray*}
  所以对于任意正实数$\varepsilon$,只要取$N>1/\varepsilon$,就能保证柯西条件成立,于是数列$S_n$有极限,不过这极限值在此处是求不出来的,在以后我们将会利用无穷级数理论,得到它的极限值,这极限值与圆周率有关:
  \[ \lim_{n \to \infty} \sum_{i=1}^n \frac{1}{i^2} = \frac{\pi^2}{6} \]
\end{example}

\begin{example}
  前$n$个正整数的阶乘的倒数和
  \[ T_n = 1 + \frac{1}{2!} + \cdots + \frac{1}{n!} \]
  它的片段和
  \begin{eqnarray*}
    T_{m+p} - T_m & = & \frac{1}{(m+1)!} + \frac{1}{(m+2)!} + \cdots + \frac{1}{(m+p)!} \\
                  & < & \frac{1}{2^{m+1}} + \frac{1}{2^{m+2}} + \cdots + \frac{1}{2^{m+p}} \\
    & = & \frac{1}{2^m} \left( 1-\frac{1}{2^p} \right) < \frac{1}{2^m}
  \end{eqnarray*}
  可见它也满足柯西收敛条件,所以这个数列也有极限,它的极限便是自然对数的底数$e$:
  \[ \lim_{n \to \infty} \sum_{i=1}^n \frac{1}{i!} = e \]
\end{example}

\subsection{聚点定理}
\label{sec:accumulate-point-theorem}

\subsection{上极限与下极限}
\label{sec:upper-limit-and-lower-limit}



\subsection{无穷级数}
\label{sec:infinite-series}

\subsection{复数数列的极限}
\label{sec:limit-of-complex-number-sequence}



%%% Local Variables:
%%% mode: latex
%%% TeX-master: "../calculus-note"
%%% End:
