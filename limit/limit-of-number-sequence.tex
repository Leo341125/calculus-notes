
\section{数列极限}
\label{sec:limit-of-number-sequence}

\subsection{数列极限的概念}
\label{sec:concept-of-limit-for-number-sequence}

在中学数学中,我们熟知反比例函数$f(x)=1/x$的图象在向无穷远处延伸时,它会无限的向横轴靠近,当$x$取正值并无限的增大时,其第一象限的一支会无限的向$x$轴正半轴靠近,但无论$x$取多大,因为$1/x>0$,所以它始终不会与$x$轴相交,这给了我们一种“无限接近但是不会相等”的直观感受。

同样的情况还有许多,我们就准备来详细的讨论下这种“无限接近又不相等”的现象。

上面反比例函数的例子是针对函数而言的,我们先从较为简单的数列开始,同样可以得到数列$x_n=1/n$这个数列,在$n$取正整数并无限增大时,数列的值无限的接近零,但却总是大于零,我们来从这种现象中提取 \emph{极限} 的概念。

首先要指出的是这里“无限接近但不等于”中的“不等于”其实是无关紧要的,例如在数列$1/n$中把下标为偶数的项全部换成零,那么这个"无限"接近并没有被破坏,而且它仍然给我们以极限的印象,只是它在下标增大的过程中,可以无限次的取极限值,而且从这个例子中还可得知,数列的单调性也不是必要的。

我们先给出一个初步的定义: 如果数列$x_n$在随着$n$的无限增大过程中可以无限的接近一个常数$A$,则称$A$是这数列当$n$趋于无穷大时的极限。

这个定义不会令人满意,因为作为数学上的一个定义,它需要具备精确性,而这个定义中出现了“无限接近”这样含义模糊不清的描述,利用这个定义,我们很难说明一个给定常数是否是一个数列的极限,我们需要将它严格化。

所谓数列$x_n$“无限接近”于常数$A$,自然指的是差值$|x_n-A|$可以任意的小,所以我们进行第一步严格化:把数列$x_n$无限接近常数$A$严格化成差值$|x_n-A|$可以任意小,于是极限的定义可以重新叙述为: 对于数列$x_n$和常数$A$,如果数列$x_n$当$n$无限增大时差值$|x_n-A|$可以任意的小,则称常数$A$是数列$x_n$当$n$趋向于无穷大时的极限。

然后我们考虑如何刻画“可以任意的小”,那就是说,差值$|x_n-A|$可以小于任意的正实数$\varepsilon$,而不管这正实数$\varepsilon$有多小。初看起来,“可以小于任意的正实数”,似乎只要存在正整数$n$,使得$|x_n-A|<\varepsilon$就可以了,也就是如下的极限定义: 对于数列$x_n$和常数$A$,如果对于无论多么小的正实数$\varepsilon$,总存在正整数$N$,使得$|x_N-A|<\varepsilon$成立,则称常数$A$是数列$x_n$在下标趋于无穷大时的极限。

这个定义看上去似乎非常符合$1/n$这个数列的特征,不管多么小的正实数$\varepsilon$,总能找到使$1/n < \varepsilon$成立的$n$,只要$n$取的足够大。然而这个定义却有一个严重的问题,我们把数列$1/n$中下标为偶数的项全部换成1,所得到的新数列显然不应该有极限,因为它的奇数下标项趋于零而偶数下标项恒为1,按我们的直观感受,它的值并不无限靠近零,也不无限靠近1,0和1都不应该是它的极限,但是按照上面的定义,它却是符合条件的!

问题出在哪呢,仍然以这个把偶数下标项都替换为1的新数列为例,显然,存在正整数$N$使得$|x_n-A|<\varepsilon$这个条件,是无法保证数列的全部项都向常数$A$靠近的,它只能保证数列中有一部分项会向常数$A$靠近,刚才这个例子也说明了这一点,所以我们需要一个更强的能保证数列的所有项都要向常数$A$靠近,我们把存在正整数$N$使得$|x_N-A|<\varepsilon$成立,改为存在正整数$N$,使得$n>N$时$|x_n-A|<\varepsilon$恒成立,这样一来这个新数列就不满足这条件了,而原来的数列$1/n$却满足这条件。

这个新的条件,利用$n>N$时$|x_n-A|<\varepsilon$恒成立,来保证了数列向$A$靠近的总体趋势。这就是我们最终的极限定义,这个定义,从模糊到精确,别看在这几段话就给出了,实际上在历史上经过了几代数学家的努力,最后才由德国数学家魏尔斯特拉斯(Weierstrass, 1815.10.31-1897.2.19)在总结前人成果的基础上给出,这个精确定义,是人类智慧的结晶。

\begin{definition}
  对于实数数列${a_n}$和实数$a$,如果对于任意小的正实数$\varepsilon$,都存在某一下标$N$,使得该数列在这之后的所有项(即$n>N$)都满足
  \begin{equation}
    \label{eq:the-definition-of-sequence-limit}
    |a_n-a|<\varepsilon
  \end{equation}
  则称该数列存在极限,实数$a$称为该数列的极限。也称该数列为收敛数列,并且收敛到实数$a$,记为
  \begin{equation}
    \label{eq:limit-definition-for-number-sequence}
    \lim_{n \to \infty}x_n = a
  \end{equation}
\end{definition}

极限为零的数列称为无穷小数列,简称\emph{无穷小}。如果数列不存在有限的极限,称为数列为\emph{发散数列}。

如果数列无论对于多大的实数$M>0$,总能从某项开始,后续的全部项都有$a_n>M$,则称数列为\emph{正无穷大}。类似的也有负无穷大和(绝对值)无穷大的概念。


\begin{example}
  设实数$a>1$,则
  \[ \lim_{n \to \infty} \frac{1}{a^n} = 0 \]

  对于无论多么小的正实数$\varepsilon$,为了找到极限定义中所要求的$N$,考虑不等式
  \[ \frac{1}{a^n} < \varepsilon \]
  也就是$a^n>1/\varepsilon$,设$a=1+\lambda$,则$\lambda>0$,按二项式定理有\footnote{我们这里并没有从$a^n>1/\varepsilon$中直接使用对数来得出$n>\log_a{(1/\varepsilon)}$,这是因为尽管中学数学中已经学过对数概念,但那时还没有给出无理指数幂的定义,所以指数的定义是不完整的,因此我们无法确认,对于底数$a$,正实数$1/\varepsilon$的对数是否存在,以后我们将在\autoref{sec:the-power-of-real-with-rational-exponent}中专门讨论指数的定义和值域问题。}
  \[ a^n = (1+\lambda)^n = 1 + n\lambda + \frac{n(n-1)}{2!}\lambda^2+\cdots+\lambda^n > 1+n \lambda \]
  所以只要$1+n\lambda>1/\varepsilon$,便能保证$a^n>1/\varepsilon$成立,也就是只需要$n > (1/\varepsilon-1) / (a-1)$就行了,所以只要选择$N>(1/\varepsilon-1)/(a-1)$就行了,这就证得了此极限。
\end{example}

\begin{example}
  \label{example:limit-of-n-sqrt-a-when-a-greater-than-1}
  设实数$a>1$且$a \neq 1$,则 $\lim_{n \to \infty} \sqrt[n]{a} = 1$

  \begin{proof}[证明一]
    利用乘法公式$x^n-1=(x-1)(x^{n-1}+x^{n-2}+\cdots+1)$可得
    \[ \sqrt[n]{a}-1 = \frac{a-1}{(\sqrt[n]{a})^{n-1}+(\sqrt[n]{a})^{n-2}+\cdots+1} < \frac{1}{n}(a-1) \]
   于是对于任意正实数$\varepsilon$,只要取$N>\frac{a-1}{\varepsilon}$便能保证$n>N$时有$0<\sqrt[n]{a}-1<\varepsilon$,所以这极限得证。
  \end{proof}

  \begin{proof}[证明二]
    设$z_n=\sqrt[n]{a}-1$,则
    \[ a = (1+z_n)^n = 1+ nz_n+\frac{1}{2!}z_n^2+\cdots+z_n^n > 1+ n z_n \]
    所以得到
    \[ 0<z_n<\frac{1}{n}(a-1) \]
    下同证明一.
  \end{proof}
\end{example}


\subsection{无穷小与无穷大}
\label{sec:infinite-small-and-great}

\subsection{极限的性质与运算}
\label{sec:properties-and-operation-of-limit}

\begin{theorem}[极限唯一性]
  如果数列$x_n$收敛,则极限唯一。
\end{theorem}

\begin{proof}[证明]
  反证法,假若有两个数$a1$和$a2$都是数列的极限,假定$a_1<a_2$,则对于任意正数$\epsilon > 0$,数列都能从某项起同时成立着 $|x_n-a_1| < \epsilon$ 和 $|x_n-a_2| < \epsilon$,于是取$\epsilon < (a_2-a_1)/2$,则前述两个不等式因为无交集而产生矛盾。
\end{proof}

\begin{theorem}
  如果数列$x_n$收敛到实数$a$,则对于任意一个小于$a$的实数$x$,数列都能从某项起恒大于$x$,同样对于任意一个大于$a$的实数$y$,数列也能从某项起恒大于$y$。
\end{theorem}

\begin{proof}[证明]
  只要在极限的定义中取 $\epsilon < a-x$即可得前半部分结论,同样再取$\epsilon < y-a$即得后半部分结论。
\end{proof}

\begin{inference}[极限的保号性]
  如果数列收敛到一个正的实数,则数列必从某项起恒保持正号,同样,若收敛到一个负的实数,则必从某项起恒保持负号。
\end{inference}


\begin{theorem}
  如果数列$x_n$和$y_n$分别收敛到$x$和$y$,则数列$x_n+y_n$、$x_n-y_n$、$x_ny_n$、$x_n/y_n$都收敛,而且极限分别是$x+y$、$x-y$、$xy$、$x/y$,在商的情况要求$y \neq 0$。
\end{theorem}

这定理可以推广到任意有限个数列的情形。

\begin{proof}[证明]
  和差的情况是容易证明的,只证明积和商的情况。

  先证明乘积的情形,由
  \begin{equation*}
    |x_ny_n-xy| = |(x_ny_n-xy_n) + (xy_n-xy)| \leqslant |y_n||x_n-x| + |x| |y_n-y|
  \end{equation*}
  任取$\epsilon > 0$,则存在$N>0$,使得$n>N$时同时恒有$|x_n-x|<\epsilon$和$|y_n-y|<\epsilon$\footnote{本来对同一个$\epsilon$,两个数列的$N$是不同的,但是可以取比这两个$N$都大的$N$,这时就同时有那两个不等式。},另外再由收敛数列的有界性,存在$M>0$,使得$ |y_n| < M$,于是就有 $|x_ny_n-xy| < (M+|x|)\epsilon$,所以$x_ny_n$收敛到$xy.$

  再来证明商的情况,先证明一个结论,如果数列$y_n$收敛到一个非零实数$y$,那么数列$1/y_n$必收敛,且收敛到$1/y$,这是因为
  \begin{equation*}
    \left| \frac{1}{y_n} - \frac{1}{y} \right| = \left| \frac{y_n-y}{yy_n} \right|
  \end{equation*}
  对于任意正实数$\epsilon>0$,上式的分子能从某一个下标$N$开始恒小于$\epsilon$,同时再取另外一个正实数$|y|/2$,数列能从某项起恒有$|y_n|>|y|/2$,于是从某个下标开始,上式就能恒小于$2\epsilon / y^2$,所以数列$1/y_n$收敛到$1/y$,再将$x_n/y_n$视为$x_n$乘以$1/y_n$并利用乘积的结果,便得商的情形。
\end{proof}

\begin{example}
  \label{example:limit-of-n-sqrt-a}
  设实数$a>0$且$a \neq 1$,证明极限$\lim_{n \to \infty} \sqrt[n]{a} = 1$.

  我们在 \autoref{example:limit-of-n-sqrt-a-when-a-greater-than-1}中已经证明了$a>1$时的情形,现在假设$0<a<1$,则有
  \[ \lim_{n \to \infty} \sqrt[n]{a} = \lim_{n \to \infty} \frac{1}{\sqrt[n]{\frac{1}{a}}} = 1 \]
\end{example}

关于无穷小,还有如此结论
\begin{theorem}
  如果数列$a_n$是无穷小,数列$b_n$有界,则数列$a_nb_n$是无穷小。
\end{theorem}

\begin{proof}[证明]
  由条件,存在正实数$M$,使得$b_n$中的所有项都满足$|b_n|\leqslant M$,所以$|a_nb_n| \leqslant M |a_n|$总是成立的,而$a_n$为无穷小,所以对于无论多么小的正实数$\varepsilon$,数列$a_n$总能从某项起恒满足$|a_n|<\varepsilon/M$,从而$|a_nb_n|<\varepsilon$,于是结论成立。
\end{proof}

\subsection{Stolz 定理}
\label{sec:stolz-theorem}

\begin{theorem}[Stolz 定理]
  对于两个数列$x_n$和$y_n$,其中$y_n$是一个严格增加到正无穷或者严格减小到负无穷的数列,如果$\lim\limits_{n\to\infty}\dfrac{x_{n+1}-x_n}{y_{n+1}-y_n} = M$,那么有$\lim\limits \dfrac{x_n}{y_n} = M$,这里的$M$可以是一个实数,也可以是正无穷或者负无穷。
\end{theorem}

\begin{proof}[证明]
  只证明$b_n$严格增加到正无穷以及$l$是一个有限数的情况,此时由两个数列增量之比收敛到$l$,所以对于任意小的正数$\varepsilon$,当$n$充分大时($n \geqslant N$)恒有
  \[ (l-\varepsilon)(b_{n+1}-b_n) < a_{n+1}-a_n < (l+\varepsilon)(b_{n+1}-b_n) \]
  累加可得
  \[ (l-\varepsilon)(b_{n}-b_N) < a_{n}-a_N < (l+\varepsilon)(b_{n}-b_N) \]
  三边同时加上$a_N$再除以$b_n$可得($b_n$必定从某一项开始恒保持正号)
  \[ l-\varepsilon+\frac{a_N-(l-\varepsilon)b_N}{b_n} < \frac{a_n}{b_n} < l+\varepsilon+\frac{a_N-(l+\varepsilon)b_N}{b_n}\]
  因为$b_n$收敛到正无穷,所以当$n$充分大时,上式左右两边以$b_n$为分母的两项的绝对值可以任意小,从而当$n$充分大时就有
  \[ l-2\varepsilon < \frac{a_n}{b_n} < l + 2\varepsilon \]
  这就表明两个数列之比也收敛到$l$.
\end{proof}

\begin{inference}
  如果数列$x_n$收敛到$A$,则$(x_1+x_2+\cdots+x_n)/n$也收敛到$A$,反之亦然。
\end{inference}

\begin{proof}[证明]
  只要在 Stolz 定理中取$a_n=x_1+x_2+\cdots+x_n$,$b_n=n$便得此结论。
\end{proof}


\subsection{单调有界定理}
\label{sec:theorem-of-monotone-bounded}

\begin{theorem}[收敛数列的有界性]
  收敛数列必有界。
\end{theorem}

\begin{proof}[证明]
  这其实从定义就可以得出了,随便取一个$\epsilon>0$,即知数列从某项起全部落在区间$(a-\epsilon, a+\epsilon)$内,这里$a$是数列极限,再扩大此区间把前面的那些项(有限个)包含进来,于是数列便有界。
\end{proof}

\begin{example}
  我们来研究一个利用确界得出极限的例子:已知非负实数数列$\{a_n\}$对任意两个正整数$n$和$m$都满足$a_{n+m} \leqslant a_n+a_m$,求证:数列$\{\frac{a_n}{n}\}$收敛。

这道题目跟1997年的一道CMO试题的条件一模一样,只是结论不同,题目如下: 已知非负实数数列$\{a_n\}$对任意两个正整数$n$和$m$都满足$a_{n+m} \leqslant a_n+a_m$,求证:对任意$n \geqslant m$都有$a_n \leqslant ma_1+\left( \frac{n}{m}-1\right)a_m$。

由于这个关联性,所以原题目的解答过程参考了后者的思路\footnote{见参考文献\cite{olympic-math}.}.

\begin{proof}[证明]
设$n>m$,有
\begin{align*}
\frac{a_n}{n} - \frac{a_m}{m} < & \frac{a_{n-m}+a_m}{n} - \frac{a_m}{m} \\
 = & \frac{n-m}{n} \left( \frac{a_{n-m}}{n-m} - \frac{a_m}{m} \right)
\end{align*}
这有点类似于辗转相除法,如果还有$n-m>m$,则继续上述步骤,经过有限步之后,必然得出
\[ \frac{a_n}{n} - \frac{a_m}{m} \leqslant \frac{s}{n} \left( \frac{a_s}{s} - \frac{a_m}{m} \right) \]
其中正整数$s$满足$1 \leqslant s \leqslant m$,但是这个$s$,一般的说是依赖于$n$的,但是它的取值集合却是有限个,所以右端除去因子$\frac{1}{n}$以外的部分是有界的,所以我们在上式中令$n \to \infty$,便能得出右端在$n \to \infty$时趋于零,这就说明: 对于每一个固定的$m$和任意小的正实数$\delta$,当$n$充分大时恒有
\[ \frac{a_n}{n} \leqslant \frac{a_m}{m} + \delta \]
这便是我们解答问题的关键所在,因为显然还能得出$a_n \leqslant a_{n-1}+a_1 \leqslant (a_{n-2}+a_1)+a_1 \leqslant \cdots \leqslant na_1$,从而$0 \leqslant \frac{a_n}{n} \leqslant a_1$说明数列$\{\frac{a_n}{n}\}$是有上下界的,因此我们只要证明如下这个引理就可以了:

引理: 如果数列$\{ x_n \}$有界,并且对于数列中的每一项$x_m$和任意小的正实数$\delta$,当$n$充分大时恒有$x_n \leqslant x_m+\delta$,那么数列收敛。

证明是很容易的,既然数列有下界,便有下确界,我们将证明这下确界便是其极限,设其下确界为$M$,那么对于无论多么小的正实数$\varepsilon$,存在数列中的某一项$x_m$满足$M \leqslant x_m  < M+\varepsilon$,然而由条件,当$n$充分大时恒有$x_n \leqslant x_m+\delta$,我们让这个$\delta<M+\varepsilon-x_m$,从而这时就有$M \leqslant x_n  < M+\varepsilon$,按极限定义,这下确界$M$便是其极限。
\end{proof}
\end{example}


\begin{theorem}
  单调递增有上界的数列必定收敛,而且收敛到它的上确界。单调递减有下界的数列也类似。
\end{theorem}

\begin{proof}[证明]
  只证明单调递增有上界的情况,假如数列$x_n$就是这样的数列,它的上确界是$M$,则数列中的全部项都满足$x_n \leqslant M$,另外,对于任意小的正实数$\epsilon$,由上确界定义,总存在某个$x_N$满足$x_N>M-\epsilon$,再由单调性即知对于$n>N$恒有$M-\epsilon < x_n \leqslant M < M+\epsilon$,所以$M$就是这数列的极限。
\end{proof}


\subsection{一个重要的数列极限}
\label{sec:a-import-sequence-limit}

这一小节我们来证明下面这个数列有极限:
\[ x_n=\left( 1+\frac{1}{n} \right)^n \]

\begin{proof}[证明一]\footnote{这个证明来自参考文献\cite{olympic-math}.}
  由多元均值不等式,把$x_n$看成$n$个$(1+1/n)$的乘积,再添加上一个因数1构成$n+1$个数的乘积,有
  \[ \left( 1+\frac{1}{n} \right)^n = 1 \cdot \left( 1+\frac{1}{n} \right)^n < \left( \frac{1+n\left( 1+\frac{1}{n} \right)}{n+1} \right)^{n+1} = \left( 1+\frac{1}{n+1} \right)^{n+1} \]
  这便表明它是递增的。

  下证它是有上界的,把$n+1$拆分成$\frac{5}{6}$和$n$个$1+\frac{1}{6n}$,由均值不等式得
  \[ n+1 = \frac{5}{6} + n \left( 1+\frac{1}{6n} \right) > (n+1)\sqrt[n+1]{\frac{5}{6} \cdot \left( 1+\frac{1}{6n} \right)^n} \]
  整理即得
  \[ \left( 1+\frac{1}{6n} \right)^n < \frac{6}{5} \]
  所以
  \[ \left( 1+\frac{1}{6n} \right)^{6n} < \left( \frac{6}{5} \right)^6 < 3 \]
  于是由单调性便知
  \[ \left( 1+\frac{1}{n} \right)^n < 3 \]
  所以数列$x_n$单调增加且有上界,故此存在极限。
\end{proof}

\begin{proof}[证明二]\footnote{这个证明来自于参考文献\cite{math-analysis}.}
  把$x_n$按二项式定理展开得
  \begin{eqnarray*}
    x_n & = & \sum_{i=0}^n C_n^i \frac{1}{n^i} \\
    & = & \sum_{i=0}^n \frac{1}{i!}\left( 1-\frac{1}{n} \right) \left( 1-\frac{2}{n} \right)\cdots \left( 1-\frac{i-1}{n} \right)
  \end{eqnarray*}
  易见对于$x_{n+1}$而言,在上式的基础上会多出$i=n+1$的一个正项,并且其它项是把上式中每一个项中的每一个因子$1-\frac{i}{n}$更换为更大的因子$1-\frac{i}{n+1}$,所以$x_{n+1}>x_n$,这是一个递增的数列.

  将每一个项中的所有$(1-i/n)$因子全部放大为1,则有
  \[  x_n < 1+\frac{1}{1!}+\frac{1}{2!}+\cdots+\frac{1}{n!} \]
  接下来有两种放缩方式都可以证明它有上限:
  \[ \frac{1}{k!} < \frac{1}{k(k-1)} = \frac{1}{k-1} - \frac{1}{k} \]
  和
  \[ \frac{1}{k!} < \frac{1}{2^{k-1}} \]
  于是
  \[ x_n < 2 + \sum_{i=2}^n \left( \frac{1}{i-1}-\frac{1}{i} \right) = 3-\frac{1}{n} < 3 \]
  或者
  \[ x_n < 2 + \sum_{i=2}^n \frac{1}{2^{i-1}} = 3-\frac{1}{2^{n-1}} < 3 \]
  所以数列单调递增有上界,故此有极限.
\end{proof}



\subsection{闭区间套定理与聚点定理}
\label{sec:theorem-of-closed-interval-sequence-and-accumulate-point}

\subsection{柯西收敛准则}
\label{sec:cauchy-convergence-rule}

\begin{theorem}[柯西收敛准则]
  数列$x_n$收敛的充分必要条件是,对于任意正实数$\epsilon$,总存在正整数$N>0$,使得任意$n_1>N$和任意$n_2>N$及任意恒有$|x_{n_1}-x_{n_2}| < \epsilon$。
\end{theorem}

\begin{proof}[证明]
  只证明必要性,充分性的证明放在实数完备性那一节。

  如果数列$x_n$收敛到$x$,那么对于任意正实数$\epsilon$,都有正整数$N$,使得$n>N$时恒有$|x_n-x|<\epsilon / 2$,于是对于任意$n_1>N$及$n_2>N$,便有$|x_{n_1}-x_{n_2}|=|(x_{n_1}-x)- (x_{n_2}-x)|\leqslant |x_{n_1}-x|+|x_{n_2}-x|<\epsilon / 2+\epsilon / 2 = \epsilon$。必要性得证。
\end{proof}

\begin{example}
  前$n$个正整数的平方倒数和
  \[ S_n = 1 + \frac{1}{2^2} + \cdots + \frac{1}{n^2} \]
  对它的片段有
  \begin{eqnarray*}
    S_{m+p}-S_m  & = & \frac{1}{(m+1)^2} + \frac{1}{(m+2)^2} + \cdots + \frac{1}{(m+p)^2} \\
                 & < & \frac{1}{m(m+1)} + \frac{1}{(m+1)(m+2)} + \cdots + \frac{1}{(m+p-1)(m+p)} \\
                 & = & \left( \frac{1}{m} - \frac{1}{m+1} \right) + \left( \frac{1}{m+1} - \frac{1}{m+2} \right) + \cdots + \left( \frac{1}{m+p-1} - \frac{1}{m+p} \right) \\
    & = & \frac{1}{m} - \frac{1}{m+p} < \frac{1}{m}
  \end{eqnarray*}
  所以对于任意正实数$\varepsilon$,只要取$N>1/\varepsilon$,就能保证柯西条件成立,于是数列$S_n$有极限,不过这极限值在此处是求不出来的,在以后我们将会利用无穷级数理论,得到它的极限值,这极限值与圆周率有关:
  \[ \lim_{n \to \infty} \sum_{i=1}^n \frac{1}{i^2} = \frac{\pi^2}{6} \]
\end{example}

\begin{example}
  前$n$个正整数的阶乘的倒数和
  \[ T_n = 1 + \frac{1}{2!} + \cdots + \frac{1}{n!} \]
  它的片段和
  \begin{eqnarray*}
    T_{m+p} - T_m & = & \frac{1}{(m+1)!} + \frac{1}{(m+2)!} + \cdots + \frac{1}{(m+p)!} \\
                  & < & \frac{1}{2^{m+1}} + \frac{1}{2^{m+2}} + \cdots + \frac{1}{2^{m+p}} \\
    & = & \frac{1}{2^m} \left( 1-\frac{1}{2^p} \right) < \frac{1}{2^m}
  \end{eqnarray*}
  可见它也满足柯西收敛条件,所以这个数列也有极限,它的极限便是自然对数的底数$e$:
  \[ \lim_{n \to \infty} \sum_{i=1}^n \frac{1}{i!} = e \]
\end{example}


\subsection{无穷级数}
\label{sec:infinite-series}

\subsection{复数数列的极限}
\label{sec:limit-of-complex-number-sequence}



%%% Local Variables:
%%% mode: latex
%%% TeX-master: "../calculus-note"
%%% End:
