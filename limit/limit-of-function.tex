
\section{函数的极限}
\label{sec:limit-of-function}

\subsection{无穷远处的极限}
\label{sec:limit-of-function-at-infinite-point}

与数列的极限有所区别,函数的极限过程有两大类,一类是自变量趋于正无穷或者负无穷的极限,一类是自变量趋于某个固定点的极限。

\begin{definition}
  设函数$f(x)$在无穷区间$(a,+\infty)$上有定义,$A$是一个实数,如果对于任意小的正实数$\varepsilon$,总存在实数$X>a$,使得$x>X$时恒有$|f(x)-A|<\varepsilon$成立,则称数$A$是函数$f(x)$在自变量$x$趋于正无穷大时的 \emph{极限},记作:
  \[ \lim_{x\to\infty}f(x) = A \]
\end{definition}
类似的可以得函数当自变量趋于负无穷大的极限定义,并且如果函数当自变量趋于正无穷大和负无穷大时都有极限而且极限相同,则称函数当自变量趋于无穷大时有极限,这也可以从绝对值来定义而不考虑自变量的符号。


\subsection{趋于某点的极限及其单侧极限}
\label{sec:limit-of-function-at-point}

当自变量趋于某点的极限定义如下:
\begin{definition}
  设函数$f(x)$在$x_0$的某空心邻域内有定义,$A$是一个实数,如果对于任意小的正实数$\varepsilon>0$,总存在另一正实数$\delta>0$,使得定义域中满足$|x-x_0|<\delta$的数$x$恒有$|f(x)-A|<\varepsilon$,则称$A$是函数$f(x)$当自变量趋于$x_0$时的极限,记作
  \[ \lim_{x\to x_0} f(x) = A \]
\end{definition}
要指出的是,函数$f(x)$在自变量趋于$x_0$时即使收敛,其极限值也并不一定等于$f(x_0)$,实际上函数在$x_0$也并不一定有定义。

考虑到$x$趋于$x_0$的方式,它可以从小于$x_0$的一侧去靠近它,也可以从大于$x_0$的一侧去靠近它,也可以时而在大于$x_0$的一侧,时而位于小于$x_0$的一侧的方式去接近它,所以在这里我们给出 \emph{单侧极限} 的概念。

\begin{definition}
  如果函数$f(x)$在$x_0$的某个右空心邻域内有定义,$A$是一个实数,如果对于任意小的正实数$\varepsilon>0$,都存在另一个正实数$\delta>0$,使得当$x_0<x<x_0+\delta$时恒有$|f(x)-A|<\varepsilon$成立,则称$A$是函数$f(x)$在$x_0$处的 \emph{右极限},记作
  \[ \lim_{x \to x_0^+} f(x) = A \]
\end{definition}
类似的,把右空心邻域改为左空心邻域,把不等式$x_0<x<x_0+\delta$换成$x_0-\delta<x<x_0$,就可以得到 \emph{左极限} 的定义,左极限记作:
  \[ \lim_{x \to x_0^-} f(x) = A \]

  显然,$\lim_{x \to x_0} f(x) = A$的充分必要条件是 $\lim_{x \to x_0^+} f(x) = \lim_{x \to x_0^-} f(x) = A$.

  \subsection{函数极限的性质}
\label{sec:properties-of-function-limit}


与数列极限的性质相仿,函数极限具有类似的性质,以下定理都以$x\to x_0$为例,但它们对于自变量趋于无穷大时的极限也是成立的。

\begin{property}[唯一性]
  函数极限$\lim_{x \to x_0}f(x)$若存在必唯一.
\end{property}

\begin{theorem}[局部有界性]
  设函数$f(x)$在$x_0$的某空心邻域内有定义,若$\lim_{x \to x_0}f(x)$存在(非无穷的有限值),则$f(x)$在$x_0$的某个空心邻域内有界。
\end{theorem}

\begin{theorem}[局部保号性]
  若函数$f(x)$在$x \to x_0$处的极限存在为$A$,则对于任意$r<A$都存在$x_0$的某个空心邻域内,在这邻域内恒有$f(x)>r$,同样,对于任意$r>A$,都存在$x_0$的某空心邻域,在这邻域内恒有$f(x)<r$.
\end{theorem}

\begin{theorem}[保不等式性]
 如果函数$f(x)$和$g(x)$都在$x_0$的某个空心邻内有定义,且在这邻域内上恒满足$f(x) \geqslant g(x)$,那么如果当$x \to x_0$时两个函数分别有极限$A$和$B$,则必有$A \geqslant B$.
\end{theorem}

\begin{theorem}[夹逼定理]
  在$x_0$的某空心邻域内有定义的三个函数$f(x)$、$g(x)$和$h(x)$,如果在这邻域内恒满足$f(x) \leqslant h(x) \leqslant g(x)$,那么如果当$x \to x_0$时$f(x)$和$g(x)$都收敛到$A$,那么这时$h(x)$也必收敛,且也收敛到$A$.
\end{theorem}

\begin{theorem}[四则运算法则]
  如果在$x_0$的某空心邻域内有定义的二函数$f(x)$和$g(x)$在$x \to x_0$时分别收敛到$A$和$B$,那么这由两个函数的和、差、积、商作成的新函数也收敛,并分别收敛到原先两个极限值和和、差、积、商,在商的情况下,要求分母不为零.
\end{theorem}

写成公式就是,在$\lim_{x \to x_0} f(x)$和$\lim_{x \to x_0}g(x)$都存在的前提下,有
\begin{eqnarray*}
  \lim_{x \to x_0} (f(x) \pm g(x)) & = & \lim_{x \to x_0} f(x) \pm \lim_{x \to x_0} g(x)  \\
  \lim_{x \to x_0} f(x)g(x) & = & \lim_{x \to x_0} f(x) \cdot \lim_{x \to x_0} g(x)  \\
  \lim_{x \to x_0} \frac{f(x)}{g(x)} & = & \frac{\lim_{x \to x_0} f(x)}{\lim_{x \to x_0} g(x)}  
\end{eqnarray*}


\subsection{复合函数的极限}
\label{sec:limit-of-composite-function}


关于复合函数的极限,有如下结论
\begin{theorem}
  \label{theorem:limit-of-combine-function}
  设有如下函数极限
  \[ \lim_{x \to x_0}g(x) = u_0, \  \lim_{u \to u_0}f(u) = y_0 \]
  则有
  \[ \lim_{x \to x_0} f(g(x))=y_0 \]
\end{theorem}

\begin{proof}[证明]
  因为当$u \to u_0$时,$f(u) \to y_0$,所以对于无论多么小的正实数$\varepsilon$,总存在另一正实数$\delta$,使得当$|u-u_0|<\delta$时恒有$|f(u)-y_0|<\varepsilon$,而又由于当$x \to x_0$时$g(x) \to u_0$,所以对于前面提到的$\delta$,存在另一正实数$r$,使得当$|x-x_0|<r$时恒有$|g(x)-u_0|<\delta$,从而有$|f(g(x))-y_0|<\varepsilon$,所以最终得到的结论就是:对于无论多么小的正实数$\varepsilon$,总存在另一正实数$r$,使得当$|x-x_0|<r$时,恒有$|f(g(x))-y_0|<\varepsilon$,这就证得结论。
\end{proof}



\subsection{与数列极限的关系}
\label{sec:relation-between-limit-of-function-and-number-sequence}

\begin{theorem}[函数极限与数列极限的关系]
  在$x_0$的某空心邻域内有定义的函数$f(x)$,当$x \to x_0$时存在极限的充分必要条件是,对于任意一个在这空心邻域内取值并以$x_0$为极限的数列$x_n$,$\lim_{n \to \infty}f(x_n)$都存在而且都相等。
\end{theorem}

\begin{proof}[证明]
  先证必要性,如果$\lim_{x \to x_0} = A$,那么对于任意小的正实数$\varepsilon > 0$,都存在另一个正实数$\delta > 0$,使得对这邻域内满足$|x-x_0|<\delta$的实数$x$都成立不等式$|f(x)-A|<\varepsilon$,那么对于任意一个也在这空心邻域内取值并以$x_0$为极限的数列$x_n$,因为它以$x_0$为极限,所以对于这个$\delta>0$,就必然能够从某一项$x_N$开始,后面的所有项都满足$|x_n-x_0|<\delta$,于是就有$|f(x_n)-A|<\varepsilon$,这就表明$f(x_n)$当$n \to \infty$时以$A$为极限,必要性得证。

  再证充分性,如果对于任意一个在这空心邻域内取值并收敛到$x_0$的数列$x_n$,对应的函数值数列$f(x_n)$都收敛到同一实数$A$,我们将证明,函数$f(x)$在$x \to x_0$时也必将收敛到$A$. 采用反证法,假使函数$f(x)$当$x \to x_0$时不以数$A$为极限,那么必然存在某个$\varepsilon_0>0$,使得无论把另一个正实数$\delta>0$限制得多么小,总有满足$|x-x_0|<\delta$的实数$x$能够使得$|f(x)-A| \geqslant \varepsilon_0$成立,于是先取$\delta=1$,得出一个符合这条件的实数$x_1$,然而取$\delta=\min\{\frac{1}{2}, |x_1-x_0|\}>0$,又可以选出$x_2$,依次这样下去,逐个令$\delta_n=\min\{\frac{1}{n}, |x_{n-1}-x_0|\}$,就可以挑选出$x_{n+1}$,这样就作出一个数列$x_n$,由$|x_n-x_0|<\delta_n<\frac{1}{n}$可知$x_n$收敛到$x_0$,但是由于$|f(x_n)-A| \geqslant \varepsilon$恒成立,可知数列$f(x_n)$并不收敛到$A$,这样,我们就证明了如果函数$f(x)$当$x \to x_0$时不以$A$为极限,那么就可以构造出一个以$x_0$为极限的数列$x_n$,使得$f(x_n)$也不以$A$为极限,这与我们的条件是矛盾的,所以充分性得证。
\end{proof}

事实上,如果任意以$x_0$为极限的数列$x_n$,函数值数列$f(x_n)$都收敛的话,这些极限值也必然相同,这是因为,如若不然,假如两个数列$x_n$和$r_n$分别以$A$和$B$为极限,那么在这两个数列中交错的取项构成另一数列$s_n$,显然$s_n$也以$x_0$为极限,而函数值数列$f(s_n)$中的奇数下标子列和偶数下标子列分别以$A$和$B$为极限,由条件知$f(s_n)$应有极限,所以$A=B$.有了这结论,上述定理中的条件可以适当减弱。


\subsection{单调有界定理与柯西收敛准则}
\label{sec:cauchy-convergence-rule-of-function-limit}

与数列的单调有界定理相仿,我们有以下定理
\begin{theorem}
  如果函数$f(x)$在$x_0$的某左空心邻域内单调递增且有上界,则函数$f(x)$在$x_0$处的左极限存在,右极限也有类似的结论。
\end{theorem}

\begin{proof}[证明]
证明很简单,只要在这左邻域内任取一单调增加并以$x_0$为极限的数列$x_n$,则函数值数列$f(x_n)$亦必是单调增加的数列,而它又有上界,所以它有极限,设这极限为$A$,则易证$A$便是函数在这左空心邻域内的上确界,那么对于无论多么小的正实数$\varepsilon$,都存在正整数$N$,使得当$n>N$时恒有$|f(x_n)-A|<\varepsilon$成立,于是取$\delta = x_0-x_{N+1}>0$,则对于任意满足$x_0-\delta<x<x_0$的实数$x$,有$x_N+1<x<x_0$,因而$A-\varepsilon<f(x_{N+1})<f(x) \leqslant A$,这表明$A$就是$f(x)$在$x_0$处的左极限。
\end{proof}

仿照数列的柯西收敛准则,有
\begin{theorem}
  函数$f(x)$在$x_0$的某空心邻域内有定义,则它在该存在极限的充分必要条件是: 任给无论多么小的正实数$\varepsilon$,恒存在另一正实数$\delta$,使得对任意满足$|x-x_0|<\delta$的两个实数$x_1$和$x_2$都成立$|f(x_1)-f(x_2)|<\varepsilon$.
\end{theorem}

\begin{proof}[证明]
  必要性是容易证明的,略去,下证充分性,取正实数数列$\varepsilon_n=1/n(n=1,2,\ldots)$,存在另一单调递减的正实数数列$\delta_n$,使得对于任意满足$|x_1-x_0|<\delta_n$和$|x_2-x_0|<\delta_n$的$x_1,x_2$都成立$|f(x_1)-f(x_2)|<\varepsilon_n$,于是函数$f(x)$在$x_0$的以$\delta_n$为半径的空心邻域内的函数值都必将限于一个长度为$\varepsilon$的闭区间$[m_n,M_n]$上,显然这个闭区间序列符合闭区间套定理的条件,所以有唯一实数$A$从属于所有的闭区间,显然这实数$A$也是函数$f(x)$在$x_0$处的极限。
\end{proof}


\subsection{两个重要的函数极限}
\label{sec:two-important-function-limit}

这一节的主要任务是建立两个重要极限.

\begin{theorem}
  \label{theorem:sinx-over-x-to-1-when-x-to-0}
  对于正弦函数,有如下极限
  \[ \lim_{x \to 0} \frac{\sin{x}}{x} = 1 \]
\end{theorem}

\begin{proof}[证明]

  如\autoref{fig:limit-of-sinx-over-x-at-0},在单位圆中,点$A$和$C$是圆周上两点,并且$\angle AOC$的弧度值为$x$,$x$是锐角,$OC$与点$A$处的切线相交于点$B$,由图可得
  
\begin{figure}[htbp]
\centering
\includegraphics{limit/pic/limit-of-sinx-over-x-at-0.pdf}
\includegraphics{limit/pic/graphy-of-function-sinx-over-x.pdf}
\caption{}
\label{fig:limit-of-sinx-over-x-at-0}
\end{figure}

  \[ S_{\triangle AOC} < S_{\text{扇形}AOC} < S_{\triangle AOB} \]
  于是便得
  \[ \sin{x} < x < \tan{x} \]
  从而
  \[ \cos{x} < \frac{\sin{x}}{x} < 1 \]
  而当$x \to 0$时, $\cos{x} \to 1$,所以最终得极限
  \[ \lim_{x \to 0} \frac{\sin{x}}{x} = 1 \]
\end{proof}

函数$f(x)=\sin{x}/x$在$x=0$附近的图象如\autoref{fig:limit-of-sinx-over-x-at-0}所示,它的图象反得在函数$y=1/x$和$y=-1/x$之间振荡,在$x=0$附近,函数值趋于1,但它在$x=0$处并无定义。

我们在数列极限的部分曾经证明过下面这个数列存在极限
\[ x_n = \left( 1+\frac{1}{n} \right)^n \]
并把它的极限记作$e$,今来把它推广成函数的极限,这就是以下的定理:
\begin{theorem}
  \[ \lim_{x \to \infty} \left( 1+\frac{1}{x} \right)^x = e \]
  这里$e$是自然对数的底数.
\end{theorem}

\begin{proof}[证明]
  设实数$x$的整数部分为$n$,即$n \leqslant x < n+1$,有
  \[ \left( 1+\frac{1}{n+1} \right)^n < \left( 1+\frac{1}{x} \right)^x < \left( 1+\frac{1}{n} \right)^{n+1} \]
  显见左右两边作为两个数列,都以$e$为极限,我们把这左右两边转化为两个函数,即设$x$的整数部分为$n$,定义
  \[ h_1(x) =  \left( 1+\frac{1}{n+1} \right)^n, \ h_2(x)=\left( 1+\frac{1}{n} \right)^{n+1} \]
  于是就有
  \[ \lim_{x \to +\infty} h_1(x) = \lim_{x \to +\infty} h_2(x) = e \]
  所以由夹逼准则即得
  \[ \lim_{x \to +\infty} \left( 1+\frac{1}{x} \right)^x = e \]
  这是函数在正无穷远处的极限,再来看负无穷处的极限,当$x$以负值趋于负无穷时,令$x=-t$,则$x$趋于负无穷等价于$t$趋于正无穷,而
  \[ \left( 1+\frac{1}{x} \right)^x = \frac{1}{\left( 1-\frac{1}{t} \right)^t} = \left( 1+\frac{1}{t-1} \right)^t = \left( 1+\frac{1}{t-1} \right)^{(t-1) \cdot \frac{t}{t-1}} \]
  利用复合函数的极限结果(\autoref{theorem:limit-of-combine-function}),便知当式右端当$t \to +\infty$时的极限是$e$,所以当$x \to -\infty$时左端的极限也就是$e$,所以最终当$x \to \infty$时,函数$(1+1/x)^x$的极限都是$e$.
\end{proof}


\subsection{无穷小与无穷大}
\label{sec:infinite-small-and-great}

\subsection{曲线的渐近线}
\label{sec:asymptotic-line-of-curve}



%%% Local Variables:
%%% mode: latex
%%% TeX-master: "../calculus-note"
%%% End:
