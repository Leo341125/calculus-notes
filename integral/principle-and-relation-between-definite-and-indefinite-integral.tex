
\section{定积分与不定积分的原理及两者之间的关系}
\label{sec:principle-and-relation-between-definite-and-indefinite-integral}

\subsection{黎曼和}
\label{sec:riemann-sum}

\subsection{定积分概念}
\label{sec:concept-of-definite-integral}

\subsection{可积条件}
\label{sec:integrable-function}

\subsection{定积分的性质}
\label{sec:properties-of-definite-integral}


\subsection{变动上限的积分函数}
\label{sec:variable-upper-limit-integral-function}

证明出:变动上限的连续函数的积分,其对于变动上限的导函数就是被积函数自己

\subsection{不定积分概念与性质,基本积分表}
\label{sec:indefinite-integral}

\subsection{牛顿-莱布尼茨公式}
\label{sec:newton-leibniz-formular}











%%% Local Variables:
%%% mode: latex
%%% TeX-master: "../calculus-note"
%%% End:
