
\section{定积分与不定积分的原理及两者之间的关系}
\label{sec:principle-and-relation-between-definite-and-indefinite-integral}

在初等范围内,我们经常见到一个量对另一个量的累积(乘积),例如物体的面积是纵横两个方向上的累积,路程是速度对时间的累积,功是力对位移的累积,等等。但是以前我们通常只会处理最简单的情况,我们只会求规则图形的面积,不会计算不规则图形的面积,只会处理匀速运动的路,不会计算任意变速运动的路程,只会计算恒定力做功,不会计算变力做功,定积分就是为了处理这类问题而被发现的。

后来牛顿与莱布尼茨又发现了定积分与函数的原函数之间存在着直接又简单的联系,与是积分学与微积分学之间的深刻关系也被揭示在世人面前,其影响之大,使得这个结论直接被冠之以微积分学基本定理,成为微积分学的基石。

\subsection{黎曼和与定积分的概念}
\label{sec:riemann-sum-and-concept-of-definite-integral}

我们先来看几个例子。

\begin{example}[曲边梯形的面积]
\begin{figure}
  \centering
  \includegraphics[scale=0.7]{integral/pic/area-of-curvilinear-trapezoid.pdf}
  \caption{曲边梯形的面积}
  \label{fig:area-of-curvilinear-trapezoid}
\end{figure}

如 \autoref{fig:area-of-curvilinear-trapezoid} 所示,定义在区间 $[a,b]$ 上的正值函数 $y=f(x)$的图象与直线$x=a$、$x=b$以及 $x$ 轴围成了一个曲边梯形,我们考虑它的面积,也就是函数图象下方的面积。

通过在区间$[a,b]$内插入一些点把区间 $[a,b]$ 划分成 $n$ 个小区间(不必是等分):
\[ a=x_0<x_1< \cdots < x_{i-1} < x_i < \cdots < x_n = b \]
这样曲边梯形就被划分成了$n$个小的曲边梯形,设曲边梯形面积是 $S$,第 $i$ 个区间上的曲边梯形面积是 $S_i$,那么
\[ S = \sum_{i=1}^n S_i \]
因为每个小区间上的曲边梯形比较小,所以我们用矩形的面积来近似其面积,在每个小区间上任意各取定一个点:$\xi_i\in [x_{i-1}, x_i](i=1,2,\ldots,n)$,用 $y_i=f(\xi_i)$来作矩形的上边界,得到一个小矩形,其面积是 $f(\xi_i) \Delta x_i$,这里 $\Delta x_i = x_i-x_{i-1}$ 是第 $i$ 个小区间的长度,于是
\[ S_i \approx f(\xi_i) \Delta x_i \]
那么曲边梯形的总面积,就可以用这$n$个小矩形的面积之和来近似代替:
\[ S = \sum_{i=1}^n S_i \approx \sum_{i=1}^n f(\xi_i) \Delta x_i \]
显然,小区间越多,各个小区间长度越小,这个误差就越小,考虑划分的极限情况,即各小区间的最大长度趋于零时,上式右端将以左边为极限。
\end{example}

\begin{example}[变速直线运动的位移]
  假定某质点做变速直线运动,其速度是时间的函数$v=v(t)$,现在我们要计算它从$t=a$到$t=b$这段时间段内发生的位移. 如果速度恒定不变,那么只要将速度与时间段长度相乘即可得出结果,但是现在速度是个时时刻刻都在改变的量,这个办法行不通了,但是我们可以将该时间段划分成很多很小的时间内,在每一个很小的时间内,速度的改变量很小,因此可以近似的看成是匀速运动,从而质点在这很小的时间内所发生的位移是可以近似求出的,于是总的位移也就可以近似求出。

  在时间段$t\in[a,b]$内插入$n-1$个时间点将其划分为$n$个小的时间区间(不必等分):
  \[ a = t_0 < \cdots < t_{i-1} < t_i < \cdots < t_n = b \]
  在每个小区间段内随机取一个点$t=\xi_i$,那么质点在该小的时间段内可以视为以大小为$v(\xi_i)$的速度匀速运动,于是在这个小区间上发生的位移$S_i$便有近似
  \[ S_i \approx v(\xi_i) \Delta t_i \]
  这里$\Delta t_i = t_i-t_{i-1}$是该小时间段的长度.于是质点在区间$[a,b]$内所发生的总位移有如近似:
  \[ S = \sum_{i=1}^n S_i \approx \sum_{i=1}^n v(\xi_i) \Delta t_i \]
  如果时间段划分得越小,那么上式的误差就越小,在各小时间段的最大长度趋于零时,误差即趋于零。
\end{example}

\begin{example}[变力做功]
  考虑一个变化的力$F$推动某物体移动了一段直线路程$L$,我们把这一段直线路程对应为区间$[a,b]$,这里$L=b-a$,而变力$F$在该路程每个位置处的大小是$F=F(l), l\in[a,b]$,同样的,我们把这段路程划分成很多小区间:
  \[ a = l_0 < \cdots l_{i-1} < l_i < \cdots < b = l_n \]
  在每个小区间上可以近似是恒力做功(只要区间划分得很小,那么在小区间上力的改变量也可以很小),在每个小区间上任意取一点$\xi \in [x_{i-1},x_i]$,在这一段小区间上可以视为以恒力$F_i=F(\xi_i)$进行的恒力做功,因此这一段小区间上做的功有近似:
  \[ W_i \approx F(\xi_{i}) \Delta l_i \]
  其中$\Delta l_i = l_i-l_{i-1}$是小区间长度.变力$F$在区间$[a,b]$内总共所做的功
  \[ W = \sum_{i=1}^n W_i \approx \sum_{i=1}^n F(\xi_i) \Delta l_i \]
\end{example}

总结上面三个例子,有很多情况下我们需要求出一个量$A$对另一个量$B$的累积$P$(即乘积),如果量$A$不依赖于量$B$的变化而变化,那么我们可以很容易的得出这个乘积:
\[ P = A \cdot \Delta B = A (B'-B_0) \]
但是通常量$A$它不是恒定不变的,而是依赖于量$B$的,或者说,它是量$B$的函数$A=A(B)$,这时为了求出这个累积量,可以通过将量$B$的变化区间$[B_0,B']$划分成许多小区间,然后在每个小区间上将量$A$视为恒定不变的,从而该小区间上的累积量$P_i$可以近似得出,于是在整个区间上的累积量$P$也就可以用它们之和来近似代替,只要把区间划分得足够细,就可以使得误差足够小,在小区间的最大长度趋于零的极限情况,就得出了累积量的精确值了,从这个意义上说,定积分其实就是一种广义的乘法运算。

这里需要说明的一点是,这个近似的误差,是在小区间的最大长度趋于零的情况下才会足够小,而不是小区间数目趋于无穷的情况下,因为即便小区间数目再多,也不能保证有个别区间长度比较大,从而该小区间上的误差无法解决,例如用$x_i = \frac{1}{i}$来划分区间$[0,1]$,这一点务必注意。

现在,是时候给出定积分的定义了:
\begin{definition}[黎曼和与定积分]
  设函数$f(x)$在闭区间$[a,b]$上有定义,在该闭区间上插入一系列分点:
  \[ a = x_0 < \cdots < x_{i-1} < x_i < \cdots < x_n = b \]
  将之划分为$n$个小区间,并在每一个小区间$[x_{i-1},x_i]$上任意取一个点$\xi_i\in [x_{i-1},x_i]$,作和式(称为\emph{黎曼和})
  \[ \sum_{i=1}^n f(x_i) \Delta x_i \]
  这里$\Delta x_i = x_i-x_{i-1}$是第$i$个小区间的长度。如果存在一个数$P$,它满足:对于任意小的正实数$\varepsilon >0$,都存在另一个正实数$\delta >0$,使得所有划分中,只要它的最大小区间的长度小于$\delta$,那么无论怎么选取各个小区间上的$\xi_i$的值,都有:
  \[ \left| \sum_{i=1}^n f(\xi_i) \Delta x_i - P \right| < \varepsilon \]
  那么就称函数$f(x)$在闭区间$[a,b]$上是\emph{可积的},而数$P$,就称为是函数$f(x)$在闭区间$[a,b]$上的\emph{定积分}值,用下面符号表示:
  \[ P = \int_a^b f(x)dx \]
\end{definition}

关于这个定义,作几点说明.

定义的后半部分其实就是极限的 $\varepsilon - \delta$ 语言,为什么不直接使用极限符号呢,因为这里有一个问题是,极限符号只能体现出小区间的最大长度趋于零,但却体现不出各小区间上的点$\xi_i$取法的任意性。

定积分的这个符号,其实它就是黎曼和的极限形式,黎曼和的$\Sigma$直接变成了被拉长的字母S(sum代表求和),黎曼和上下标变成了区间的上下限,而黎曼和中的区间长度$\Delta x_i$变成了自变量$x$的微分,可以理解为无限小的区间长度,这么一看,定积分符号就是黎曼和的极限形式就容易理解了。

\subsection{可积条件}
\label{sec:integrable-function}

\subsection{定积分的性质}
\label{sec:properties-of-definite-integral}


\subsection{变动上限的积分函数}
\label{sec:variable-upper-limit-integral-function}

证明出:变动上限的连续函数的积分,其对于变动上限的导函数就是被积函数自己

\subsection{不定积分概念与性质,基本积分表}
\label{sec:indefinite-integral}

\subsection{牛顿-莱布尼茨公式}
\label{sec:newton-leibniz-formular}











%%% Local Variables:
%%% mode: latex
%%% TeX-master: "../calculus-note"
%%% End:
