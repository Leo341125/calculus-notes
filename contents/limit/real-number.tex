
\section{实数理论}
\label{sec:real-number-theory}

分析学的基础建立在实数的公理化体系之上,在讨论极限理论之前,先来讨论一下实数的理论。

\subsection{实数的十进制表示与大小关系}
\label{sec:decimal-system}

在人类历史上,为了计数而引进了自然数,最初以算筹的数量代表对应的数字,但这对于较大的数比较困难,为了表示数100就需要100根算筹,于是发明了十进制,这样所需的算筹数量就大大减少,之所以是十进制很可能是因为人正好有十根手指头,便于比划数字。后来为了解决多人平分食物等生活资料的问题又引进了整数之比即有理数的概念,再往后毕达哥拉斯学派根据勾股定理,发现了边长为1的正方形的对角线的长度不是有理数,引发了第一次数学危机,这次危机随着无理数的引入而得以解决。有理数与无理数一起,构成了全体实数。但在实数范围内,像$x^2+1=0$这样的代数方程没有解,为了从理论上解决这个问题而引入了虚数的概念,实数与虚数一起构成了复数,代数方程的理论在复数范围内得到彻底的解决。

本节只讨论实数。在十进制下,一个实数$x$具有如下表示:
\begin{equation}
  \label{eq:decimal-format-of-real}
 x=a_na_{n-1}\cdots a_1a_0.a_{-1}a_{-2}\cdots 
\end{equation}
其中$a_i \in \{0,1,2,3,4,5,6,7,8,9 \}$,并且最左边的数位$a_n$非零(否则省略这一位不写),十进制就是说,这个式子表示的数值其实是
\[ x=10^na_n+10^{n-1}a_{n-1}+\cdots+10a_1+a_0+\frac{a_{-1}}{10}+\frac{a_{-2}}{10^2}+\cdots \]
即\autoref{eq:decimal-format-of-real}实际表示的数值是它每一位数字与该数位上的权值之积的和,这一点是十分重要的,因为这样我们就只需要$0-9$这十个数符就可以表示出任意实数,而不必为每一个数都去发明一个对应的数符,那样既是不可能的,也是很难使用的。

在这种表示下,数位$a_0$称为个位,$a_1$称为十位,$a_2$称为百位,依次类推,在$a_0$以后的部分称为小数部分,$a_0$以及$a_0$左边的部分称为整数部分,两部分之间用小数点来分隔出明确的界限。

需要说明的是,实数十进制表示的小数部分是可以无限延伸的,但整数部分只能是有限位,并且规定,如果小数部分从某一位起全部都是零,则可以省写这些零,这样的小数称为有限小数,否则便称为无限小数。

如果无限小数的小数部分有连续重复出现的片段,例如 $0.12345678678678678\cdots$,这以后的数位全是重复的片段$678$,就称这小数为循环小数,并简写为$0.12345\dot{6}7\dot{8}$,即在循环片段的首尾两个数字上加点。如果没有这样的连续重复出现片段,则称为无限不循环小数。

关于整数的一个极为深刻的结论是
\begin{theorem}[带余除法]
  对任意两个整数$a$和$b$,其中$b$为正整数,则存在唯一一对整数$q$与$r(0\leqslant r < b)$,使得$a=qb+r$成立.这整数$q$及$r$分别称为$a$除以$b$所得的\emph{商}和\emph{余数}.
\end{theorem}

\begin{proof}[证明]
  以$b$的倍数为界点将全体实数划分为区间序列$\ldots,[-2b,-b),[-b,0),[0,b),[b,2b),\ldots$,这些左闭右开区间两两无交集,且它们的并集就是全体实数,那么整数$a$必定从属于其中某一个区间,假定是$[mb,(m+1)b)$,则取$q=m,r=a-mb$即满足定理条件,反过来,如果还有另一组$q_1$及$r_1$满足定理中条件,那么有$q_1b \leqslant a < (q_1+1)b$,这即表明$q_1=m$,从而$r_1=a_{mb}$,这就证得了商及余数的唯一性。
\end{proof}

利用带余除法,可以证明
\begin{theorem}
  有理数都是无限循环小数.
\end{theorem}

\begin{proof}[证明]
  设有理数$\frac{a}{b}$,其中$a$与$b$是整数,由于这结论与数的符号无关,所以假定这分子分母还是正的。这个证明过程其实就是两个正整数做除法的过程,思路就是在这个除法过程中,每一步所得的余数,或者是零从而被除尽,或者便要重复出现.

  先用$a$除以$b$,记商与余数分别为$q$及$r$,即$a=qb+r(0\leqslant r < b)$,如果$r>0$,再用$10r$除以$b$,所得的商与余数分别记为$q_1$与$r_1$,如果仍然有$r_1>0$,则再将$10r_1$除以$b$得到商$q_2$与余数$r_2$,依次类推,得到序列$q_i$与$r_i$,这时有$q_i(i \leqslant 1)$只能取$0$到$9$中的数字,这是因为$10r_{i-1}=q_ib+r_i$,而$0 \leqslant r_{i-1} < b$,所以$q_i$不能超过9,而由于$0 \leqslant r_i < b$,所以$r_i$也只能在集合$\{0,1,2,\ldots,b-1\}$这个有限集中取值,如果某一次取到了零$r_m=0$,则这个除法过程就结束了,而最终有
  \[ \frac{a}{b} = q + \sum_{i=0}^{m-1}\frac{q_i}{10^i} = q.q_1q_2\cdots q_{m-1} \]
  即为有限小数。如果$r_i$始终不能取到零,那么必然存在某个$i$及$j(> i)$使得$r_i=r_j$,既然出现了相同的余数,那么在分别用$10r_i$和$10r_j$去除以$b$时也会得出相同的商$q_{i+1}$和$q_{j+1}$,于是进一步出现相同的$r_{i+1}$与$r_{j+1}$,这个过程将无限重复下去,这时就有
  \[ \frac{a}{b} = q.q_1q_2 \cdots q_iq_{i+1} \cdots q_jq_{j+1} \cdots \]
  这里从$q_i$到$q_{j-1}$便是一个重复片段,为小数的循环部分(不一定是最小循环片段),即为无限循环小数。
\end{proof}

\subsection{最小自然数原理}
\label{sec:minimum-nature-number-principle}

\subsection{确界定理}
\label{sec:least-bound-theorem}

对于一个实数集,如果存在实数$M$,使得集合中的全部数$x$都满足$x \leqslant M$,则称实数$M$是这数集的一个\emph{上界},如果不等式是反向的,则称这实数是这数集的一个\emph{下界},显然,如果$M$是某个数集的上界,则比$M$大的所有实数也都是这数集的上界,对下界亦有类似结论。

如果数集既有上界又有下界,则称数集\emph{有界},有界数集的所有项的数值能够被某个区间所全部包含。

有界的另一种表述是,存在正实数$M>0$,使得数集的全部数$x$都满足$|x| \leqslant M$,这与前述说法是等价的。

\begin{definition}
对于一个有上界的实数集,如果某个实数$M$满足: (1)它是这数集的上界. (2)对于无论多么小的正实数$\epsilon$,总存在数集中的数$x$使得$x>M-\epsilon$,则称实数$M$是这数集的\emph{上确界},类似的有\emph{下确界}的定义.
\end{definition}

显然,上确界是最小的上界,下确界是最大的下界。

\begin{theorem}[确界定理]
若实数集合(无论有限集无限集)有上界,则有上确界,下界亦有相应结论。
\end{theorem}

\begin{proof}[证明]
设实数集合$A$有上界$M$,我们先构造出一个数$K$,再证明构造出的这个数正是这集合的上确界。

根据最小数原理,集合$A$中元素的整数部分有最大值,令$K$的整数部分与之相同,这整数部分记作$K_0$。

再将集合$A$中所有元素乘以10后舍去小数部分,这些新数组成的新集合记作$A_1$,这集合有上界$10M$,因此按最小数原理,它也有最大值,而且这最大值除个位以外的部分正是$K_0$(按$K_0$的定义),取这最大值的个位作为$K$的十分位。$K$的其余数位依次类推,$K$在$10^{-n}$上的数位是将集合$A$中全体元素乘以$10^n$后舍去小数部分所得新集合中最大数的个位数。

现在证明,数$K$是集合$A$的上确界,先证明它是上界,反证法,若它不是上界,则$A$中存在比它更大的数$x_0$,那么按实数大小关系定义,在比较$x_0$与$K$时,从左边开始往右比较,第一个不相同的数位上,$x_0$在该数位上的数大于$K$在该数位上的数,但这与$K$在这一数位上的数值的确定方法相矛盾,所以$K$是上界。其次需要证明,$K$是最小的上界,设$L$是一个小于$K$的实数,那么它与$K$相比,从左边开始第一个不相同的数位上,它对应的数较小,假定这数位就是$10^{-n}$,并设$K$和$L$在舍去这一数位以后的全部数位后所得的数分别是$K_n$和$L_n$,那么$K_n>L_n$,但根据$K$的确定过程可知,对于任何正整数$n$,$A$中都存在不小于$K_n$的数,自然这数也就大于$L_n$,因此$K$是最小的上界,即为上确界。
\end{proof}

\subsection{无理指数幂}
\label{sec:irrational-power}

在中学数学里,我们已经有了指数的概念,但那时的指数,受限于有理数的情形,虽然给出了定义在全体实数上的指数函数,却没有说明当指数是无理数时,这个幂是何种意义,本小节就来解决这个问题,我们将通过极限来定义无理指数幂。

先回顾一下有理指数幂的定义,设实数$a>0$且$a \neq 1$,它的正整数$n$次幂定义为
\[ a^n = aa\cdots a(n\text{个}a) \]
在这定义下,显然有$a^n>0$(正值性),而且对于两个正整数$n$和$m$有
\begin{equation}
  \label{eq:exponent-multiple-rule-with-positive-integer}
  a^{n+m}=a^na^m
\end{equation}
这称为指数运算的乘法公式。

如果正整数$n<m$,则在$a>1$时成立
\[ a^n<a^m \]
在$0<a<1$时不等式反向,这就是说,定义在正整数集上的指数函数是单调函数,$a>$时是单调增加的,$0<a<1$时是单调减少的。

现在来把指数推广到任意整数,我们希望上述乘法公式在推广后对任意整数都成立,所以在其中令$m=0$得$a^n=a^n \cdot a^0$,这要对任意正整数$n$都成立则必须$a^0=1$,于是我们就把这作为零次幂的定义,于是指数的正值性仍然成立,而定义在非负整数集上的指数函数仍然保持着它在正整数集上的单调性,接着在上述乘法公式中取$m=-n$,则得到
\[ a^{-n} = \frac{1}{a^n} \]
于是我们把它作为负整数指数幂的定义,这样,我们就把指数概念推广到了任意整数的情形,而且上述乘法公式对于任意两个整数都成立,而且还可以得到下面的公式
\[ a^{n-m} = \frac{a^n}{a^m} \]
容易证明,在把指数函数的定义域从正整数集扩充到全体整数后,正值性和单调性仍然成立,这里就单调性作一证明。
\begin{theorem}
  设两个整数$n<m$,实数$a>1$且$a \neq 1$,则在$a>1$时有$a^n<a^m$,在$0<a<1$时有$a^n>a^m$.
\end{theorem}

\begin{proof}[证明]
  在$a>1$的情况下,如果$n$是负整数而$m$是非负整数,则利用定义在正整数集上的指数函数的单调性得
  \[ a^n = \frac{1}{a^{-n}} < \frac{1}{a^0} = 1 = a^0 \leqslant a^m \]
  而在$n$和$m$都是负整数的情形,$-n$和$-m$是两个正整数并且$-n>-m$,所以利用定义在正整数集上的指数函数的单调性,有
  \[ a^n = \frac{1}{a^{-n}} < \frac{1}{a^{-m}} = a^m \]
  这就证得$a>1$时,定义在整数集上的指数是增函数,而在$0<a<1$时,由
  \[ a^n = \frac{1}{a^{-n}} \]
  知它是减函数。
\end{proof}

由这定理即有如下推论
\begin{inference}
  \label{inference:exponent-compare-to-1}
对于$a>1$和正整数$n$,有$a^n>1$,而对于负整数$n$,则$0<a^n<1$,而对于$0<a<1$,正整数$n$则是$0<a^n<1$,对负整数$n$则是$a^n>1$.
\end{inference}


我们再继续把指数向有理数范围内推广,我们先证下面的结论
\begin{theorem}
 设$n$和$m$是任意两个整数,实数$a>0$且$a \neq 1$,则$(a^m)^n = a^{mn} = (a^n)^m$. 
\end{theorem}

\begin{proof}[证明]
  先证$n$和$m$都是正整数的情形,$(a^m)^n$代表$n$个$a^m$相乘,而$a^m$代表$m$个$a$相乘,所以最终便是$mn$个$a$相乘,所以$(a^m)^n=a^{mn}$,同理$(a^n)^m=a^{nm}=a^{mn}$

  如果$m$和$m$中至少有一个是零,则结论显然是成立的,然后按负正整数指数幂的定义也容易得出结论对于$n$和$m$中至少有一个是负整数时也是成立的。
\end{proof}

有了这个定理,我们来考虑有理整数幂,设$x=n/m$为有理数,其中$n$和$m$是一对互素的整数并且$m$是正整数,我们推广的依据是使得刚才定理中的结论对有理数也成立,这就是说,有下式成立
\[ (a^{\frac{n}{m}})^m = a^{\frac{n}{m}m} = a^n \]
于是得到
\[ a^{\frac{n}{m}} = \sqrt[m]{a^n} \]
我们就把它作为有理指数幂和定义,乘法公式和负指数幂的公式仍然是成立的,并且指数函数在有理数集上的正值性和单调性仍然是成立的,这里我们证明一下单调性。

根据有理指数幂的定义,不难证明\autoref{inference:exponent-compare-to-1}在有理数上也是成立的,假定$x$和$y$是两个有理数并且$x<y$,则由正值性知$a^x>0$,$a^y>0$,在$a>1$时
\[ \frac{a^x}{a^y} = a^{x-y} \]
如果$a>1$,那么由$x-y$是一个负有理数,利用\autoref{inference:exponent-compare-to-1} 在有理数的情形即知$a^{x-y}<1$,所以$a^x<a^y$,同理可证$0<a<1$的情形,这就证明了定义在有理数集上的指数函数的单调性。

现在,我们再作一次推广,把指数推广到全体实数,这只要定义无理指数幂就可以了,但是这时指数的乘法公式似乎已经对我们没什么帮助了,我们转而寻求用有理数逼近无理数时有理指数幂的极限来定义无理指数幂,这就是以下的定义
\begin{definition}
  设实数$a>0$且$a \neq 1$,各项均为有理数的数列$r_n$收敛到一个无理数$r$,则实数$a$的$r$次幂定义为
  \[ a^r = \lim_{n \to \infty} a^{r_n} \]
\end{definition}

这里有几个疑问:这个极限存在吗?对于收敛到无理数$r$的所有有理数列,这个极限都相等吗? 下面这个定理就肯定了这一点。
\begin{theorem}
  设实数$a>0$且$a \neq 1$,$r$是一个无理数,则任意收敛到$r$的有理数数列都收敛,而且极限值都相同。
\end{theorem}

先证明下面的引理
\begin{lemma}
  \label{lemma:a-power-rn-to-1-when-rational-rn-to-0}
  设有理数数列$r_n$收敛到零,实数$a>0$且$a \neq 1$,则有极限$\lim_{n \to \infty} a^{r_n} = 1$.
\end{lemma}

\begin{proof}[证明]
  我们在\autoref{example:limit-of-n-sqrt-a}中已经证得$\lim_{n \to \infty} \sqrt[n]{a} = 1$,因此对于任意小的正实数$\varepsilon$,都存在正整数$N$,使得$n>N$时恒有$|\sqrt[n]{a}-1|<\varepsilon$,在$a>1$时,就是$1<\sqrt[n]{a}<1+\epsilon$,现在就任意取定一个$n_0>N$,从而有$1<\sqrt[n_0]{a}<1+\varepsilon$ 而由引理条件,$r_n$以零为极限,所以对于正实数$1/n_0$,存在正整数$N_1$,使得当$n>N_1$时有$|r_n|<1/n_0$,这时按照定义在有理数集上的指数函数的单调性(下式中做了限定$\varepsilon<1$)
  \[ 1-\varepsilon<\frac{1}{1+\varepsilon}<\frac{1}{\sqrt[n_0]{a}}<a^{r_n}<\sqrt[n_0]{a}<1+\varepsilon \]
  由此即知$|a^{r_n}-1|<\varepsilon$,所以只要$n>\max\{N, N_1\}$时便能保证$|a^{r_n}-1|<\varepsilon$,这即表明引理中的极限成立,而类似的可以证明$0<a<1$的情况。
\end{proof}

现在回过头来证明前面的定理
\begin{proof}[证明]
  我们先通过两个特殊的有理数数列来确定出这个极限值来,再证明所有收敛到无理数$r$的有理数数列都必以它为极限。

  取无理数$r$的$n$位不足近似值$x_n$和$n$位过剩近似值$y_n$,即$x_n$是把无理数$r$的第$n$位小数以后的小数全部舍去而得的有理数,$y_n$是把它第$n$位小数以后的小数收上来而得到的有理数,显然$x_n<r<y_n$并且$y_n-x_n=10^{-n}$,同时,$x_n$单调不减,而$y_n$单调不增,考虑由它们构成的两个有理指数幂的数列$a^{x_n}$和$a^{y_n}$,由定义在有理数集上的指数函数的单调性,在$a>1$的假定下有
  \[ a^{x_n} \leqslant a^{x_{n+1}} \leqslant \cdots \leqslant a^{y_{n+1}} \leqslant a^{y_n} \]
  于是作闭区间序列$U_n = [a^{x_n},a^{y_n}]$,则显见$U_{n+1} \subset U_n$,而区间的长度$a^{y_n}-a^{x_n}=a^{x_n}(a^{y_n-x_n}-1)$,因为$y_n-x_n=10^{-n} \to 1$,由刚才所证的\autoref{lemma:a-power-rn-to-1-when-rational-rn-to-0},$a^{y_n-x_n}-1$是一个无穷小,而前面的因子$a^{x_n}<a^{y_n} \leqslant a^{y_1}$是有界量,所以这闭区间的长度序列趋于零,于是由闭区间套定理,存在唯一实数$K$,使得$a^{x_n}<K<a^{y_n}$对一切正整数$n$成立,显然$\lim_{n \to \infty}a^{x_n} = \lim_{n \to \infty}a^{y_n}=K$。

  接下来我们需要证明,对于其它任何收敛到无理数$r$的有理数数列$z_n$,也必将有$\lim_{n \to \infty}a^{z_n}=K$,这是因为
  \[ \lim_{n \to \infty} \frac{a^{z_n}}{a^{x_n}} = \lim_{n \to \infty} a^{z_n-x_n} = 1 \]
  因此
  \[ \lim_{n \to \infty} a^{z_n} = \lim_{n \to \infty} a^{x_n} \frac{a^{z_n}}{a^{x_n}} = \lim_{n \to \infty} a^{x_n} \cdot \lim_{n \to \infty} \frac{a^{z_n}}{a^{x_n}} = K \cdot 1 = K \]
  这就在$a>1$的情况下证明了定理,而$0<a<1$时的情况完全类似。
\end{proof}

这样无理指数幂定义的存在性和唯一性问题就解决了,我们把指数幂和概念推广到了指数是任意实数的场合,这时有一个问题就冒出来了,那就是我们是用的极限而不是指数乘法公式来推导无理指数幂的,那么推广后的实数指数幂是否仍满足前面的指数乘法公式呢?进一步,在有理数上成立的那些运算性质,是否都仍然成立呢? 这由以下定理回答
\begin{theorem}
  \label{theorem:real-exponent-compute-rule}
  设实数$a>0$且$a \neq 1$,$x$和$y$是任意两个实数,则
  (1).
  \[ a^{x+y} = a^xa^y \]
  (2).
  \[ a^{-x} = \frac{1}{a^x} \]
\end{theorem}

为了证明它,先证一个引理
\begin{lemma}
  \label{lemma:a-power-rn-to-a-pow-r-when-rational-rn-to-rational-r}
  设实数$a>0$且$a \neq 1$,$r_n$是一个有理数数列,并且收敛到一个有理数$r$,则$\lim_{n \to \infty} a^{r_n} = a^{r}$.
\end{lemma}

\begin{proof}[证明]
  由有理数数列$r_n$收敛到有理数$r$即知有理数数列$r_n-r$收敛到零,由\autoref{lemma:a-power-rn-to-1-when-rational-rn-to-0}即知$\lim_{n \to \infty}a^{r_n-r} = 1$,从而
  \[ \lim_{n \to \infty} a^{r_n} = \lim_{n \to \infty} a^{r+(r_n-r)} = \lim_{n \to \infty} a^ra^{r_n-r} = a^r \lim_{n \to \infty}a^{r_n-r} = a^r \]
\end{proof}

现在来证明\autoref{theorem:real-exponent-compute-rule}
\begin{proof}[证明]
  (1).只需证明$x$和$y$中至少有一个无理数的情形,假定$x$是无理数,设$x_n$是一个以$x$为极限的有理数数列,则有
  \[ a^{x_n+y} = a^{x_n} \cdot a^y \]
  显然$x_n+y$是一个以$x+y$为极限的有理数数列,而$x+y$为无理数,所以上式左边的极限是$a^{x+y}$,显然右端的极限是$a^xa^y$,由极限的唯一性即得
  \[ \lim_{n \to \infty}a^{x_n+y} = \lim_{n \to \infty}a^{x_n} \cdot a^y \]
  这就表明
  \[ a^{x+y} = a^xa^y \]
  当$x$和$y$都是无理数时,设$x_n$和$y_n$是两个分别收敛到$x$和$y$的有理数数列,有
  \[ a^{x_n+y_n} = a^{x_n}a^{y_n} \]
  显然右端以$a^xa^y$为极限,对于左边,如果$x+y$是无理数,则$x_n+y_n$是收敛到无理数$x+y$的有理数数列,所以它的极限是$a^{x+y}$,如果$x+y$是有理数,则$x_n+y_n$是收敛到有理数$x+y$的有理数数列,由\autoref{lemma:a-power-rn-to-a-pow-r-when-rational-rn-to-rational-r}知左边极限也是$a^{x+y}$,所以无论$x+y$是有理数还是无理数,左边都以$a^{x+y}$为极限,由极限的唯一性,得$a^{x+y}=a^xa^y$.

  (2). 同样只需要证明$x$为无理数的情形,设有理数数列$x_n$以$x$为极限,则显然有
  \[ a^{-x_n} = \frac{1}{a^{x_n}} \]
  显然$-x_n$是以无理数$-x$为极限的有理数数列,所以上式左边以$a^{-x}$为极限,右边显然以$1/a^x$为极限,由极限的唯一性,结论成立。
\end{proof}

有了定理中的这两条,显然对于实数$x$和$y$,也有
\[ a^{x-y} = \frac{a^x}{a^y} \]
所以有理指数幂的运算性质,在实数范围内仍然是成立的。

既然指数扩展到了全体实数,那么我们也可以将指数函数的定义域扩充到全体实数上了,我们先来证明指数函数在R都是单调的

\begin{theorem}
  设实数$a>1$且$a>0$,指数函数$f(x)=a^x$在$a>1$时是$R$上的严格递增函数,在$0<a<1$时是严格递减函数。
\end{theorem}

\begin{proof}[证明]
  先证$a>1$的情况,这时任取两个实数$x<y$,只需证明$a^x<a^y$,此处只需要证明$x$和$y$中至少有一个无理数的情形,先假定$x$是无理数而$y$是有理数,则可以在$x$和$y$之间取一个有理数$z$,即$x<z<y$(这总是可以办到的),然后设$x_n$是一个收敛到$x$的有理数数列,则由定义在有理数集上的指数函数的单调性和数列极限的保号性,在$x_n$中必定从某项起恒成立下面不等式
  \[ a^{x_n}<a^z<a^y \]
  对上式取极限即得
  \[ a^x \leqslant a^z < a^y \]
  这就证明了$a^x<a^y$,这里引入$z$就是为了去掉$a^x \leqslant a^y$中的等号。

  同理可证$x$为有理数而$y$为无理数的情形,现在来看$x$和$y$都是无理数的情形,这时通过有理数列逼近的方法只能得出$a^x \leqslant a^y$,为了去掉等号,我们在$x$和$y$之间插入两个有理数$r$和$s$,即$x<r<s<y$,这时令$x_n$和$y_n$是两个分别收敛到$x$和$y$的有理数数列,就有
  \[ a^{x_n} < a^r < a^s < a^{y_n} \]
  取极限即得
  \[ a^x \leqslant a^r < a^s \leqslant a^y \]
所以$a^x < a^y$,这就表明$a>1$时,指数函数$0<a<1$是$R$上的递增函数,而对于$0<a<1$的情况,容易知道$a^{-x}=1/a^x$对无理数$x$也是成立的,所以由这关系即可知道$0<a<1$时指数函数是单调递减的。
\end{proof}

接着考虑指数函数在$R$上的连续性,结论是,指数函数是实数集$R$上的连续函数。为着证明这一点,我们需要先证一个极限
\begin{lemma}
  \label{lemma:a-power-x-to-1-when-real-x-to-0}
  设实数$a>0$且$a \neq 1$,则有极限$\lim_{x \to 0} a^x = 1$
\end{lemma}

\begin{proof}[证明]
  只证明$a>1$的情形,$0<a<1$是类似的。

  由引理\autoref{lemma:a-power-rn-to-1-when-rational-rn-to-0},当$x$取有理数并趋于零时,有$a^x$趋于1,所以对于无论多么小的正实数$\varepsilon$,总存在另一正实数$\delta$,使得满足$|x|<\delta$的一切有理数$x$都成立$1-\varepsilon<a^x<1+\varepsilon$,而对于满足$|x|<\delta$的任一无理数$x$,必然可以在区间$(-\delta,\delta)$上找到一个有理数$x'$和,使得$0<|x|<|x'|<\delta$,这时在$a>1$的情况下,利用前面已经证过的指数函数在实数集上的单调性,就有
  \[ 1-\varepsilon < a^{-x'}<a^x<a^{x'} < 1+\varepsilon \]
  这就表明区间$(-\delta,\delta)$上的全体实数$x$无论有理数无理数都满足$|a^x-1|<\varepsilon$,所以此极限得证。
\end{proof}

\begin{theorem}
  定义在$R$上的指数函数$f(x)=a^x$($a>0$且$a \neq 1$),是$R$上的连续函数。
\end{theorem}

\begin{proof}[证明]
  只要证明它在$R$上任何一点$x_0$处都连续即可,因为
  \[ a^{x_0+h} - a^{x_0} = a^{x_0}(a^h-1) \]
  由\autoref{lemma:a-power-x-to-1-when-real-x-to-0}即知$\lim_{h \to 0} a^{x_0+h} = a^{x_0}$,这就表明指数函数在$x_0$处是连续的,由$x_0$的任意性即得知它在整个$R$上都是连续的。
\end{proof}

现在,我们有了完整的指数定义,就可以考虑它的逆运算了,也就是对数,设实数$a>0$且$a \neq 1$,$y$是一个正实数,如果实数$x$满足方程$a^x=y$,则称$x$是$y$的以$a$为底的 \emph{对数},显然,指数和对数互为逆运算。

对数的定义有一个问题,满足方程$a^x=y$的实数$x$是否一定存在呢,实数$y$必须是正实数吗?为解决这个问题,我们还需要证明指数函数的另一个性质
\begin{theorem}
  设实数$a>0$且$a \neq 1$,则指数函数$f(x)=a^x$的函数值可以取遍一切正实数,换句话说,它的值域是$(0,+\infty)$.
\end{theorem}

这利用连续函数在闭区间上的介值性便可以证明,我们在本节的后文给出。


\subsection{复数}
\label{sec:complex-number}



%%% Local Variables:
%%% mode: latex
%%% TeX-master: "../../book"
%%% End:
