
\section{函数极限存在的条件}
\label{sec:the-condition-of-function-limit-exist}

\begin{theorem}[函数极限与数列极限的关系]
  在$x_0$的某空心邻域内有定义的函数$f(x)$,当$x \to x_0$时存在极限的充分必要条件是,对于任意一个在这空心邻域内取值并以$x_0$为极限的数列$x_n$,$\lim_{n \to \infty}f(x_n)$都存在而且都相等。
\end{theorem}

\begin{proof}[证明]
  先证必要性,如果$\lim_{x \to x_0} = A$,那么对于任意小的正实数$\varepsilon > 0$,都存在另一个正实数$\delta > 0$,使得对这邻域内满足$|x-x_0|<\delta$的实数$x$都成立不等式$|f(x)-A|<\varepsilon$,那么对于任意一个也在这空心邻域内取值并以$x_0$为极限的数列$x_n$,因为它以$x_0$为极限,所以对于这个$\delta>0$,就必然能够从某一项$x_N$开始,后面的所有项都满足$|x_n-x_0|<\delta$,于是就有$|f(x_n)-A|<\varepsilon$,这就表明$f(x_n)$当$n \to \infty$时以$A$为极限,必要性得证。

  再证充分性,如果对于任意一个在这空心邻域内取值并收敛到$x_0$的数列$x_n$,对应的函数值数列$f(x_n)$都收敛到同一实数$A$,我们将证明,函数$f(x)$在$x \to x_0$时也必将收敛到$A$. 采用反证法,假使函数$f(x)$当$x \to x_0$时不以数$A$为极限,那么必然存在某个$\varepsilon_0>0$,使得无论把另一个正实数$\delta>0$限制得多么小,总有满足$|x-x_0|<\delta$的实数$x$能够使得$|f(x)-A| \geqslant \varepsilon_0$成立,于是先取$\delta=1$,得出一个符合这条件的实数$x_1$,然而取$\delta=\min\{\frac{1}{2}, |x_1-x_0|\}>0$,又可以选出$x_2$,依次这样下去,逐个令$\delta_n=\min\{\frac{1}{n}, |x_{n-1}-x_0|\}$,就可以挑选出$x_{n+1}$,这样就作出一个数列$x_n$,由$|x_n-x_0|<\delta_n<\frac{1}{n}$可知$x_n$收敛到$x_0$,但是由于$|f(x_n)-A| \geqslant \varepsilon$恒成立,可知数列$f(x_n)$并不收敛到$A$,这样,我们就证明了如果函数$f(x)$当$x \to x_0$时不以$A$为极限,那么就可以构造出一个以$x_0$为极限的数列$x_n$,使得$f(x_n)$也不以$A$为极限,这与我们的条件是矛盾的,所以充分性得证。
\end{proof}

%%% Local Variables:
%%% mode: latex
%%% TeX-master: "../../book"
%%% End:
