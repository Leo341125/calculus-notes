
\section{函数的连续性}
\label{sec:the-continuity-of-function}

\subsection{连续的概念与性质}
\label{sec:the-concept-and-properties-of-continuity-function}

\begin{definition}
  如果函数$f(x)$在$x_0$处的极限正好是该点处的函数值$f(x_0)$则称函数在$x_0$处 \emph{连续},即
  \[ \lim_{x \to x_0} = f(x_0) \]
\end{definition}

如果是该点处的左极限等于该点处函数值,则称函数在该点处 \emph{左连续},类似的有 \emph{右连续}的概念。

连续用极限的精确语言描述就是,对于无论多么小的正实数$\varepsilon$,恒存在另一正实数$\delta$,使得对区间$(x_0-\delta, x_0+\delta)$上的一切实数成立着$|f(x)-f(x_0)|<\varepsilon$成立。

如果函数在某点处不连续,则称该点是函数的一个 \emph{间断点},如果函数在该点处存在极限,只是这极限与函数值不相等或者该点根本就没有定义函数值,那么称这点是 \emph{可去间断点},可以通过改变或者定义该点的函数值为该点的极限值的方式来将函数进行 \emph{连续开拓}。如果函数在某点处分别存在左极限和右极限,但是两个极限不相等,则称该点是函数的 \emph{跳跃间断点},跳跃间断点和可去间断点统称 \emph{第一类间断点},第一类间断点的特征是函数在该点存在两个方向的单侧极限。除第一类间断点之外的其它间断点统称 \emph{第二类间断点},显然,第二类间断点处至少有一个单侧极限不存在。

\begin{definition}
  如果函数在某个区间上处处连续,则称函数在这区间上连续,或者说它是这区间上的连续函数。
\end{definition}

讨论下函数在某点处连续时所具有的性质:
\begin{theorem}[局部有界性]
  若函数在某点处连续,则必在该点的某邻域上有界。
\end{theorem}

\begin{theorem}[局部保号性]
  若函数在$x_0$处连续,则对于任意小于$f(x_0)$的实数$r$,存在$x_0$的某邻域$(x_0-\delta,x_0+\delta)$,使得函数在该邻域内恒有$f(x)>r$,类似的,对于任意大于$f(x_0)$的实数$r$,也存在$x_0$的某邻域$(x_0-\delta,x_0+\delta)$,使得函数在该区间上恒有$f(x)<r$.
\end{theorem}

\begin{inference}
  如果函数在某点处连续,且该点处函数值为正,则存在该点的某邻域内,函数在这邻域内恒为正号,同理,如果该点函数值为负,则函数必在该点的某邻域内恒保持负号。
\end{inference}

\begin{theorem}
  如果函数$f(x)$和$g(x)$都在$x_0$处连续,则它们的和、差、积、商所作成的函数在该点也连续,在商的情形,要求$g(x_0) \neq 0$。
\end{theorem}

\begin{theorem}[复合函数的连续性]
  设函数$g(x)$在$x_0$处连续,记$u_0=g(x_0)$,若另一函数$f(u)$在$u_0$处连续,则复合函数$f(g(x))$在$x_0$处连续。
\end{theorem}

以上性质和定理均可直接由极限的相应性质得出。

\subsection{实数的无理指数幂}
\label{sec:the-power-of-real-with-rational-exponent}

在中学数学里,我们已经有了指数的概念,但那时的指数,受限于有理数的情形,虽然给出了定义在全体实数上的指数函数,却没有说明当指数是无理数时,这个幂是何种意义,本小节就来解决这个问题,我们将通过极限来定义无理指数幂。

\begin{definition}
  设实数$a>0$且$a \neq 1$,各项均为有理数的数列$r_n$收敛到一个无理数$r$,则实数$a$的$r$次幂定义为
  \[ a^r = \lim_{n \to \infty} a^{r_n} \]
\end{definition}

这里有几个疑问:这个极限存在吗?对于收敛到无理数$r$的所有有理数列,这个极限都相等吗? 下面这个定理就肯定了这一点。
\begin{theorem}
  设实数$a>0$且$a \neq 1$,$r$是一个无理数,则任意收敛到$r$的有理数数列都收敛,而且极限值都相同。
\end{theorem}

先证明下面的引理
\begin{lemma}
  设有理数数列$r_n$收敛到零,实数$a>0$且$a \neq 1$,则有极限$\lim_{n \to \infty} a^{r_n} = 1$.
\end{lemma}

\begin{proof}[证明]
  我们在\autoref{example:limit-of-n-sqrt-a}中已经证得$\lim_{n \to \infty} \sqrt[n]{a} = 1$,因此对于任意小的正实数$\varepsilon$,都存在正整数$N$,使得$n>N$时恒有$|\sqrt[n]{a}-1|<\varepsilon$,在$a>1$时,就是$1<\sqrt[n]{a}<1+\epsilon$,现在就任意取定一个$n_0>N$,从而有$1<\sqrt[n_0]{a}<1+\varepsilon$ 而由引理条件,$r_n$以零为极限,所以对于正实数$1/n_0$,存在正整数$N_1$,使得当$n>N_1$时有$|r_n|<1/n_0$,这时按照定义在有理数集上的指数函数的单调性,
  \[ 1<a^{r_n}<\sqrt[n_0]{a}<1+\varepsilon \]
  由此即知$|a^{r_n}-1|<\varepsilon$,所以只要$n>\max{N, N_1}$时便能保证$|a^{r_n}-1|<\varepsilon$,这即表明引理中的极限成立,而类似的可以证明$0<a<1$的情况。
\end{proof}

现在回过头来证明前面的定理
\begin{proof}[证明]
  我们先通过两个特殊的有理数数列来确定出这个极限值来,再证明所有收敛到无理数$r$的有理数数列都必以它为极限。

  取无理数$r$的$n$位不足近似值$x_n$和$n$位过剩近似值$y_n$,即$x_n$是把无理数$r$的第$n$位小数以后的小数全部舍去而得的有理数,$y_n$是把它第$n$位小数以后的小数收上来而得到的有理数,显然$x_n<r<y_n$并且$y_n-x_n=10^{-n}$,同时,$x_n$单调不减,而$y_n$单调不增,考虑由它们构成的两个有理指数幂的数列$a^{x_n}$和$a^{y_n}$,由定义在有理数集上的指数函数的单调性,在$a>1$的假定下有
  \[ a^{x_n} \leqslant a^{x_{n+1}} \leqslant \cdots a^{y_{n+1}} \leqslant a^{y_n} \]
  于是作闭区间序列$U_n = [a^{x_n},a^{y_n}]$,则显见$U_{n+1} \subset U_n$,而区间的长度$a^{y_n}-a^{x_n}=a^{x_n}(a^{y_n-x_n}-1)$,因为$y_n-x_n=10^{-n} \to 1$,由刚才所证的引理,$a^{y_n-x_n}-1$是一个无穷小,而前面的因子$a^{x_n}<a^{y_n} \leqslant a^{y_1}$是有界量,所以这闭区间的长度序列趋于零,于是由闭区间套定理,存在唯一实数$K$,使得$a^{x_n}<K<a^{y_n}$对一切正整数$n$成立,显然$\lim_{n \to \infty}a^{x_n} = \lim_{n \to \infty}a^{y_n}=K$。

  接下来我们需要证明,对于其它任何收敛到无理数$r$的有理数数列$z_n$,也必将有$\lim_{n \to \infty}a^{z_n}=K$,这是因为
  \[ \lim_{n \to \infty} \frac{a^{z_n}}{a^{x_n}} = \lim_{n \to \infty} a^{z_n-x_n} = 1 \]
  因此
  \[ \lim_{n \to \infty} a^{z_n} = \lim_{n \to \infty} a^{x_n} \frac{a^{z_n}}{a^{x_n}} = \lim_{n \to \infty} a^{x_n} \cdot \lim_{n \to \infty} \frac{a^{z_n}}{a^{x_n}} = K \cdot 1 = K \]
  这就在$a>1$的情况下证明了定理,而$0<a<1$时的情况完全类似。
\end{proof}

这样无理指数幂的定义问题就解决了,不过又有另一个问题冒出来了,那就是现在我们可以将指数函数的定义域拓展到全体实数上了,那么指数函数在有理数集上的运算性质和单调性,在实数集上是否还能保持成立呢?回答仍然是肯定的,先看指数运算法则
\begin{theorem}
  设实数$a>0$且$a \neq 1$,$x$和$y$是任意两个实数,则
  (1).
  \[ a^{x+y} = a^xa^y \]
  (2).
  \[ a^{-x} = \frac{1}{a^x} \]
\end{theorem}

由定理即可得直接得出
\[ a^{x-y} = a^x a^{-y} = \frac{a^x}{a^y} \]

\begin{theorem}
  定义在$R$上的指数函数$a^x$当$a>1$时是增函数,当$0<a<1$时是减函数.
\end{theorem}

\subsection{初等函数的连续性}
\label{sec:the-continuity-of-elementary-function}

在中学里,我们接触过几类 \emph{基本初等函数}: 幂函数、指数函数、对数函数、三角函数. 我们在这一小节里来证明这些函数在它们的定义域的各个区间上都是连续函数,在有了这个结论之后,根据连续的性质,所有的初等函数就都是连续函数了。

1. 幂函数
\begin{theorem}
  幂函数$f(x)=x^p$在定义域的各个区间上连续。
\end{theorem}

\begin{proof}[证明]
  因为如果$p<0$,有$f(x)=1/x^{-p}$,如果分母是连续的,则$f(x)$就是连续的,所以只要证明$p>0$的情况就可以了。

  先证明$p$是正整数的情况,这时由
  \[ (x_0+h)^p-x_0^p = \sum_{i=1}^nx_0^{p-i}h^i \]
  显然当$h \to 0$时,右边的各项(有限项)都趋于0,因此$(x_0+h)^p \to x_0^p$,所以函数在$x_0$处连续,由$x_0$的任意性,$p$为正整数的情形得证。
\end{proof}

2. 指数函数
\begin{theorem}
  指数函数$f(x)=a^x(a>0,a\neq 1)$是$R$上的连续函数。
\end{theorem}



%%% Local Variables:
%%% mode: latex
%%% TeX-master: "../../book"
%%% End:
