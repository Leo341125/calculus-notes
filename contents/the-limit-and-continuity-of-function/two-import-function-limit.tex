
\section{两个重要极限}
\label{sec:two-import-function-limit}

这一节的主要任务是建立两个重要极限.

\begin{theorem}
  \label{theorem:sinx-over-x-to-1-when-x-to-0}
  对于正弦函数,有如下极限
  \[ \lim_{x \to 0} \frac{\sin{x}}{x} = 1 \]
\end{theorem}

\begin{proof}[证明]

  如\autoref{fig:limit-of-sinx-over-x-at-0},在单位圆中,点$A$和$C$是圆周上两点,并且$\angle AOC$的弧度值为$x$,$x$是锐角,$OC$与点$A$处的切线相交于点$B$,由图可得
  
\begin{figure}[htbp]
\centering
\includegraphics{contents/the-limit-and-continuity-of-function/pic/limit-of-sinx-over-x-at-0.pdf}
\includegraphics{contents/the-limit-and-continuity-of-function/pic/graphy-of-function-sinx-over-x.pdf}
\caption{}
\label{fig:limit-of-sinx-over-x-at-0}
\end{figure}

  \[ S_{\triangle AOC} < S_{\text{扇形}AOC} < S_{\triangle AOB} \]
  于是便得
  \[ \sin{x} < x < \tan{x} \]
  从而
  \[ \cos{x} < \frac{\sin{x}}{x} < 1 \]
  而当$x \to 0$时, $\cos{x} \to 1$,所以最终得极限
  \[ \lim_{x \to 0} \frac{\sin{x}}{x} = 1 \]
\end{proof}

函数$f(x)=\sin{x}/x$在$x=0$附近的图象如\autoref{fig:limit-of-sinx-over-x-at-0}所示,它的图象反得在函数$y=1/x$和$y=-1/x$之间振荡,在$x=0$附近,函数值趋于1,但它在$x=0$处并无定义。

我们在数列极限的部分曾经证明过下面这个数列存在极限
\[ x_n = \left( 1+\frac{1}{n} \right)^n \]
并把它的极限记作$e$,今来把它推广成函数的极限,这就是以下的定理:
\begin{theorem}
  \[ \lim_{x \to \infty} \left( 1+\frac{1}{x} \right)^x = e \]
  这里$e$是自然对数的底数.
\end{theorem}

\begin{proof}[证明]
  设实数$x$的整数部分为$n$,即$n \leqslant x < n+1$,有
  \[ \left( 1+\frac{1}{n+1} \right)^n < \left( 1+\frac{1}{x} \right)^x < \left( 1+\frac{1}{n} \right)^{n+1} \]
  显见左右两边作为两个数列,都以$e$为极限,我们把这左右两边转化为两个函数,即设$x$的整数部分为$n$,定义
  \[ h_1(x) =  \left( 1+\frac{1}{n+1} \right)^n, \ h_2(x)=\left( 1+\frac{1}{n} \right)^{n+1} \]
  于是就有
  \[ \lim_{x \to +\infty} h_1(x) = \lim_{x \to +\infty} h_2(x) = e \]
  所以由夹逼准则即得
  \[ \lim_{x \to +\infty} \left( 1+\frac{1}{x} \right)^x = e \]
  这是函数在正无穷远处的极限,再来看负无穷处的极限,当$x$以负值趋于负无穷时,令$x=-t$,则$x$趋于负无穷等价于$t$趋于正无穷,而
  \[ \left( 1+\frac{1}{x} \right)^x = \frac{1}{\left( 1-\frac{1}{t} \right)^t} = \left( 1+\frac{1}{t-1} \right)^t = \left( 1+\frac{1}{t-1} \right)^{(t-1) \cdot \frac{t}{t-1}} \]
  利用复合函数的极限结果(\autoref{theorem:limit-of-combine-function}),便知当式右端当$t \to +\infty$时的极限是$e$,所以当$x \to -\infty$时左端的极限也就是$e$,所以最终当$x \to \infty$时,函数$(1+1/x)^x$的极限都是$e$.
\end{proof}

%%% Local Variables:
%%% mode: latex
%%% TeX-master: "../../book"
%%% End:
