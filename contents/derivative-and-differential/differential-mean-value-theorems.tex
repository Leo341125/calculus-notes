
\section{微分中值定理}
\label{sec:differential-mean-value-theorems}

\begin{theorem}[罗尔定理]
  如果函数$f(x)$在开区间$(a,b)$上可导,而且在$x_0 \in (a,b)$处取得极大值或者极小值,则必有$f'(x_0)=0$.
\end{theorem}

\begin{proof}[证明]
  假定在$x_0$处取得极大值,那么必存在$x_0$的某邻域$(x_0-\delta,x_0+\delta)$,使得任意$x \in (x_0-\delta,x_0+\delta)$有$f(x) \leqslant f(x_0)$,于是当$x \in (x_0-\delta, x_0)$时,成立
  \[ \frac{f(x)-f(x_0)}{x-x_0} \geqslant 0 \]
  而在$x \in (x_0,x_0+\delta)$时,则成立
  \[ \frac{f(x)-f(x_0)}{x-x_0} \leqslant 0 \]
  于是分别就有
  \[ f'(x_0-0) = \lim_{x \to x_0^-} \frac{f(x)-f(x_0)}{x-x_0} \geqslant 0 \]
  和
  \[ f'(x_0+0) = \lim_{x \to x_0^+} \frac{f(x)-f(x_0)}{x-x_0} \leqslant 0 \]
  但由$f'(x_0)$是存在的,所以只能$f'(x_0)=0$,极小值的情况也是类似的。
\end{proof}

%%% Local Variables:
%%% mode: latex
%%% TeX-master: "../../book"
%%% End:
