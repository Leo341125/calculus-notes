
\section{微分中值定理}
\label{sec:differential-mean-value-theorems}

\begin{theorem}
  \label{theorem:derivative-at-extreme-value-is-0}
  如果函数$f(x)$在$x_0$处取得极大值或者极小值,且该点处可导,则必有$f'(x_0)=0$.
\end{theorem}

\begin{proof}[证明]
  假定在$x_0$处取得极大值,那么必存在$x_0$的某邻域$(x_0-\delta,x_0+\delta)$,使得任意$x \in (x_0-\delta,x_0+\delta)$有$f(x) \leqslant f(x_0)$,于是当$x \in (x_0-\delta, x_0)$时,成立
  \[ \frac{f(x)-f(x_0)}{x-x_0} \geqslant 0 \]
  而在$x \in (x_0,x_0+\delta)$时,则成立
  \[ \frac{f(x)-f(x_0)}{x-x_0} \leqslant 0 \]
  于是分别就有
  \[ f'(x_0-0) = \lim_{x \to x_0^-} \frac{f(x)-f(x_0)}{x-x_0} \geqslant 0 \]
  和
  \[ f'(x_0+0) = \lim_{x \to x_0^+} \frac{f(x)-f(x_0)}{x-x_0} \leqslant 0 \]
  但由$f'(x_0)$是存在的,所以只能$f'(x_0)=0$,极小值的情况也是类似的。
\end{proof}

\begin{theorem}[罗尔(Rolle)中值定理]
  如果函数$f(x)$在闭区间$[a,b]$内连续,在开区间$(a,b)$内可导,并且$f(a)=f(b)$,则存在$ \xi \in (a,b)$使得$f'(\xi)=0$.
\end{theorem}

\begin{proof}[证明]
  由连续函数性质,闭区间上的连续函数存在最大值和最小值,我们来证明在开区间上必然能够取到这两个最值中的至少一个,然后由\autoref{theorem:derivative-at-extreme-value-is-0}便能得出结论。

  由条件,$f(a)=f(b)=K$,设函数$f(x)$在闭区间上的最大值和最小值分别是$M$和$m$,如果$M=m=K$,那么开区间上任何一点处都是极值,结论是成立的,在$M \neq m$时,两者中至少有一个不等于$K$,假定$M \neq K$,那么这最大值$M$就只能在开区间上某点处取得,而根据\autoref{theorem:derivative-at-extreme-value-is-0},该点处导数为零,如果是$m \neq K$,同理可证。
\end{proof}

\begin{theorem}[拉格朗日(Lagrange)中值定理]
  如果函数$f(x)$在闭区间$[a,b]$上连续,在开区间$(a,b)$内可导,则存在$\xi \in (a,b)$使得
  \[ f'(\xi) = \frac{f(a)-f(b)}{a-b} \]
\end{theorem}

\begin{proof}[证明]
  显然这是罗尔定理的推广,设法让它符合罗尔定理中两个端点的函数值相等的条件,于是让它减去连接首尾两个端点的一次函数,这个一次函数是
  \[ l(x) = f(a) + \frac{f(a)-f(b)}{a-b}(x-a) \]
  作函数$g(x)=f(x)-l(x)$,则显然$g(x)$仍然在闭区间上连续开区间上可导,并且这时$g(a)=g(b)=0$,所以根据罗尔定理,存在$\xi \in (a,b)$,使得$g'(\xi)=0$,于是
  \[ f'(\xi) = \frac{f(a)-f(b)}{a-b} \]
\end{proof}

\begin{theorem}[柯西(Cauchy)中值定理]
  如果函数$f(x)$和函数$g(x)$都在闭区间$[a,b]$上连续且在开区间$(a,b)$内可导,并且$g(b) \neq g(a)$,同时导函数$f'(x)$和$g'(x)$不同时取零值,那么存在$\xi \in (a,b)$,使得
  \[ \frac{f'(\xi)}{g'(\xi)} = \frac{f(a)-f(b)}{g(a)-g(b)} \]
\end{theorem}

\begin{proof}[证明]
  作辅助函数
  \[ h(x) = f(x)-f(a)-\frac{f(a)-f(b)}{g(a)-g(b)}(g(x)-g(a)) \]
  显见$h(x)$满足罗尔中值定理的条件,因此存在$\xi \in (a,b)$,使得$h'(\xi)=0$,于是
  \[ \frac{f'(\xi)}{g'(\xi)} = \frac{f(a)-f(b)}{g(a)-g(b)} \]
\end{proof}

柯西中值定理实际上就是把拉格朗日中值定理应用到了用参数方程$x=f(t),y=g(t)$表示的曲线上。

%%% Local Variables:
%%% mode: latex
%%% TeX-master: "../../book"
%%% End:
