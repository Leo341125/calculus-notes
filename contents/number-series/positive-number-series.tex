
\section{正项级数的收敛性判别}
\label{sec:positive-number-series}

对于级数$\sum_{n=1}^{\infty}a_n$,如果它的每一项都是正数(或者当$n$充分大时恒保持正号),我们有一系列的判别方法可以判断它的收敛性。

由于正项级数的部分和是单调增加的,所以回想起数列极限的单调有界定理,我们就有

\begin{theorem}
  正项级数收敛的充分必要条件是它的部分和有界。
\end{theorem}

\begin{example}
  级数$\sum_{n=1}^{\infty}\frac{1}{n(n+1)}$的部分和$S_n=1-\frac{1}{n+1}<1$,所以级数收敛。
\end{example}

\begin{example}
  级数$\sum_{n=1}^{\infty}\frac{1}{n!}$的部分和$S_n=1+\frac{1}{1!}+\frac{1}{2!}+\cdots+\frac{1}{n!}$,我们在\autoref{sec:a-import-sequence-limit}便已经得到过$S_n<3$,所以级数收敛。
\end{example}

\begin{example}
  \label{example:series-ln-1-plus-1-over-n-converage}
  级数$\sum_{n=1}^{\infty}\ln{\left( 1+\frac{1}{n} \right)}$,注意到$\ln{\left( 1+\frac{1}{n} \right)} = \ln{(n+1)}-\ln{n}$,所以部分和$S_n=\ln{(n+1)}$,显然无界,所以级数发散。
\end{example}

\subsection{比较判别法}
\label{sec:compare-method-aboud-series-converage}

如下的比较判别法是相当重要的:
\begin{theorem}[比较判别法]
  \label{theorem:comparison-method-about-series-converage}
  对于两个正项级数$\sum_{n=1}^{\infty}a_n$和$\sum_{n=1}^{\infty}b_n$,如果从某一项起恒有$a_n \leqslant b_n$,那么由$\sum_{n=1}^{\infty}b_n$的收敛便能推得$\sum_{n=1}^{\infty}a_n$也收敛,同理,由$\sum_{n=1}^{\infty}a_n$发散便能推得$\sum_{n=1}^{\infty}b_n$也发散。
\end{theorem}

\begin{proof}[证明]
  由条件,存在下标$N$,使得当$n>N$时恒有$a_n \leqslant b_n$,分别用$A_n$和$B_n$表示两个级数的部分和,则
  \[ A_n = A_N + (a_{N+1}+\cdots+a_n), \  B_n=B_N+(b_{N+1}+\cdots+b_n) \]
  显然$A_n - A_N \leqslant B_n-B_N$,所以如果$\sum_{n=1}^{\infty}b_n$收敛,则$B_n$有上界,从而$A_n$也有上界,所以$\sum_{n=1}^{\infty}a_n$也收敛。而如果$\sum_{n=1}^{\infty}$是发散的,那么$A_n$必定没有上界,从而$B_n$也不可能有上界,因而$\sum_{n=1}^{\infty}b_n$也必然发散。
\end{proof}

实际上,条件$a_n \leqslant b_n$可以改成$a_n \leqslant \lambda b_n$,其中$\lambda$是一个正常数,这是因为正项级数$\sum_{n=1}^{\infty}b_n$跟正项级数$\sum_{n=1}^{\infty}\lambda b_n$的收敛性是相同的。

这个判别法还有以下的极限形式
\begin{theorem}
  如果正项级数$\sum_{n=1}^{\infty}a_n$和正项级数$\sum_{n=1}^{\infty}b_n$的通项之比有极限(有限的或无穷的均可)
  \[ \lim_{n \to \infty} \frac{a_n}{b_n} = K \]
  则如果$K$是正常数,那么两个级数同时收敛同时发散。如果$K=0$,则由$\sum_{n=1}^{\infty}b_n$收敛可推得$\sum_{n=1}^{\infty}a_n$也收敛,如果$K=+\infty$,由由$\sum_{n=1}^{\infty}b_n$发散可推得$\sum_{n=1}^{\infty}a_n$也发散。
\end{theorem}

\begin{example}
  设$a_n=\frac{1}{n}$,$b_n=\ln{\left( 1+\frac{1}{n} \right)}$,由极限
  \[ \lim_{n \to \infty} \frac{\frac{1}{n}}{\ln{\left( 1+\frac{1}{n} \right)}} =1 \]
  知道它们同时收敛同时发散,而在\autoref{example:series-ln-1-plus-1-over-n-converage}中已经知道$\sum_{n=1}^{\infty}b_n$是发散的,所以$\sum_{n=1}^{\infty}a_n$也是发散的。
\end{example}


\begin{example}
  我们来研究一个重要的级数,即级数$\sum_{n=1}^{\infty}\frac{1}{n^s}$的收敛性。

  如果$s=1$,则级数成为
  \[ \sum_{n=1}^{\infty}\frac{1}{n} = 1+\frac{1}{2} + \frac{1}{3} + \cdots \]
  有如下的片段和估计
  \[ \frac{1}{n+1} + \frac{1}{n+2} + \cdots + \frac{1}{2n} > n \cdots \frac{1}{2n} = \frac{1}{2} \]
  片断和不能任意小,违反柯西收敛准则,所以$s=1$时级数发散,于是$s<1$时级数也都发散,因为部分和会更大。

  而对于$s>1$的情况,我们将证明它是收敛的,这是因为我们有
  \[ \frac{1}{(n+1)^s} + \frac{1}{(n+2)^s} + \cdots + \frac{1}{(2n)^s} < n \cdot \frac{1}{n^s} = \frac{1}{n^{s-1}} \]
  所以我们把正整数分段,每一段以$2^k+1$作为开始,以$2^{k+1}$作为结尾,就有
  \[ \sum_{i=1}^{2^n}\frac{1}{i^s} = 1+\sum_{k=0}^n \sum_{i=1}^{2^k} \frac{1}{(2^k+i)^s} < 1+\sum_{k=0}^n \frac{1}{(2^k)^{s-1}} =  1+\sum_{k=0}^n \left( \frac{1}{2^{s-1}} \right)^k \]
  注意到$s>1$,所以上式最右边是一个公比小于1的等比级数,显然它是收敛的,于是左边的部分和有上界,从而级数收敛。

  这个级数的和作为$s$的函数,便是著名的 \emph{黎曼函数}$\zeta(s)$,即对于$s>1$,
  \[ \zeta (s) = \sum_{n=1}^{\infty} \frac{1}{n^s} \]
  这函数在数论中有非常重要的地位。
\end{example}



我们还有另一种形式的比较判别法
\begin{theorem}
  对于两个正项级数$\sum_{n=1}^{\infty}a_n$和$\sum_{n=1}^{\infty}b_n$,如果从某一项起恒有
  \[ \frac{a_{n+1}}{a_n} \leqslant \frac{b_{n+1}}{b_n} \]
  那么由$\sum_{n=1}^{\infty}b_n$收敛可推出$\sum_{n=1}^{\infty}a_n$也收敛,同样,由$\sum_{n=1}^{\infty}a_n$发散也可以推出$\sum_{n=1}^{\infty}b_n$发散。
\end{theorem}

\begin{proof}[证明]
  设从$n>N$时就有条件中的不等式恒成立,则可得
  \[ \frac{a_n}{a_N} \leqslant \frac{b_n}{b_N} \]
  由\autoref{theorem:comparison-method-about-series-converage}即得结论。
\end{proof}

利用比较判别法,我们把给定级数与一些已知为收敛的级数相比较,可以开发出一系列更具体的判别法,下文的判别法,基本都是如此。

\subsection{柯西判别法与达朗贝尔判别法}
\label{sec:cauchy-dalembert-method-aboud-series-converage}

因为几何级数$\sum_{n=1}^{\infty}q^n$在$0<q<1$时收敛,我们以它为比较标准,就可以得到柯西判别法和达朗贝尔判别法。

\begin{theorem}[柯西判别法]
  对于正项级数$\sum_{n=1}^{\infty}a_n$,作柯西序列
  \[ \mathcal{C}_n = \sqrt[n]{a_n} \]
  如果存在正实数$0<q<1$,使得当$n$充分大时恒有$\mathcal{C}_n \leqslant q$,那么级数$\sum_{n=1}^{\infty}a_n$收敛,如果当$n$充分大时恒有$\mathcal{C}_n \geqslant 1$,那么级数发散。
\end{theorem}

\begin{proof}[证明]
  设当$n>N$时恒有$\mathcal{C}_n \leqslant q$,那么此时有$a_n \leqslant q^n$,而级数$\sum_{n=1}^{\infty}q^n$在$0<q<1$时是收敛的,由比较判别法可得$\sum_{n=1}^{\infty}a_n$也收敛。而如果当$n>N$时恒有$\mathcal{C}_n \geqslant 1$,那么此时恒有$a_n \geqslant 1$,通项不能趋于零,故级数发散。
\end{proof}

柯西判别法也有极限形式
\begin{inference}
  对于正项级数$\sum_{n=1}^{\infty}a_n$,如果柯西序列有极限$\lim_{n \to \infty} \mathcal{C}_n=q$,那么当$0<q<1$时级数收敛,当$q>1$时级数发散,当$q=1$时可能收敛也可能发散。
\end{inference}

\begin{theorem}[达朗贝尔判别法]
  对于正项级数$\sum_{n=1}^{\infty}a_n$,作达朗贝尔序列
  \[ \mathcal{D}_n = \frac{a_{n+1}}{a_n} \]
  如果当$n$充分大时有$\mathcal{D}_n \leqslant q$,其中$0<q<1$为常数,那么级数收敛,如果当$n$充分大时恒有$\mathcal{D}_n \geqslant 1$,那么级数发散。
\end{theorem}

\begin{proof}[证明]
  设当$n>N$时,$\mathcal{D}_n \leqslant q$,其中$0<q<1$,那么自然就有$a_n \leqslant a_N q^{n-N}$,由比较判别法即知原级数收敛。而如果当$n>N$时$\mathcal{D}_n \geqslant 1$,那么自然有$a_n \geqslant a_N$,通项不趋于零,级数发散。
\end{proof}

达朗贝尔判别法的极限形式
\begin{inference}
  对于正项级数$\sum_{n=1}^{\infty}a_n$,如果达朗贝尔序列有极限$\lim_{n \to \infty} \mathcal{D}_n = q$,在$0<q<q$时级数收敛,在$q>1$时级数发散,$q=1$时级数可能收敛也可能发散。
\end{inference}

由达朗贝尔的条件出发,可以得它也满足柯西判别法的条件,因为由$a_n \leqslant a_N q^{n-N}$可得$\sqrt[n]{a_n} \leqslant q \sqrt[n]{a_N/q^N}$,后一根式极限为1,所以当$n$充分大时它可以保证$q \sqrt[n]{a_N/q^N}<q'<1$,这里$q'$是比$q$稍大些但仍然小于1的常数,这样就得出了柯西判别法的条件,所以能用达朗判别法判断为收敛的级数,也能用柯西判别法来判别,但不一定能有达朗贝尔判别法来得方便。

\subsection{拉阿伯判别法}
\label{sec:raabe-method-about-positive-series}

我们已知级数$\sum_{n=1}^{\infty}\frac{1}{n^s}$在$s>1$时收敛,在$s \leqslant 1$时发散,把给定的级数与这个级数相比较,就得出拉阿伯判别法。

\begin{theorem}[拉阿伯判别法]
  对于正项级数$\sum_{n=1}^{\infty}a_n$,记拉阿伯序列
  \[ \mathcal{R}_n = n \left( \frac{a_n}{a_{n+1}}-1 \right) \]
  如果存在常数$r>1$,使得当$n$充分大时恒有$\mathcal{R}_n \geqslant r$,则级数收敛,如果当$n$充分大时恒有$\mathcal{R}_n \leqslant 1$,则级数发散。
\end{theorem}

\begin{proof}[证明]
  假定当$n>N$时恒有$\mathcal{R}_n \geqslant r$,则可得
  \[ \frac{a_{n+1}}{a_n} \leqslant \frac{n}{n+r} \]
  对于右边的分式,我们将说明存在$s>1$,使得当$n$充分大时恒有
  \[ \frac{n}{n+r} \leqslant \left( \frac{n}{n+1} \right)^s \]
  这$s$只要满足下式
  \[ s \leqslant \frac{\ln{\left( 1+\frac{r}{n} \right)}}{\ln{\left( 1+\frac{1}{n} \right)}} \]
  借由当$x \to 0$时的等价无穷小$\ln(1+x) \sim x$,我们可得出上式右边以$r$为极限,而$r>1$,所以取$1<s<r$,则上式右边充分大时便能恒大于$s$,于是
  \[ \frac{a_{n+1}}{a_n} \leqslant \left( \frac{n}{n+1} \right)^s \]
  便能在$n$充分大时恒成立,由级数$\sum_{n=1}^{\infty}\frac{1}{n^s}$在$s>1$时收敛知原级数收敛。

  如果当$n$充分大时$\mathcal{R}_n \leqslant 1$,则此时
  \[ \frac{a_{n+1}}{a_n} \geqslant \frac{n}{n+1} \]
  由级数$\sum_{n=1}^{\infty}\frac{1}{n}$的发散知原级数发散。
\end{proof}

拉阿伯判别法的极限形式是
\begin{inference}
  对于正项级数$\sum_{n=1}^{\infty}a_n$,如果拉阿伯序列存在极限$\lim_{n \to \infty} \mathcal{R}_n=R$,则$R>1$时级数收敛,$R<1$时级数发散,$R=1$时级数可能收敛也可能发散。
\end{inference}

\subsection{库默尔判别法}
\label{sec:kummer-method-about-series-converage}

库默尔判别法是一个泛型化的判别法,利用它可以构造一系列具体的判别法,包括达朗贝尔判别法和拉阿伯判别法。

\begin{theorem}[库默尔(Kummer)判别法]
  对于正项级数$\sum_{n=1}^{\infty}a_n$,我们选取一个序列$c_n$,使得级数$\sum_{n=1}^{\infty}\frac{1}{c_n}$是发散的,作库默尔序列
  \[ \mathcal{K}_n = c_n \cdot \frac{a_n}{a_{n+1}} - c_{n+1} \]
  如果存在正常数$r$,使得当$n$充分大时恒有$\mathcal{K}_n \geqslant r$,那么级数$\sum_{n=1}^{\infty}a_n$收敛,如果当$n$充分大时恒有$\mathcal{K}_n \leqslant 0$,则级数$\sum_{n=1}^{\infty}$发散。
\end{theorem}

在库默尔判别法中,取$c_n=1$,就得出达朗贝尔判别法,取$c_n=n$,就得出拉阿伯判别法。

库默尔判别法的极限形式是
\begin{theorem}
  对于正项级数$\sum_{n=1}^{\infty}a_n$及选定的序列$c_n$,如果库默尔序列有极限$\lim_{n \to \infty} \mathcal{K}_n=K$,则如果$K>0$,则级数收敛,如果$K<0$,则级数发散,而对于$K=0$,则级数可能收敛也可能发散。
\end{theorem}

\subsection{积分判别法}
\label{sec:the-integrate-method-of-determination-number-series}

可以得反常积分来判别一类级数的敛散性.
\begin{theorem}[积分判别式法]
  若可积函数$f(x)$在$[0,+\infty)$上恒取非负值且单调递减,则反常积分$\int_0^{+\infty}f(x)dx$与级数$\sum_{n=1}^{\infty}f(n)$有相同的敛散性。
\end{theorem}

\begin{proof}[证明]
  令$S_n=\sum_{i=1}^nf(i)$,$I_n=\int_0^nf(x)dx$,则$S_n$和$I_n$都是单调增加的数列,又
  \[ I_n=\sum_{i=1}^n\int_{i-1}^if(x)dx \]
  由单调性,有
  \[ \sum_{i=1}^nf(i) \leqslant I_n \leqslant \sum_{i=1}^nf(i-1) \]
  即
  \[ S_n \leqslant I_n \leqslant f(0)+S_{n-1} \]
  可见,如果级数$\sum_{n=1}^{\infty}f(n)$收敛,则$S_n$有极限$S$,则$I_n$单调增加并有上界$f(0)+S$,因而$I_n$有极限(不一定是$S$),即反常积分$\int_{i=1}^{+\infty}f(x)dx$收敛。反之,如果反常积分收敛,则$I_n$有极限$I$,这时$S_n$单调增加并有上界$I$,因此$S_n$收敛,即级数$\sum_{i=1}^{\infty}f(n)$收敛。
\end{proof}

需要说明的是,上述定理中的区间可以是任意的左闭右开区间$[a,+\infty)$,这时级数只要从大于$a$的任一正整数开始即可,从定理证明过程可以看出这并没有什么影响。

\begin{example}
  对于正实数$p$,考虑如下的级数
  \[ \sum_{n=1}^{\infty}\frac{1}{n^p} \]
  对应的函数$f(x)=1/x^p$在$[1/2,+\infty)$上单调递减,由于$p>1$时反常积分$\int_1^{+\infty}f(x)dx$收敛,所以此时级数也收敛,在$p \leqslant 1$时反常积分是发散的,所以此时级数也是发散的。
\end{example}



%%% Local Variables:
%%% mode: latex
%%% TeX-master: "../../book"
%%% End:
