
\section{正项级数的收敛性判别}
\label{sec:positive-number-series}

对于级数$\sum_{n=1}^{\infty}a_n$,如果它的每一项都是正数(或者当$n$充分大时恒保持正号),我们有一系列的判别方法可以判断它的收敛性。

由于正项级数的部分和是单调增加的,所以回想起数列极限的单调有界定理,我们就有

\begin{theorem}
  正项级数收敛的充分必要条件是它的部分和有界。
\end{theorem}

\begin{example}
  级数$\sum_{n=1}^{\infty}\frac{1}{n(n+1)}$的部分和$S_n=1-\frac{1}{n+1}<1$,所以级数收敛。
\end{example}

\begin{example}
  级数$\sum_{n=1}^{\infty}\frac{1}{n!}$的部分和$S_n=1+\frac{1}{1!}+\frac{1}{2!}+\cdots+\frac{1}{n!}$,我们在\autoref{sec:a-import-sequence-limit}便已经得到过$S_n<3$,所以级数收敛。
\end{example}

\begin{example}
  \label{example:series-ln-1-plus-1-over-n-converage}
  级数$\sum_{n=1}^{\infty}\ln{\left( 1+\frac{1}{n} \right)}$,注意到$\ln{\left( 1+\frac{1}{n} \right)} = \ln{(n+1)}-\ln{n}$,所以部分和$S_n=\ln{(n+1)}$,显然无界,所以级数发散。
\end{example}

如下的比较判别法是相当重要的:
\begin{theorem}[比较判别法]
  对于两个正项级数$\sum_{n=1}^{\infty}a_n$和$\sum_{n=1}^{\infty}b_n$,如果从某一项起恒有$a_n \leqslant b_n$,那么由$\sum_{n=1}^{\infty}b_n$的收敛便能推得$\sum_{n=1}^{\infty}a_n$也收敛,同理,由$\sum_{n=1}^{\infty}a_n$发散便能推得$\sum_{n=1}^{\infty}b_n$也发散。
\end{theorem}

\begin{proof}[证明]
  由条件,存在下标$N$,使得当$n>N$时恒有$a_n \leqslant b_n$,分别用$A_n$和$B_n$表示两个级数的部分和,则
  \[ A_n = A_N + (a_{N+1}+\cdots+a_n), \  B_n=B_N+(b_{N+1}+\cdots+b_n) \]
  显然$A_n - A_N \leqslant B_n-B_N$,所以如果$\sum_{n=1}^{\infty}b_n$收敛,则$B_n$有上界,从而$A_n$也有上界,所以$\sum_{n=1}^{\infty}a_n$也收敛。而如果$\sum_{n=1}^{\infty}$是发散的,那么$A_n$必定没有上界,从而$B_n$也不可能有上界,因而$\sum_{n=1}^{\infty}b_n$也必然发散。
\end{proof}

这个判别法还有以下的极限形式
\begin{theorem}
  如果正项级数$\sum_{n=1}^{\infty}a_n$和正项级数$\sum_{n=1}^{\infty}b_n$的通项之比有极限(有限的或无穷的均可)
  \[ \lim_{n \to \infty} \frac{a_n}{b_n} = K \]
  则如果$K$是正常数,那么两个级数同时收敛同时发散。如果$K=0$,则由$\sum_{n=1}^{\infty}b_n$收敛可推得$\sum_{n=1}^{\infty}a_n$也收敛,如果$K=+\infty$,由由$\sum_{n=1}^{\infty}b_n$发散可推得$\sum_{n=1}^{\infty}a_n$也发散。
\end{theorem}

\begin{example}
  设$a_n=\frac{1}{n}$,$b_n=\ln{\left( 1+\frac{1}{n} \right)}$,由极限
  \[ \lim_{n \to \infty} \frac{\frac{1}{n}}{\ln{\left( 1+\frac{1}{n} \right)}} =1 \]
  知道它们同时收敛同时发散,而在\autoref{example:series-ln-1-plus-1-over-n-converage}中已经知道$\sum_{n=1}^{\infty}b_n$是发散的,所以$\sum_{n=1}^{\infty}a_n$也是发散的。
\end{example}

%%% Local Variables:
%%% mode: latex
%%% TeX-master: "../../book"
%%% End:
