
\section{正项级数的收敛性判别}
\label{sec:the-determination-of-positive-number-series}

\subsection{积分判别法}
\label{sec:the-integrate-method-of-determination-number-series}

可以得反常积分来判别一类级数的敛散性.
\begin{theorem}[积分判别式法]
  若可积函数$f(x)$在$[0,+\infty)$上恒取非负值且单调递减,则反常积分$\int_0^{+\infty}f(x)dx$与级数$\sum_{n=1}^{\infty}f(n)$有相同的敛散性。
\end{theorem}

\begin{proof}[证明]
  令$S_n=\sum_{i=1}^nf(i)$,$I_n=\int_0^nf(x)dx$,则$S_n$和$I_n$都是单调增加的数列,又
  \[ I_n=\sum_{i=1}^n\int_{i-1}^if(x)dx \]
  由单调性,有
  \[ \sum_{i=1}^nf(i) \leqslant I_n \leqslant \sum_{i=1}^nf(i-1) \]
  即
  \[ S_n \leqslant I_n \leqslant f(0)+S_{n-1} \]
  可见,如果级数$\sum_{n=1}^{\infty}f(n)$收敛,则$S_n$有极限$S$,则$I_n$单调增加并有上界$f(0)+S$,因而$I_n$有极限(不一定是$S$),即反常积分$\int_{i=1}^{+\infty}f(x)dx$收敛。反之,如果反常积分收敛,则$I_n$有极限$I$,这时$S_n$单调增加并有上界$I$,因此$S_n$收敛,即级数$\sum_{i=1}^{\infty}f(n)$收敛。
\end{proof}

需要说明的是,上述定理中的区间可以是任意的左闭右开区间$[a,+\infty)$,这时级数只要从大于$a$的任一正整数开始即可,从定理证明过程可以看出这并没有什么影响。

\begin{example}
  对于正实数$p$,考虑如下的级数
  \[ \sum_{n=1}^{\infty}\frac{1}{n^p} \]
  对应的函数$f(x)=1/x^p$在$[1/2,+\infty)$上单调递减,由于$p>1$时反常积分$\int_1^{+\infty}f(x)dx$收敛,所以此时级数也收敛,在$p \leqslant 1$时反常积分是发散的,所以此时级数也是发散的。
\end{example}

%%% Local Variables:
%%% mode: latex
%%% TeX-master: "../../book"
%%% End:
