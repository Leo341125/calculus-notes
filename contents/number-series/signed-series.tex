
\section{一般项级数}
\label{sec:signed-series}

\subsection{交错级数}
\label{sec:alternating-sign-series}

\begin{definition}
  如果序列$\{a_n\}$的任意相邻两项的符号都相反,即整个序列交错的取正值和负值,则称其为\emph{交错序列},而对应级数$\sum_{i=1}^{\infty}a_n$为\emph{交错级数}.
\end{definition}

对于交错级数的收敛性有如下结论

\begin{theorem}
  如果数列$\{a_n\}(n=0,1,\ldots)$单调递减并趋于零,则级数$\sum_{i=0}^{\infty}(-1)^{n}a_n$收敛。
\end{theorem}

\begin{proof}[证明]
级数$\sum_{i=0}^{\infty}(-1)^{n-1}a_n$与级数$\sum_{i=0}^n(-1)^na_n$的收敛性是相同的,这里只是为了方便而让首项$a_0$的符号是正的。  

作级数的部分和$S_n=a_0-a_1+a_2-\cdots+(-1)^na_n$,显然
\[ S_{2n+1}=(a_0-a_1)+(a_2-a_3)+\cdots+(a_{2n}-a_{2n+1}) \]
由$\{a_n\}$单调递减可知$S_{2n+1}$是单调增加的正项数列,但是
\[ S_{2n+1}=a_0-(a_1-a_2)-(a_3-a_4)-\cdots-(a_{2n-1}-a_{2n})-a_{2n+1} \]
显然就有$S_{2n+1}<a_0$,即$S_{2n+1}$又有上界,所以$S_{2n+1}$收敛,同样的方法还可得出$S_{2n}$是单调递减有下界,因而也收敛,而由$S_{2n+1}=S_{2n}+a_{2n+1}$及$a_n \to 0(n \to \infty)$可知这两个子列只能收敛到相同的极限,即级数$\sum_{i=0}^{\infty}(-1)^na_n$收敛。
\end{proof}

%%% Local Variables:
%%% mode: latex
%%% TeX-master: "../../book"
%%% End:
