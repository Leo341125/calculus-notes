
\section{实数的完备性}
\label{sec:completeness-of-real-number}

\subsection{实数完备性的几个定理}

\begin{theorem}[确界定理]
  如果数集有上界,则必有上确界,如果有下界,则必有下确界。
\end{theorem}

\begin{theorem}[单调有界定理]
  单调递增有上界的数列收敛,单调递减有下界的数列也收敛。
\end{theorem}

\begin{theorem}[柯西收敛准则]
  数列$x_n$收敛的充分必要条件是,对于任意正实数$\epsilon$,都存在正整数$N$,使得对于任意满足$n_1>N,n_2>N$的$n_1,n_2$都成立$|x_{n_1}-x_{n_2}| < \epsilon$。
\end{theorem}

\begin{theorem}[闭区间套定理]
  如果闭区间无穷序列$[a_i,b_i](i=1,2,\ldots)$满足两个条件: (1)$[a_k,b_k]\supset[a_{k+1},b_{k+1}](i=1,2,\ldots)$,(2)$\lim_{n\to\infty}(b_n-a_n)=0$,则存在唯一实数,同时位于所有闭区间内。
\end{theorem}

\begin{theorem}[有限覆盖定理]
  如果有无穷多个开区间的并集覆盖了一个闭区间,那么能够从中选取有限个开区间,它们的并集就足够覆盖这个闭区间了。
\end{theorem}

\begin{definition}
  对于一个数集和一个实数,如果在这实数的任意空心邻域内都有这数集中的点,则这实数称为这数集的\emph{聚点}。
\end{definition}

\begin{theorem}
  有界的无穷数集必有聚点。
\end{theorem}

\subsection{确界定理推证其它定理}

\begin{theorem}
  确界定理与单调有界定理等价。
\end{theorem}

\begin{proof}[证明]
  先证必要性,假定数列$x_n$单调递增有上界,由确界定理,必有上确界,记为$M$,将证明这上确界就是该数列的极限,这是因为由上确界定义可知,对于任意正实数$\epsilon>0$,存在某个$x_N$,使得$x_N>M-\epsilon$,由单调递增知对于一切$n>N$就都有$x_n>M-\epsilon$,显然又有$x_n<M<M+\epsilon$,所以$M$就是这数列的极限,单调递减有下界的情况也是一样的,下确界就是它的极限。必要性得证。

  再证充分性,假定非空数集$E$有上界$M$,如果数集中有数能够等于$M$,则显然$M$就是上确界,所以只考虑$M\notin E$的情况,作序列$M_n$,取$M_0=M$,以后的选择是这样的,在确定了数$M_i$之后,$M_{i+1}$的确定方式是,考虑无穷数集$T_m = M_i-(M_i-x)/m(m=2,3,\ldots)$,这无穷数集中或者有数集$E$的上界,或者没有,如果有,就取下标最小的那个$T_m$作为$M_{i+1}$,如果没有,就取$M_{i+1}=M_i$,按此方式定义出一个单调递减(单调不增)序列$M_i(i=0,1,\ldots)$,每一个$M_i$都是数集$E$的上界,显然$M_i>x$,所以序列$M_i$是单调递减有下界的,按单调有界定理,它就有极限$M_{\infty}$,剩下的就是证明,这极限$M_{\infty}$就是数集$E$的上确界。首先它必是数集$E$的一个上界,若不然,就必存在某个$r \in E$满足$r>M_{\infty}$,然而序列$M_i$既然以$M_{\infty}$为极限,则必定从某一项起,后续的$M_i$都满足$M_{\infty} \leqslant M_i<M_{\infty}+(r-M_{\infty})=r$,这与$M_i$都是数集$E$的上界矛盾,所以$M_f$必定是数集$E$的一个上界。继续用反证法证明它是上确界,若不然,则存在$\epsilon > 0$,使得$M_{\infty}$的单侧空心邻域$(M_{\infty}-\epsilon, M_{\infty})$上不存在数集$E$中的数,也就是$M_{\infty}-\epsilon$是数集$E$的一个上界,下面来推矛盾,取一个远远小于1的正因子$\lambda$,不等式$M_i<M_{\infty}-\lambda \epsilon$能在$i$充分大时恒成立,只要$\lambda$选的适当小,无穷序列$M_i-(M_i-x)/m(m=2,3,\ldots)$中就必定有某个数能够落在区间$(M_{\infty}-\epsilon, M_{\infty})$上,这样一来,$M_{i+1}$的选择就应该小于$M$,但这显然矛盾。
\end{proof}

\begin{theorem}
  确界定理与柯西收敛准则的充分性条件等价。
\end{theorem}

\begin{proof}[证明]
  先由确界定理推证柯西收敛准则的充分性,假如数列$x_n$满足柯西收敛条件,任取定一个正实数$\epsilon>0$并作序列$\epsilon_n=\epsilon / (2^n)(n=0,1,\ldots)$,则对于每一个$\epsilon_i$,都存在相应的正整数$N_i$,使得数列下标大于$N_i$的任意两项相差不超过$\epsilon_i$,于是这下标大于$N_i$的所有项都落在一个长度不超过$\epsilon_i$的开区间上,就假设是$(r_i, r_i+\epsilon_i)$,所有的$r_i$就组成一个数集,这数集显然也是有界的,因为每一个$r_i$都不超过$r_0+\epsilon$,由确界定理,这数集有上确界,设其为$M$,下面来证明,这$M$就是数列的极限。对任何正实数$\delta$,由上确界定义,必有某个$M-\delta<r_i\leqslant M$,而数列从某一项起的后续所有项能够落在区间$(r_i,r_i+\epsilon_i)$中,所以只要$\epsilon=\epsilon_0$选的合适(比如说就选$M-\delta$),便可以使数列从某一项起的后续所有项全部落在区间$(M-\delta, M)$上,由$\delta$的任意性,知$M$为这数列的极限。
\end{proof}

\subsection{单调有界定理推证其它定理}

\subsection{柯西收敛准则推证其它定理}

\subsection{闭区间套定理推证其它定理}

\subsection{有限覆盖定理推证其它定理}

\subsection{聚点定理推证其它定理}





%%% Local Variables:
%%% mode: latex
%%% TeX-master: "../../book"
%%% End:
