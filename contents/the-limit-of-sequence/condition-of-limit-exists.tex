
\section{数列极限存在的条件}
\label{sec:condition-of-limit-exists}

\subsection{有界数列与确界定理}
\label{sec:bound-number-sequence}

对于一个实数集,如果存在实数$M$,使得集合中的全部数$x$都满足$x \leqslant M$,则称实数$M$是这数集的一个\emph{上界},如果不等式是反向的,则称这实数是这数集的一个\emph{下界},显然,如果$M$是某个数集的上界,则比$M$大的所有实数也都是这数集的上界,对下界亦有类似结论。

如果数集既有上界又有下界,则称数集\emph{有界},有界数集的所有项的数值能够被某个区间所全部包含。

有界的另一种表述是,存在正实数$M>0$,使得数集的全部数$x$都满足$|x| \leqslant M$,这与前述说法是等价的。

\begin{theorem}[收敛数列的有界性]
  收敛数列必有界。
\end{theorem}

\begin{proof}[证明]
  这其实从定义就可以得出了,随便取一个$\epsilon>0$,即知数列从某项起全部落在区间$(a-\epsilon, a+\epsilon)$内,这里$a$是数列极限,再扩大此区间把前面的那些项(有限个)包含进来,于是数列便有界。
\end{proof}

\begin{definition}
对于一个有上界的实数集,如果某个实数$M$满足: (1)它是这数集的上界. (2)对于无论多么小的正实数$\epsilon$,总存在数集中的数$x$使得$x>M-\epsilon$,则称实数$M$是这数集的\emph{上确界},类似的有\emph{下确界}的定义.
\end{definition}

显然,上确界是最小的上界,下确界是最大的下界。

确界定理的证明依赖于实数的完备性,这放在后续章节来完成。

\begin{theorem}
  如果数集有上界,则必有上确界,如果有下界,则必有下确界。
\end{theorem}

\subsection{单调有界定理}
\label{sec:monotone-bound-theorem}

\begin{theorem}
  单调递增有上界的数列必定收敛,而且收敛到它的上确界。单调递减有下界的数列也类似。
\end{theorem}

\begin{proof}[证明]
  只证明单调递增有上界的情况,假如数列$x_n$就是这样的数列,它的上确界是$M$,则数列中的全部项都满足$x_n \leqslant M$,另外,对于任意小的正实数$\epsilon$,由上确界定义,总存在某个$x_N$满足$x_N>M-\epsilon$,再由单调性即知对于$n>N$恒有$M-\epsilon < x_n \leqslant M < M+\epsilon$,所以$M$就是这数列的极限。
\end{proof}

\subsection{柯西收敛准则}
\label{sec:cauchy-converage-principle}

\begin{theorem}[柯西收敛准则]
  数列$x_n$收敛的充分必要条件是,对于任意正实数$\epsilon$,总存在正整数$N>0$,使得任意$n_1>N$和任意$n_2>N$及任意恒有$|x_{n_1}-x_{n_2}| < \epsilon$。
\end{theorem}

\begin{proof}[证明]
  只证明必要性,充分性的证明放在实数完备性那一节。

  如果数列$x_n$收敛到$x$,那么对于任意正实数$\epsilon$,都有正整数$N$,使得$n>N$时恒有$|x_n-x|<\epsilon / 2$,于是对于任意$n_1>N$及$n_2>N$,便有$|x_{n_1}-x_{n_2}|=|(x_{n_1}-x)- (x_{n_2}-x)|\leqslant |x_{n_1}-x|+|x_{n_2}-x|<\epsilon / 2+\epsilon / 2 = \epsilon$。必要性得证。
\end{proof}


%%% Local Variables:
%%% mode: latex
%%% TeX-master: "../../book"
%%% End:
