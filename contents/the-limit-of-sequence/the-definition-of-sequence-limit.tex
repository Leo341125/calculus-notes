
\section{数列极限的概念}
\label{sec:the-definition-of-sequence-limit}

\begin{definition}
  对于实数数列${a_n}$,如果存在实数$a$,使得对于任意小的正实数$\epsilon$,都存在某一下标$N$,使得该数列在这之后的所有项(即$n>N$)都满足
  \begin{equation}
    \label{eq:the-definition-of-sequence-limit}
    |a_n-a|<\epsilon
  \end{equation}
  则称该数列存在极限,实数$a$称为该数列的极限。也称该数列为收敛数列,并且收敛到实数$a$。
\end{definition}

极限为零的数列称为无穷小数列。

%%% Local Variables:
%%% mode: latex
%%% TeX-master: "../../book"
%%% End:
