
\section{收敛数列的性质}
\label{sec:the-properties-of-convergent-sequence.tex}

\subsection{唯一性与保号性}
\label{sec:limit-uniqueness-and-sign-keep-of-limit}

\begin{theorem}[极限唯一性]
  如果数列$x_n$收敛,则极限唯一。
\end{theorem}

\begin{proof}[证明]
  反证法,假若有两个数$a1$和$a2$都是数列的极限,假定$a_1<a_2$,则对于任意正数$\epsilon > 0$,数列都能从某项起同时成立着 $|x_n-a_1| < \epsilon$ 和 $|x_n-a_2| < \epsilon$,于是取$\epsilon < (a_2-a_1)/2$,则前述两个不等式因为无交集而产生矛盾。
\end{proof}

\begin{theorem}
  如果数列$x_n$收敛到实数$a$,则对于任意一个小于$a$的实数$x$,数列都能从某项起恒大于$x$,同样对于任意一个大于$a$的实数$y$,数列也能从某项起恒大于$y$。
\end{theorem}

\begin{proof}[证明]
  只要在极限的定义中取 $\epsilon < a-x$即可得前半部分结论,同样再取$\epsilon < y-a$即得后半部分结论。
\end{proof}

\begin{inference}[极限的保号性]
  如果数列收敛到一个正的实数,则数列必从某项起恒保持正号,同样,若收敛到一个负的实数,则必从某项起恒保持负号。
\end{inference}

\subsection{极限的四则运算}
\label{sec:operate-limit}

\begin{theorem}
  如果数列$x_n$和$y_n$分别收敛到$x$和$y$,则数列$x_n+y_n$、$x_n-y_n$、$x_ny_n$、$x_n/y_n$都收敛,而且极限分别是$x+y$、$x-y$、$xy$、$x/y$,在商的情况要求$y \neq 0$。
\end{theorem}

这定理可以推广到任意有限个数列的情形。

\begin{proof}[证明]
  和差的情况是容易证明的,只证明积和商的情况。

  先证明乘积的情形,由
  \begin{equation*}
    |x_ny_n-xy| = |(x_ny_n-xy_n) + (xy_n-xy)| \leqslant |y_n||x_n-x| + |x| |y_n-y|
  \end{equation*}
  任取$\epsilon > 0$,则存在$N>0$,使得$n>N$时同时恒有$|x_n-x|<\epsilon$和$|y_n-y|<\epsilon$\footnote{本来对同一个$\epsilon$,两个数列的$N$是不同的,但是可以取比这两个$N$都大的$N$,这时就同时有那两个不等式。},另外再由收敛数列的有界性,存在$M>0$,使得$ |y_n| < M$,于是就有 $|x_ny_n-xy| < (M+|x|)\epsilon$,所以$x_ny_n$收敛到$xy.$

  再来证明商的情况,先证明一个结论,如果数列$y_n$收敛到一个非零实数$y$,那么数列$1/y_n$必收敛,且收敛到$1/y$,这是因为
  \begin{equation*}
    \left| \frac{1}{y_n} - \frac{1}{y} \right| = \left| \frac{y_n-y}{yy_n} \right|
  \end{equation*}
  对于任意正实数$\epsilon>0$,上式的分子能从某一个下标$N$开始恒小于$\epsilon$,同时再取另外一个正实数$|y|/2$,数列能从某项起恒有$|y_n|>|y|/2$,于是从某个下标开始,上式就能恒小于$2\epsilon / y^2$,所以数列$1/y_n$收敛到$1/y$,再将$x_n/y_n$视为$x_n$乘以$1/y_n$并利用乘积的结果,便得商的情形。
\end{proof}

\begin{example}
  \label{example:limit-of-n-sqrt-a}
  设实数$a>0$且$a \neq 1$,证明极限$\lim_{n \to \infty} \sqrt[n]{a} = 1$.

  我们在 \autoref{example:limit-of-n-sqrt-a-when-a-greater-than-1}中已经证明了$a>1$时的情形,现在假设$0<a<1$,则有
  \[ \lim_{n \to \infty} \sqrt[n]{a} = \lim_{n \to \infty} \frac{1}{\sqrt[n]{\frac{1}{a}}} = 1 \]
\end{example}

\begin{theorem}[Stolz 定理]
  对于两个数列$x_n$和$y_n$,其中$y_n$是一个单调增或者单调减的数列,如果$\lim_{n\to\infty}\frac{x_{n+1}-x_n}{y_{n+1}-y_n} = M$,那么有$\lim \frac{x_n}{y_n} = M$。
\end{theorem}

\begin{inference}
  如果数列$x_n$收敛到$A$,则$(x_1+x_2+\cdots+x_n)/n$也收敛到$A$,反之亦然。
\end{inference}

%%% Local Variables:
%%% mode: latex
%%% TeX-master: "../../book"
%%% End:
