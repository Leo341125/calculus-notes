
\chapter*{序}
\addcontentsline{toc}{chapter}{序}

微积分是近代建立起来的数学理论,历经数代数学家不断添砖加瓦,构成了今天我们所看到的一部宏伟而严丝合缝的理论,它是高等数学的开端,使人类的数学思维和手段经历了一次质的飞跃。

微积分理论体系严谨,整个理论体系大致分为极限论、微分学、积分学和级数理论四部分,其中极限论是其余各部分的基础,也是初等数学和高等数学(局限于分析学)的分界点,包含数列极限和函数极限两部分内容,微分学包含导数和微分、微分中值定理,还有延伸学科微分方程(又细分为常微分方程和偏微分方程),积分学包含不定积分、定积分、无穷积分与瑕积分、重积分、曲线积分与曲面积分,级数理论则包括正项级数理论和一般项级数理论、傅里叶级数以及无穷乘积等内容。

我写这份笔记的主要参考书是前苏联数学家菲赫金哥尔茨的《微积分学教程》(高等教育出版社),这部书篇幅巨大(三卷本),内容丰富,除了微积分的理论体系之外,还包含大量的例子和材料,那些例子并不仅仅是单纯的例题,而是微积分理论中由众多数学家所推导出的大量重要的结果,而应用材料则囊括了数学上的、物理上的、工程学上的、天文学上的,几乎遍及微积分的每一个角落,所以此书取材的广度和深度,都让人叹为观止。该书另一个突出特点是,在严格性和易读性之间,把握了一个极好的平衡,在不失严格性的同时,又保持了极好的可读性,所以该书无论是用于自学,还是用于教学,都具有极大的参考价值。

之所以写作这份笔记,是因为上大学期间贪玩,微积分学的非常粗糙,然而像这样的理论体系,如果不仔细的弄清楚它的每一个细节,将永远只是一个模糊的印象,这份笔记的意义便在于此,远期也可以考虑把它写成一份可以供他人自学的材料。

致谢也是不能缺少的,首先要感谢的是本书参考文献的作者们,是他们让我接触到了这么多精彩的数学内容,当然还有一些书没有在参考文献之列(其实是我也想不起来是从哪儿看到的了),也一并感谢。还要感谢的是《计算机程序设计艺术》一书的作者 Donald E. Knuth,他开发的\TeX 排版系统以及由之发展而来的 \LaTeX 系统,使得我排版自己的书籍成为可能,还有编辑器 Emacs 的作者 Richard Stallman,这个编辑器所带来的强大的功能和编辑体验对我完成这份笔记功不可没。最后还要感谢我的夫人和女儿,女儿的降生给了我们这个家庭前所未有的欢乐,我对她的信心是在她的学生生涯,数学学科不至于成为她的拦路虎。夫人在照顾女儿上的付出才让我得以有精力来完成这份笔记。还有我的父母和哥哥。

\vspace{1.5cm}

\hfill zhcosin<zhcosin@163.com> \hspace{0.2em}

\hfill 2017-05-25 于成都华阳 \hspace{1.5em}


%%% Local Variables:
%%% mode: latex
%%% TeX-master: "book"
%%% End:
